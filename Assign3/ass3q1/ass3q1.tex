
% Default to the notebook output style

    


% Inherit from the specified cell style.




    
\documentclass[a4paper]{article}

    
    
    \usepackage[T1]{fontenc}
    % Nicer default font (+ math font) than Computer Modern for most use cases
    \usepackage{mathpazo}

    % Basic figure setup, for now with no caption control since it's done
    % automatically by Pandoc (which extracts ![](path) syntax from Markdown).
    \usepackage{graphicx}
    % We will generate all images so they have a width \maxwidth. This means
    % that they will get their normal width if they fit onto the page, but
    % are scaled down if they would overflow the margins.
    \makeatletter
    \def\maxwidth{\ifdim\Gin@nat@width>\linewidth\linewidth
    \else\Gin@nat@width\fi}
    \makeatother
    \let\Oldincludegraphics\includegraphics
    % Set max figure width to be 80% of text width, for now hardcoded.
    \renewcommand{\includegraphics}[1]{\Oldincludegraphics[width=.8\maxwidth]{#1}}
    % Ensure that by default, figures have no caption (until we provide a
    % proper Figure object with a Caption API and a way to capture that
    % in the conversion process - todo).
    \usepackage{caption}
    \DeclareCaptionLabelFormat{nolabel}{}
    \captionsetup{labelformat=nolabel}

    \usepackage{adjustbox} % Used to constrain images to a maximum size 
    \usepackage{xcolor} % Allow colors to be defined
    \usepackage{enumerate} % Needed for markdown enumerations to work
    \usepackage{geometry} % Used to adjust the document margins
    \usepackage{amsmath} % Equations
    \usepackage{amssymb} % Equations
    \usepackage{textcomp} % defines textquotesingle
    % Hack from http://tex.stackexchange.com/a/47451/13684:
    \AtBeginDocument{%
        \def\PYZsq{\textquotesingle}% Upright quotes in Pygmentized code
    }
    \usepackage{upquote} % Upright quotes for verbatim code
    \usepackage{eurosym} % defines \euro
    \usepackage[mathletters]{ucs} % Extended unicode (utf-8) support
    \usepackage[utf8x]{inputenc} % Allow utf-8 characters in the tex document
    \usepackage{fancyvrb} % verbatim replacement that allows latex
    \usepackage{grffile} % extends the file name processing of package graphics 
                         % to support a larger range 
    % The hyperref package gives us a pdf with properly built
    % internal navigation ('pdf bookmarks' for the table of contents,
    % internal cross-reference links, web links for URLs, etc.)
    \usepackage{hyperref}
    \usepackage{longtable} % longtable support required by pandoc >1.10
    \usepackage{booktabs}  % table support for pandoc > 1.12.2
    \usepackage[inline]{enumitem} % IRkernel/repr support (it uses the enumerate* environment)
    \usepackage[normalem]{ulem} % ulem is needed to support strikethroughs (\sout)
                                % normalem makes italics be italics, not underlines
    

    
    
    % Colors for the hyperref package
    \definecolor{urlcolor}{rgb}{0,.145,.698}
    \definecolor{linkcolor}{rgb}{.71,0.21,0.01}
    \definecolor{citecolor}{rgb}{.12,.54,.11}

    % ANSI colors
    \definecolor{ansi-black}{HTML}{3E424D}
    \definecolor{ansi-black-intense}{HTML}{282C36}
    \definecolor{ansi-red}{HTML}{E75C58}
    \definecolor{ansi-red-intense}{HTML}{B22B31}
    \definecolor{ansi-green}{HTML}{00A250}
    \definecolor{ansi-green-intense}{HTML}{007427}
    \definecolor{ansi-yellow}{HTML}{DDB62B}
    \definecolor{ansi-yellow-intense}{HTML}{B27D12}
    \definecolor{ansi-blue}{HTML}{208FFB}
    \definecolor{ansi-blue-intense}{HTML}{0065CA}
    \definecolor{ansi-magenta}{HTML}{D160C4}
    \definecolor{ansi-magenta-intense}{HTML}{A03196}
    \definecolor{ansi-cyan}{HTML}{60C6C8}
    \definecolor{ansi-cyan-intense}{HTML}{258F8F}
    \definecolor{ansi-white}{HTML}{C5C1B4}
    \definecolor{ansi-white-intense}{HTML}{A1A6B2}

    % commands and environments needed by pandoc snippets
    % extracted from the output of `pandoc -s`
    \providecommand{\tightlist}{%
      \setlength{\itemsep}{0pt}\setlength{\parskip}{0pt}}
    \DefineVerbatimEnvironment{Highlighting}{Verbatim}{commandchars=\\\{\}}
    % Add ',fontsize=\small' for more characters per line
    \newenvironment{Shaded}{}{}
    \newcommand{\KeywordTok}[1]{\textcolor[rgb]{0.00,0.44,0.13}{\textbf{{#1}}}}
    \newcommand{\DataTypeTok}[1]{\textcolor[rgb]{0.56,0.13,0.00}{{#1}}}
    \newcommand{\DecValTok}[1]{\textcolor[rgb]{0.25,0.63,0.44}{{#1}}}
    \newcommand{\BaseNTok}[1]{\textcolor[rgb]{0.25,0.63,0.44}{{#1}}}
    \newcommand{\FloatTok}[1]{\textcolor[rgb]{0.25,0.63,0.44}{{#1}}}
    \newcommand{\CharTok}[1]{\textcolor[rgb]{0.25,0.44,0.63}{{#1}}}
    \newcommand{\StringTok}[1]{\textcolor[rgb]{0.25,0.44,0.63}{{#1}}}
    \newcommand{\CommentTok}[1]{\textcolor[rgb]{0.38,0.63,0.69}{\textit{{#1}}}}
    \newcommand{\OtherTok}[1]{\textcolor[rgb]{0.00,0.44,0.13}{{#1}}}
    \newcommand{\AlertTok}[1]{\textcolor[rgb]{1.00,0.00,0.00}{\textbf{{#1}}}}
    \newcommand{\FunctionTok}[1]{\textcolor[rgb]{0.02,0.16,0.49}{{#1}}}
    \newcommand{\RegionMarkerTok}[1]{{#1}}
    \newcommand{\ErrorTok}[1]{\textcolor[rgb]{1.00,0.00,0.00}{\textbf{{#1}}}}
    \newcommand{\NormalTok}[1]{{#1}}
    
    % Additional commands for more recent versions of Pandoc
    \newcommand{\ConstantTok}[1]{\textcolor[rgb]{0.53,0.00,0.00}{{#1}}}
    \newcommand{\SpecialCharTok}[1]{\textcolor[rgb]{0.25,0.44,0.63}{{#1}}}
    \newcommand{\VerbatimStringTok}[1]{\textcolor[rgb]{0.25,0.44,0.63}{{#1}}}
    \newcommand{\SpecialStringTok}[1]{\textcolor[rgb]{0.73,0.40,0.53}{{#1}}}
    \newcommand{\ImportTok}[1]{{#1}}
    \newcommand{\DocumentationTok}[1]{\textcolor[rgb]{0.73,0.13,0.13}{\textit{{#1}}}}
    \newcommand{\AnnotationTok}[1]{\textcolor[rgb]{0.38,0.63,0.69}{\textbf{\textit{{#1}}}}}
    \newcommand{\CommentVarTok}[1]{\textcolor[rgb]{0.38,0.63,0.69}{\textbf{\textit{{#1}}}}}
    \newcommand{\VariableTok}[1]{\textcolor[rgb]{0.10,0.09,0.49}{{#1}}}
    \newcommand{\ControlFlowTok}[1]{\textcolor[rgb]{0.00,0.44,0.13}{\textbf{{#1}}}}
    \newcommand{\OperatorTok}[1]{\textcolor[rgb]{0.40,0.40,0.40}{{#1}}}
    \newcommand{\BuiltInTok}[1]{{#1}}
    \newcommand{\ExtensionTok}[1]{{#1}}
    \newcommand{\PreprocessorTok}[1]{\textcolor[rgb]{0.74,0.48,0.00}{{#1}}}
    \newcommand{\AttributeTok}[1]{\textcolor[rgb]{0.49,0.56,0.16}{{#1}}}
    \newcommand{\InformationTok}[1]{\textcolor[rgb]{0.38,0.63,0.69}{\textbf{\textit{{#1}}}}}
    \newcommand{\WarningTok}[1]{\textcolor[rgb]{0.38,0.63,0.69}{\textbf{\textit{{#1}}}}}
    
    
    % Define a nice break command that doesn't care if a line doesn't already
    % exist.
    \def\br{\hspace*{\fill} \\* }
    % Math Jax compatability definitions
    \def\gt{>}
    \def\lt{<}
    % Document parameters
    \title{Fourier Approximations \\ Assignment 3}
    \author{Rohithram R, EE16B031 \\ B.Tech Electrical Engineering, IIT Madras}
    \date{\today \\ First created on February 13,2018}	
    
    
    
    

    % Pygments definitions
    
\makeatletter
\def\PY@reset{\let\PY@it=\relax \let\PY@bf=\relax%
    \let\PY@ul=\relax \let\PY@tc=\relax%
    \let\PY@bc=\relax \let\PY@ff=\relax}
\def\PY@tok#1{\csname PY@tok@#1\endcsname}
\def\PY@toks#1+{\ifx\relax#1\empty\else%
    \PY@tok{#1}\expandafter\PY@toks\fi}
\def\PY@do#1{\PY@bc{\PY@tc{\PY@ul{%
    \PY@it{\PY@bf{\PY@ff{#1}}}}}}}
\def\PY#1#2{\PY@reset\PY@toks#1+\relax+\PY@do{#2}}

\expandafter\def\csname PY@tok@w\endcsname{\def\PY@tc##1{\textcolor[rgb]{0.73,0.73,0.73}{##1}}}
\expandafter\def\csname PY@tok@c\endcsname{\let\PY@it=\textit\def\PY@tc##1{\textcolor[rgb]{0.25,0.50,0.50}{##1}}}
\expandafter\def\csname PY@tok@cp\endcsname{\def\PY@tc##1{\textcolor[rgb]{0.74,0.48,0.00}{##1}}}
\expandafter\def\csname PY@tok@k\endcsname{\let\PY@bf=\textbf\def\PY@tc##1{\textcolor[rgb]{0.00,0.50,0.00}{##1}}}
\expandafter\def\csname PY@tok@kp\endcsname{\def\PY@tc##1{\textcolor[rgb]{0.00,0.50,0.00}{##1}}}
\expandafter\def\csname PY@tok@kt\endcsname{\def\PY@tc##1{\textcolor[rgb]{0.69,0.00,0.25}{##1}}}
\expandafter\def\csname PY@tok@o\endcsname{\def\PY@tc##1{\textcolor[rgb]{0.40,0.40,0.40}{##1}}}
\expandafter\def\csname PY@tok@ow\endcsname{\let\PY@bf=\textbf\def\PY@tc##1{\textcolor[rgb]{0.67,0.13,1.00}{##1}}}
\expandafter\def\csname PY@tok@nb\endcsname{\def\PY@tc##1{\textcolor[rgb]{0.00,0.50,0.00}{##1}}}
\expandafter\def\csname PY@tok@nf\endcsname{\def\PY@tc##1{\textcolor[rgb]{0.00,0.00,1.00}{##1}}}
\expandafter\def\csname PY@tok@nc\endcsname{\let\PY@bf=\textbf\def\PY@tc##1{\textcolor[rgb]{0.00,0.00,1.00}{##1}}}
\expandafter\def\csname PY@tok@nn\endcsname{\let\PY@bf=\textbf\def\PY@tc##1{\textcolor[rgb]{0.00,0.00,1.00}{##1}}}
\expandafter\def\csname PY@tok@ne\endcsname{\let\PY@bf=\textbf\def\PY@tc##1{\textcolor[rgb]{0.82,0.25,0.23}{##1}}}
\expandafter\def\csname PY@tok@nv\endcsname{\def\PY@tc##1{\textcolor[rgb]{0.10,0.09,0.49}{##1}}}
\expandafter\def\csname PY@tok@no\endcsname{\def\PY@tc##1{\textcolor[rgb]{0.53,0.00,0.00}{##1}}}
\expandafter\def\csname PY@tok@nl\endcsname{\def\PY@tc##1{\textcolor[rgb]{0.63,0.63,0.00}{##1}}}
\expandafter\def\csname PY@tok@ni\endcsname{\let\PY@bf=\textbf\def\PY@tc##1{\textcolor[rgb]{0.60,0.60,0.60}{##1}}}
\expandafter\def\csname PY@tok@na\endcsname{\def\PY@tc##1{\textcolor[rgb]{0.49,0.56,0.16}{##1}}}
\expandafter\def\csname PY@tok@nt\endcsname{\let\PY@bf=\textbf\def\PY@tc##1{\textcolor[rgb]{0.00,0.50,0.00}{##1}}}
\expandafter\def\csname PY@tok@nd\endcsname{\def\PY@tc##1{\textcolor[rgb]{0.67,0.13,1.00}{##1}}}
\expandafter\def\csname PY@tok@s\endcsname{\def\PY@tc##1{\textcolor[rgb]{0.73,0.13,0.13}{##1}}}
\expandafter\def\csname PY@tok@sd\endcsname{\let\PY@it=\textit\def\PY@tc##1{\textcolor[rgb]{0.73,0.13,0.13}{##1}}}
\expandafter\def\csname PY@tok@si\endcsname{\let\PY@bf=\textbf\def\PY@tc##1{\textcolor[rgb]{0.73,0.40,0.53}{##1}}}
\expandafter\def\csname PY@tok@se\endcsname{\let\PY@bf=\textbf\def\PY@tc##1{\textcolor[rgb]{0.73,0.40,0.13}{##1}}}
\expandafter\def\csname PY@tok@sr\endcsname{\def\PY@tc##1{\textcolor[rgb]{0.73,0.40,0.53}{##1}}}
\expandafter\def\csname PY@tok@ss\endcsname{\def\PY@tc##1{\textcolor[rgb]{0.10,0.09,0.49}{##1}}}
\expandafter\def\csname PY@tok@sx\endcsname{\def\PY@tc##1{\textcolor[rgb]{0.00,0.50,0.00}{##1}}}
\expandafter\def\csname PY@tok@m\endcsname{\def\PY@tc##1{\textcolor[rgb]{0.40,0.40,0.40}{##1}}}
\expandafter\def\csname PY@tok@gh\endcsname{\let\PY@bf=\textbf\def\PY@tc##1{\textcolor[rgb]{0.00,0.00,0.50}{##1}}}
\expandafter\def\csname PY@tok@gu\endcsname{\let\PY@bf=\textbf\def\PY@tc##1{\textcolor[rgb]{0.50,0.00,0.50}{##1}}}
\expandafter\def\csname PY@tok@gd\endcsname{\def\PY@tc##1{\textcolor[rgb]{0.63,0.00,0.00}{##1}}}
\expandafter\def\csname PY@tok@gi\endcsname{\def\PY@tc##1{\textcolor[rgb]{0.00,0.63,0.00}{##1}}}
\expandafter\def\csname PY@tok@gr\endcsname{\def\PY@tc##1{\textcolor[rgb]{1.00,0.00,0.00}{##1}}}
\expandafter\def\csname PY@tok@ge\endcsname{\let\PY@it=\textit}
\expandafter\def\csname PY@tok@gs\endcsname{\let\PY@bf=\textbf}
\expandafter\def\csname PY@tok@gp\endcsname{\let\PY@bf=\textbf\def\PY@tc##1{\textcolor[rgb]{0.00,0.00,0.50}{##1}}}
\expandafter\def\csname PY@tok@go\endcsname{\def\PY@tc##1{\textcolor[rgb]{0.53,0.53,0.53}{##1}}}
\expandafter\def\csname PY@tok@gt\endcsname{\def\PY@tc##1{\textcolor[rgb]{0.00,0.27,0.87}{##1}}}
\expandafter\def\csname PY@tok@err\endcsname{\def\PY@bc##1{\setlength{\fboxsep}{0pt}\fcolorbox[rgb]{1.00,0.00,0.00}{1,1,1}{\strut ##1}}}
\expandafter\def\csname PY@tok@kc\endcsname{\let\PY@bf=\textbf\def\PY@tc##1{\textcolor[rgb]{0.00,0.50,0.00}{##1}}}
\expandafter\def\csname PY@tok@kd\endcsname{\let\PY@bf=\textbf\def\PY@tc##1{\textcolor[rgb]{0.00,0.50,0.00}{##1}}}
\expandafter\def\csname PY@tok@kn\endcsname{\let\PY@bf=\textbf\def\PY@tc##1{\textcolor[rgb]{0.00,0.50,0.00}{##1}}}
\expandafter\def\csname PY@tok@kr\endcsname{\let\PY@bf=\textbf\def\PY@tc##1{\textcolor[rgb]{0.00,0.50,0.00}{##1}}}
\expandafter\def\csname PY@tok@bp\endcsname{\def\PY@tc##1{\textcolor[rgb]{0.00,0.50,0.00}{##1}}}
\expandafter\def\csname PY@tok@fm\endcsname{\def\PY@tc##1{\textcolor[rgb]{0.00,0.00,1.00}{##1}}}
\expandafter\def\csname PY@tok@vc\endcsname{\def\PY@tc##1{\textcolor[rgb]{0.10,0.09,0.49}{##1}}}
\expandafter\def\csname PY@tok@vg\endcsname{\def\PY@tc##1{\textcolor[rgb]{0.10,0.09,0.49}{##1}}}
\expandafter\def\csname PY@tok@vi\endcsname{\def\PY@tc##1{\textcolor[rgb]{0.10,0.09,0.49}{##1}}}
\expandafter\def\csname PY@tok@vm\endcsname{\def\PY@tc##1{\textcolor[rgb]{0.10,0.09,0.49}{##1}}}
\expandafter\def\csname PY@tok@sa\endcsname{\def\PY@tc##1{\textcolor[rgb]{0.73,0.13,0.13}{##1}}}
\expandafter\def\csname PY@tok@sb\endcsname{\def\PY@tc##1{\textcolor[rgb]{0.73,0.13,0.13}{##1}}}
\expandafter\def\csname PY@tok@sc\endcsname{\def\PY@tc##1{\textcolor[rgb]{0.73,0.13,0.13}{##1}}}
\expandafter\def\csname PY@tok@dl\endcsname{\def\PY@tc##1{\textcolor[rgb]{0.73,0.13,0.13}{##1}}}
\expandafter\def\csname PY@tok@s2\endcsname{\def\PY@tc##1{\textcolor[rgb]{0.73,0.13,0.13}{##1}}}
\expandafter\def\csname PY@tok@sh\endcsname{\def\PY@tc##1{\textcolor[rgb]{0.73,0.13,0.13}{##1}}}
\expandafter\def\csname PY@tok@s1\endcsname{\def\PY@tc##1{\textcolor[rgb]{0.73,0.13,0.13}{##1}}}
\expandafter\def\csname PY@tok@mb\endcsname{\def\PY@tc##1{\textcolor[rgb]{0.40,0.40,0.40}{##1}}}
\expandafter\def\csname PY@tok@mf\endcsname{\def\PY@tc##1{\textcolor[rgb]{0.40,0.40,0.40}{##1}}}
\expandafter\def\csname PY@tok@mh\endcsname{\def\PY@tc##1{\textcolor[rgb]{0.40,0.40,0.40}{##1}}}
\expandafter\def\csname PY@tok@mi\endcsname{\def\PY@tc##1{\textcolor[rgb]{0.40,0.40,0.40}{##1}}}
\expandafter\def\csname PY@tok@il\endcsname{\def\PY@tc##1{\textcolor[rgb]{0.40,0.40,0.40}{##1}}}
\expandafter\def\csname PY@tok@mo\endcsname{\def\PY@tc##1{\textcolor[rgb]{0.40,0.40,0.40}{##1}}}
\expandafter\def\csname PY@tok@ch\endcsname{\let\PY@it=\textit\def\PY@tc##1{\textcolor[rgb]{0.25,0.50,0.50}{##1}}}
\expandafter\def\csname PY@tok@cm\endcsname{\let\PY@it=\textit\def\PY@tc##1{\textcolor[rgb]{0.25,0.50,0.50}{##1}}}
\expandafter\def\csname PY@tok@cpf\endcsname{\let\PY@it=\textit\def\PY@tc##1{\textcolor[rgb]{0.25,0.50,0.50}{##1}}}
\expandafter\def\csname PY@tok@c1\endcsname{\let\PY@it=\textit\def\PY@tc##1{\textcolor[rgb]{0.25,0.50,0.50}{##1}}}
\expandafter\def\csname PY@tok@cs\endcsname{\let\PY@it=\textit\def\PY@tc##1{\textcolor[rgb]{0.25,0.50,0.50}{##1}}}

\def\PYZbs{\char`\\}
\def\PYZus{\char`\_}
\def\PYZob{\char`\{}
\def\PYZcb{\char`\}}
\def\PYZca{\char`\^}
\def\PYZam{\char`\&}
\def\PYZlt{\char`\<}
\def\PYZgt{\char`\>}
\def\PYZsh{\char`\#}
\def\PYZpc{\char`\%}
\def\PYZdl{\char`\$}
\def\PYZhy{\char`\-}
\def\PYZsq{\char`\'}
\def\PYZdq{\char`\"}
\def\PYZti{\char`\~}
% for compatibility with earlier versions
\def\PYZat{@}
\def\PYZlb{[}
\def\PYZrb{]}
\makeatother


    % Exact colors from NB
    \definecolor{incolor}{rgb}{0.0, 0.0, 0.5}
    \definecolor{outcolor}{rgb}{0.545, 0.0, 0.0}



    
    % Prevent overflowing lines due to hard-to-break entities
    \sloppy 
    % Setup hyperref package
    \hypersetup{
      breaklinks=true,  % so long urls are correctly broken across lines
      colorlinks=true,
      urlcolor=urlcolor,
      linkcolor=linkcolor,
      citecolor=citecolor,
      }
    % Slightly bigger margins than the latex defaults
    
    \geometry{verbose,tmargin=1in,bmargin=1in,lmargin=1in,rmargin=1in}
    
    

    \begin{document}
    
    
    \maketitle
    
    \begin{abstract}    
	\end{abstract}
	This report presents a study of different methods of creating a  $tan^{-1}(x)$ from its integral definition of $\int_{0}^{x} dx/(1+t^{2})$ using scipy's quad function to integrate and other method is of numerical integration using trapezoidal rule,which can be used even for non-integrable functions to integrate it.
	And it also discusses the advantage of Vectorization of code compared to for loops and also on finding estimate errors when actual error is unknown i.e function is non-integrable by halving the stepsize till it reaches certain tolerance!.	
	\section{Introduction}
	This report discusses 7 tasks in python to find Fourier Approximations of two function $e^{x}$ and $\cos(\cos(x))$ from its integral definition and using Least Squares method.\\
	We will fit two functions, $e^{x}$ and $\cos(\cos(x))$ over the interval [0;2$\pi$) using the fourier series
 	\begin{equation}
    a_{0} + \sum\limits_{n=1}^{\infty} {{a_{n}\cos(nx_{i})+b_{n}\sin(nx_{i})}} \approx f(x_{i}) 
    \end{equation}
    	The equations used here to find the Fourier coefficients are as follows:
    \begin{equation}
         a_{0} = \frac{1}{2\pi}\int\limits_{0}^{2\pi} f(x)dx 
    \end{equation}
    \begin{equation}
         a_{n} = \frac{1}{\pi}\int\limits_{0}^{2\pi} f(x)\cos(nx)dx 
    \end{equation}
    \begin{equation}
         b_{n} = \frac{1}{\pi}\int\limits_{0}^{2\pi} f(x)\sin(nx)dx 
    \end{equation}
	
    \section{Python code}   
        
    \begin{Verbatim}[commandchars=\\\{\}]
{\color{incolor}In [{\color{incolor}1}]:} \PY{c+c1}{\PYZsh{} load libraries and set plot parameters}
        \PY{k+kn}{from} \PY{n+nn}{pylab} \PY{k}{import} \PY{o}{*}
        \PY{k+kn}{from} \PY{n+nn}{scipy}\PY{n+nn}{.}\PY{n+nn}{integrate} \PY{k}{import} \PY{n}{quad}
        \PY{o}{\PYZpc{}}\PY{k}{matplotlib} inline
        
        \PY{k+kn}{from} \PY{n+nn}{IPython}\PY{n+nn}{.}\PY{n+nn}{display} \PY{k}{import} \PY{n}{set\PYZus{}matplotlib\PYZus{}formats}
        \PY{n}{set\PYZus{}matplotlib\PYZus{}formats}\PY{p}{(}\PY{l+s+s1}{\PYZsq{}}\PY{l+s+s1}{pdf}\PY{l+s+s1}{\PYZsq{}}\PY{p}{,} \PY{l+s+s1}{\PYZsq{}}\PY{l+s+s1}{png}\PY{l+s+s1}{\PYZsq{}}\PY{p}{)}
        \PY{n}{plt}\PY{o}{.}\PY{n}{rcParams}\PY{p}{[}\PY{l+s+s1}{\PYZsq{}}\PY{l+s+s1}{savefig.dpi}\PY{l+s+s1}{\PYZsq{}}\PY{p}{]} \PY{o}{=} \PY{l+m+mi}{75}
        
        \PY{n}{plt}\PY{o}{.}\PY{n}{rcParams}\PY{p}{[}\PY{l+s+s1}{\PYZsq{}}\PY{l+s+s1}{figure.autolayout}\PY{l+s+s1}{\PYZsq{}}\PY{p}{]} \PY{o}{=} \PY{k+kc}{False}
        \PY{n}{plt}\PY{o}{.}\PY{n}{rcParams}\PY{p}{[}\PY{l+s+s1}{\PYZsq{}}\PY{l+s+s1}{figure.figsize}\PY{l+s+s1}{\PYZsq{}}\PY{p}{]} \PY{o}{=} \PY{l+m+mi}{10}\PY{p}{,} \PY{l+m+mi}{6}
        \PY{n}{plt}\PY{o}{.}\PY{n}{rcParams}\PY{p}{[}\PY{l+s+s1}{\PYZsq{}}\PY{l+s+s1}{axes.labelsize}\PY{l+s+s1}{\PYZsq{}}\PY{p}{]} \PY{o}{=} \PY{l+m+mi}{18}
        \PY{n}{plt}\PY{o}{.}\PY{n}{rcParams}\PY{p}{[}\PY{l+s+s1}{\PYZsq{}}\PY{l+s+s1}{axes.titlesize}\PY{l+s+s1}{\PYZsq{}}\PY{p}{]} \PY{o}{=} \PY{l+m+mi}{20}
        \PY{n}{plt}\PY{o}{.}\PY{n}{rcParams}\PY{p}{[}\PY{l+s+s1}{\PYZsq{}}\PY{l+s+s1}{font.size}\PY{l+s+s1}{\PYZsq{}}\PY{p}{]} \PY{o}{=} \PY{l+m+mi}{16}
        \PY{n}{plt}\PY{o}{.}\PY{n}{rcParams}\PY{p}{[}\PY{l+s+s1}{\PYZsq{}}\PY{l+s+s1}{lines.linewidth}\PY{l+s+s1}{\PYZsq{}}\PY{p}{]} \PY{o}{=} \PY{l+m+mf}{2.0}
        \PY{n}{plt}\PY{o}{.}\PY{n}{rcParams}\PY{p}{[}\PY{l+s+s1}{\PYZsq{}}\PY{l+s+s1}{lines.markersize}\PY{l+s+s1}{\PYZsq{}}\PY{p}{]} \PY{o}{=} \PY{l+m+mi}{4}
        \PY{n}{plt}\PY{o}{.}\PY{n}{rcParams}\PY{p}{[}\PY{l+s+s1}{\PYZsq{}}\PY{l+s+s1}{legend.fontsize}\PY{l+s+s1}{\PYZsq{}}\PY{p}{]} \PY{o}{=} \PY{l+m+mi}{14}
        \PY{n}{plt}\PY{o}{.}\PY{n}{rcParams}\PY{p}{[}\PY{l+s+s1}{\PYZsq{}}\PY{l+s+s1}{legend.numpoints}\PY{l+s+s1}{\PYZsq{}}\PY{p}{]} \PY{o}{=} \PY{l+m+mi}{2}
        \PY{n}{plt}\PY{o}{.}\PY{n}{rcParams}\PY{p}{[}\PY{l+s+s1}{\PYZsq{}}\PY{l+s+s1}{legend.loc}\PY{l+s+s1}{\PYZsq{}}\PY{p}{]} \PY{o}{=} \PY{l+s+s1}{\PYZsq{}}\PY{l+s+s1}{best}\PY{l+s+s1}{\PYZsq{}}
        \PY{n}{plt}\PY{o}{.}\PY{n}{rcParams}\PY{p}{[}\PY{l+s+s1}{\PYZsq{}}\PY{l+s+s1}{legend.fancybox}\PY{l+s+s1}{\PYZsq{}}\PY{p}{]} \PY{o}{=} \PY{k+kc}{True}
        \PY{n}{plt}\PY{o}{.}\PY{n}{rcParams}\PY{p}{[}\PY{l+s+s1}{\PYZsq{}}\PY{l+s+s1}{legend.shadow}\PY{l+s+s1}{\PYZsq{}}\PY{p}{]} \PY{o}{=} \PY{k+kc}{True}
        \PY{n}{plt}\PY{o}{.}\PY{n}{rcParams}\PY{p}{[}\PY{l+s+s1}{\PYZsq{}}\PY{l+s+s1}{text.usetex}\PY{l+s+s1}{\PYZsq{}}\PY{p}{]} \PY{o}{=} \PY{k+kc}{True}
        \PY{n}{plt}\PY{o}{.}\PY{n}{rcParams}\PY{p}{[}\PY{l+s+s1}{\PYZsq{}}\PY{l+s+s1}{font.family}\PY{l+s+s1}{\PYZsq{}}\PY{p}{]} \PY{o}{=} \PY{l+s+s2}{\PYZdq{}}\PY{l+s+s2}{serif}\PY{l+s+s2}{\PYZdq{}}
        \PY{n}{plt}\PY{o}{.}\PY{n}{rcParams}\PY{p}{[}\PY{l+s+s1}{\PYZsq{}}\PY{l+s+s1}{font.serif}\PY{l+s+s1}{\PYZsq{}}\PY{p}{]} \PY{o}{=} \PY{l+s+s2}{\PYZdq{}}\PY{l+s+s2}{cm}\PY{l+s+s2}{\PYZdq{}}
        \PY{n}{plt}\PY{o}{.}\PY{n}{rcParams}\PY{p}{[}\PY{l+s+s1}{\PYZsq{}}\PY{l+s+s1}{text.latex.preamble}\PY{l+s+s1}{\PYZsq{}}\PY{p}{]} \PY{o}{=} \PY{l+s+sa}{r}\PY{l+s+s2}{\PYZdq{}}\PY{l+s+s2}{\PYZbs{}}\PY{l+s+s2}{usepackage}\PY{l+s+si}{\PYZob{}subdepth\PYZcb{}}\PY{l+s+s2}{, }\PY{l+s+s2}{\PYZbs{}}\PY{l+s+s2}{usepackage}\PY{l+s+si}{\PYZob{}type1cm\PYZcb{}}\PY{l+s+s2}{\PYZdq{}}
\end{Verbatim}


    \subsection{Question 1}\label{question-1}

\begin{itemize}
\tightlist
\item
  Define Python functions for the two functions \(e^{x}\) and
  \(\cos(\cos(x))\) which return a vector (or scalar) value.
\item
  Plot the functions over the interval {[}−2\(\pi\),\(4\pi\)).
\item
  Discuss periodicity of both functions
\item
  Plot the expected functions from fourier series
\end{itemize}

    \begin{Verbatim}[commandchars=\\\{\}]
{\color{incolor}In [{\color{incolor}2}]:} \PY{c+c1}{\PYZsh{}Functions for \PYZdl{}e\PYZca{}\PYZob{}x\PYZcb{}\PYZdl{} and \PYZdl{}\PYZbs{}cos(\PYZbs{}cos(x))\PYZdl{} is defined}
        \PY{k}{def} \PY{n+nf}{fexp}\PY{p}{(}\PY{n}{x}\PY{p}{)}\PY{p}{:}        
            \PY{k}{return} \PY{n}{exp}\PY{p}{(}\PY{n}{x}\PY{p}{)}
        
        \PY{k}{def} \PY{n+nf}{fcoscos}\PY{p}{(}\PY{n}{x}\PY{p}{)}\PY{p}{:}
            \PY{k}{return} \PY{n}{cos}\PY{p}{(}\PY{n}{cos}\PY{p}{(}\PY{n}{x}\PY{p}{)}\PY{p}{)}
\end{Verbatim}


    \begin{Verbatim}[commandchars=\\\{\}]
{\color{incolor}In [{\color{incolor}3}]:} \PY{n}{x} \PY{o}{=} \PY{n}{linspace}\PY{p}{(}\PY{o}{\PYZhy{}}\PY{l+m+mi}{2}\PY{o}{*}\PY{n}{pi}\PY{p}{,} \PY{l+m+mi}{4}\PY{o}{*}\PY{n}{pi}\PY{p}{,}\PY{l+m+mi}{400}\PY{p}{)}  
        \PY{c+c1}{\PYZsh{}Period of function created using fourier coefficients will be 2pi}
        \PY{n}{period} \PY{o}{=} \PY{l+m+mi}{2}\PY{o}{*}\PY{n}{pi} 
        \PY{n}{exp\PYZus{}fn} \PY{o}{=} \PY{n}{fexp}\PY{p}{(}\PY{n}{x}\PY{p}{)}               \PY{c+c1}{\PYZsh{}finding exp(x) for all values in x vector}
        \PY{n}{cos\PYZus{}fn} \PY{o}{=} \PY{n}{fcoscos}\PY{p}{(}\PY{n}{x}\PY{p}{)}            \PY{c+c1}{\PYZsh{}finding cos(cos(x)) for all values in x vector}
\end{Verbatim}


    \begin{Verbatim}[commandchars=\\\{\}]
{\color{incolor}In [{\color{incolor}4}]:} \PY{c+c1}{\PYZsh{}Plotting original function vs expected function for exp(x)}
        \PY{n}{fig1} \PY{o}{=} \PY{n}{figure}\PY{p}{(}\PY{p}{)}
        \PY{n}{ax1} \PY{o}{=} \PY{n}{fig1}\PY{o}{.}\PY{n}{add\PYZus{}subplot}\PY{p}{(}\PY{l+m+mi}{111}\PY{p}{)}
        \PY{n}{ax1}\PY{o}{.}\PY{n}{semilogy}\PY{p}{(}\PY{n}{x}\PY{p}{,}\PY{n}{exp\PYZus{}fn}\PY{p}{,}\PY{l+s+s1}{\PYZsq{}}\PY{l+s+s1}{k}\PY{l+s+s1}{\PYZsq{}}\PY{p}{,}\PY{n}{label}\PY{o}{=}\PY{l+s+s2}{\PYZdq{}}\PY{l+s+s2}{Original Function}\PY{l+s+s2}{\PYZdq{}}\PY{p}{)}
        \PY{c+c1}{\PYZsh{}plotting expected function by dividing the x by period and giving remainder as}
        \PY{c+c1}{\PYZsh{}input to the function, so that x values repeat after given period.}
        \PY{n}{ax1}\PY{o}{.}\PY{n}{semilogy}\PY{p}{(}\PY{n}{x}\PY{p}{,}\PY{n}{fexp}\PY{p}{(}\PY{n}{x}\PY{o}{\PYZpc{}}\PY{k}{period}),\PYZsq{}\PYZhy{}\PYZhy{}\PYZsq{},label=\PYZdq{}Expected Function from fourier series\PYZdq{})
        \PY{n}{ax1}\PY{o}{.}\PY{n}{legend}\PY{p}{(}\PY{p}{)}
        \PY{n}{title}\PY{p}{(}\PY{l+s+s2}{\PYZdq{}}\PY{l+s+s2}{Plot of \PYZdl{}e\PYZca{}}\PY{l+s+si}{\PYZob{}x\PYZcb{}}\PY{l+s+s2}{\PYZdl{}}\PY{l+s+s2}{\PYZdq{}}\PY{p}{)}
        \PY{n}{xlabel}\PY{p}{(}\PY{l+s+s2}{\PYZdq{}}\PY{l+s+s2}{x}\PY{l+s+s2}{\PYZdq{}}\PY{p}{)}
        \PY{n}{ylabel}\PY{p}{(}\PY{l+s+s2}{\PYZdq{}}\PY{l+s+s2}{\PYZdl{}e\PYZca{}}\PY{l+s+si}{\PYZob{}x\PYZcb{}}\PY{l+s+s2}{\PYZdl{}}\PY{l+s+s2}{\PYZdq{}}\PY{p}{)}
        \PY{n}{grid}\PY{p}{(}\PY{p}{)}
        \PY{n}{savefig}\PY{p}{(}\PY{l+s+s2}{\PYZdq{}}\PY{l+s+s2}{Figure1.jpg}\PY{l+s+s2}{\PYZdq{}}\PY{p}{)}
\end{Verbatim}


    \begin{center}
    \adjustimage{max size={0.9\linewidth}{0.9\paperheight}}{output_4_0.pdf}
    \end{center}
    { \hspace*{\fill} \\}
    
    \begin{Verbatim}[commandchars=\\\{\}]
{\color{incolor}In [{\color{incolor}5}]:} \PY{c+c1}{\PYZsh{}Plotting original function vs expected function for cos(cos((x)))}
        \PY{n}{fig2} \PY{o}{=} \PY{n}{figure}\PY{p}{(}\PY{p}{)}
        \PY{n}{ax2} \PY{o}{=} \PY{n}{fig2}\PY{o}{.}\PY{n}{add\PYZus{}subplot}\PY{p}{(}\PY{l+m+mi}{111}\PY{p}{)}
        \PY{n}{ax2}\PY{o}{.}\PY{n}{plot}\PY{p}{(}\PY{n}{x}\PY{p}{,}\PY{n}{cos\PYZus{}fn}\PY{p}{,}\PY{l+s+s1}{\PYZsq{}}\PY{l+s+s1}{b}\PY{l+s+s1}{\PYZsq{}}\PY{p}{,}\PY{n}{linewidth}\PY{o}{=}\PY{l+m+mi}{4}\PY{p}{,}\PY{n}{label}\PY{o}{=}\PY{l+s+s2}{\PYZdq{}}\PY{l+s+s2}{Original Function}\PY{l+s+s2}{\PYZdq{}}\PY{p}{)}
        \PY{c+c1}{\PYZsh{}plotting expected function by dividing the x by period and giving remainder as}
        \PY{c+c1}{\PYZsh{}input to the function, so that x values repeat after given period.}
        \PY{n}{ax2}\PY{o}{.}\PY{n}{semilogy}\PY{p}{(}\PY{n}{x}\PY{p}{,}\PY{n}{fcoscos}\PY{p}{(}\PY{n}{x}\PY{o}{\PYZpc{}}\PY{k}{period}),\PYZsq{}y\PYZhy{}\PYZhy{}\PYZsq{},label=\PYZdq{}Expected Function from fourier series\PYZdq{})
        \PY{n}{ax2}\PY{o}{.}\PY{n}{legend}\PY{p}{(}\PY{n}{loc}\PY{o}{=}\PY{l+s+s1}{\PYZsq{}}\PY{l+s+s1}{upper right}\PY{l+s+s1}{\PYZsq{}}\PY{p}{)}
        \PY{n}{title}\PY{p}{(}\PY{l+s+s2}{\PYZdq{}}\PY{l+s+s2}{Plot of \PYZdl{}}\PY{l+s+s2}{\PYZbs{}}\PY{l+s+s2}{cos(}\PY{l+s+s2}{\PYZbs{}}\PY{l+s+s2}{cos(x))\PYZdl{}}\PY{l+s+s2}{\PYZdq{}}\PY{p}{)}
        \PY{n}{xlabel}\PY{p}{(}\PY{l+s+s2}{\PYZdq{}}\PY{l+s+s2}{x}\PY{l+s+s2}{\PYZdq{}}\PY{p}{)}
        \PY{n}{ylabel}\PY{p}{(}\PY{l+s+s2}{\PYZdq{}}\PY{l+s+s2}{\PYZdl{}}\PY{l+s+s2}{\PYZbs{}}\PY{l+s+s2}{cos(}\PY{l+s+s2}{\PYZbs{}}\PY{l+s+s2}{cos(x))\PYZdl{}}\PY{l+s+s2}{\PYZdq{}}\PY{p}{)}
        \PY{n}{grid}\PY{p}{(}\PY{p}{)}
        \PY{n}{savefig}\PY{p}{(}\PY{l+s+s2}{\PYZdq{}}\PY{l+s+s2}{Figure2.jpg}\PY{l+s+s2}{\PYZdq{}}\PY{p}{)}
        \PY{n}{show}\PY{p}{(}\PY{p}{)}
\end{Verbatim}


    \begin{center}
    \adjustimage{max size={0.9\linewidth}{0.9\paperheight}}{output_5_0.pdf}
    \end{center}
    { \hspace*{\fill} \\}
    
    \subsubsection{Results and Discussion :}\label{results-and-discussion}

\begin{itemize}
\tightlist
\item
  We observe that \(e^{x}\) is not periodic, whereas \(\cos(\cos(x))\)
  is periodic as the expected and original function matched for the
  latter but not for \(e^{x}\).
\item
  Period of \(\cos(\cos(x))\) is \(2\pi\) as we observe from graph and
  \(e^{x}\) monotously increasing hence not periodic.
\item
  We get expected function by:

  \begin{itemize}
  \tightlist
  \item
    plotting expected function by dividing the x by period and giving
    remainder as input to the function, so that x values repeat after
    given period.
  \item
    That is f(x\%period) is now the expected periodic function from
    fourier series.
  \end{itemize}
\end{itemize}

    \subsection{Question 2}\label{question-2}

\begin{itemize}
\tightlist
\item
  Obtain the first 51 coefficients i.e \(a_{0}, a_{1}, b_{1},....\) for
  \(e^{x}\) and \(\cos(\cos(x))\) using scipy quad function
\item
  And to calculate the function using those coefficients and comparing
  with original funcitons graphically.
\end{itemize}

    \begin{Verbatim}[commandchars=\\\{\}]
{\color{incolor}In [{\color{incolor}6}]:} \PY{c+c1}{\PYZsh{}function to calculate }
        \PY{k}{def} \PY{n+nf}{fourier\PYZus{}an}\PY{p}{(}\PY{n}{x}\PY{p}{,}\PY{n}{k}\PY{p}{,}\PY{n}{f}\PY{p}{)}\PY{p}{:}
            \PY{k}{return} \PY{n}{f}\PY{p}{(}\PY{n}{x}\PY{p}{)}\PY{o}{*}\PY{n}{cos}\PY{p}{(}\PY{n}{k}\PY{o}{*}\PY{n}{x}\PY{p}{)}
        
        \PY{k}{def} \PY{n+nf}{fourier\PYZus{}bn}\PY{p}{(}\PY{n}{x}\PY{p}{,}\PY{n}{k}\PY{p}{,}\PY{n}{f}\PY{p}{)}\PY{p}{:}
            \PY{k}{return} \PY{n}{f}\PY{p}{(}\PY{n}{x}\PY{p}{)}\PY{o}{*}\PY{n}{sin}\PY{p}{(}\PY{n}{k}\PY{o}{*}\PY{n}{x}\PY{p}{)}
\end{Verbatim}


    \begin{Verbatim}[commandchars=\\\{\}]
{\color{incolor}In [{\color{incolor}7}]:} \PY{c+c1}{\PYZsh{}function to find the fourier coefficients taking function \PYZsq{}f\PYZsq{} as argument.}
        \PY{k}{def} \PY{n+nf}{find\PYZus{}coeff}\PY{p}{(}\PY{n}{f}\PY{p}{)}\PY{p}{:}
            
            \PY{n}{coeff} \PY{o}{=} \PY{p}{[}\PY{p}{]}
            \PY{n}{coeff}\PY{o}{.}\PY{n}{append}\PY{p}{(}\PY{p}{(}\PY{n}{quad}\PY{p}{(}\PY{n}{f}\PY{p}{,}\PY{l+m+mi}{0}\PY{p}{,}\PY{l+m+mi}{2}\PY{o}{*}\PY{n}{pi}\PY{p}{)}\PY{p}{[}\PY{l+m+mi}{0}\PY{p}{]}\PY{p}{)}\PY{o}{/}\PY{p}{(}\PY{l+m+mi}{2}\PY{o}{*}\PY{n}{pi}\PY{p}{)}\PY{p}{)}
            \PY{k}{for} \PY{n}{i} \PY{o+ow}{in} \PY{n+nb}{range}\PY{p}{(}\PY{l+m+mi}{1}\PY{p}{,}\PY{l+m+mi}{26}\PY{p}{)}\PY{p}{:}
                \PY{n}{coeff}\PY{o}{.}\PY{n}{append}\PY{p}{(}\PY{p}{(}\PY{n}{quad}\PY{p}{(}\PY{n}{fourier\PYZus{}an}\PY{p}{,}\PY{l+m+mi}{0}\PY{p}{,}\PY{l+m+mi}{2}\PY{o}{*}\PY{n}{pi}\PY{p}{,}\PY{n}{args}\PY{o}{=}\PY{p}{(}\PY{n}{i}\PY{p}{,}\PY{n}{f}\PY{p}{)}\PY{p}{)}\PY{p}{[}\PY{l+m+mi}{0}\PY{p}{]}\PY{p}{)}\PY{o}{/}\PY{n}{pi}\PY{p}{)}
                \PY{n}{coeff}\PY{o}{.}\PY{n}{append}\PY{p}{(}\PY{p}{(}\PY{n}{quad}\PY{p}{(}\PY{n}{fourier\PYZus{}bn}\PY{p}{,}\PY{l+m+mi}{0}\PY{p}{,}\PY{l+m+mi}{2}\PY{o}{*}\PY{n}{pi}\PY{p}{,}\PY{n}{args}\PY{o}{=}\PY{p}{(}\PY{n}{i}\PY{p}{,}\PY{n}{f}\PY{p}{)}\PY{p}{)}\PY{p}{[}\PY{l+m+mi}{0}\PY{p}{]}\PY{p}{)}\PY{o}{/}\PY{n}{pi}\PY{p}{)}
                
            \PY{k}{return} \PY{n}{coeff}
\end{Verbatim}


    \begin{Verbatim}[commandchars=\\\{\}]
{\color{incolor}In [{\color{incolor}8}]:} \PY{c+c1}{\PYZsh{}function to create \PYZsq{}A\PYZsq{} matrix for calculating function back from coefficients}
        \PY{c+c1}{\PYZsh{} with no\PYZus{}of rows, columns and vector x as arguments}
        \PY{k}{def} \PY{n+nf}{createAmatrix}\PY{p}{(}\PY{n}{nrow}\PY{p}{,}\PY{n}{ncol}\PY{p}{,}\PY{n}{x}\PY{p}{)}\PY{p}{:}
            \PY{n}{A} \PY{o}{=} \PY{n}{zeros}\PY{p}{(}\PY{p}{(}\PY{n}{nrow}\PY{p}{,}\PY{n}{ncol}\PY{p}{)}\PY{p}{)} \PY{c+c1}{\PYZsh{} allocate space for A}
            \PY{n}{A}\PY{p}{[}\PY{p}{:}\PY{p}{,}\PY{l+m+mi}{0}\PY{p}{]}\PY{o}{=}\PY{l+m+mi}{1} \PY{c+c1}{\PYZsh{} col 1 is all ones}
            \PY{k}{for} \PY{n}{k} \PY{o+ow}{in} \PY{n+nb}{range}\PY{p}{(}\PY{l+m+mi}{1}\PY{p}{,}\PY{l+m+mi}{26}\PY{p}{)}\PY{p}{:}
                \PY{n}{A}\PY{p}{[}\PY{p}{:}\PY{p}{,}\PY{l+m+mi}{2}\PY{o}{*}\PY{n}{k}\PY{o}{\PYZhy{}}\PY{l+m+mi}{1}\PY{p}{]}\PY{o}{=}\PY{n}{cos}\PY{p}{(}\PY{n}{k}\PY{o}{*}\PY{n}{x}\PY{p}{)} \PY{c+c1}{\PYZsh{} cos(kx) column}
                \PY{n}{A}\PY{p}{[}\PY{p}{:}\PY{p}{,}\PY{l+m+mi}{2}\PY{o}{*}\PY{n}{k}\PY{p}{]}\PY{o}{=}\PY{n}{sin}\PY{p}{(}\PY{n}{k}\PY{o}{*}\PY{n}{x}\PY{p}{)} \PY{c+c1}{\PYZsh{} sin(kx) column}
            \PY{c+c1}{\PYZsh{}endfor}
            \PY{k}{return} \PY{n}{A}
\end{Verbatim}


    \begin{Verbatim}[commandchars=\\\{\}]
{\color{incolor}In [{\color{incolor}9}]:} \PY{c+c1}{\PYZsh{}Function to compute function from coefficients with argument as coefficient vector \PYZsq{}c\PYZsq{}}
        \PY{k}{def} \PY{n+nf}{computeFunctionfromCoeff}\PY{p}{(}\PY{n}{c}\PY{p}{)}\PY{p}{:}
            \PY{n}{A} \PY{o}{=} \PY{n}{createAmatrix}\PY{p}{(}\PY{l+m+mi}{400}\PY{p}{,}\PY{l+m+mi}{51}\PY{p}{,}\PY{n}{x}\PY{p}{)}
            \PY{n}{f\PYZus{}fourier} \PY{o}{=} \PY{n}{A}\PY{o}{.}\PY{n}{dot}\PY{p}{(}\PY{n}{c}\PY{p}{)}
            \PY{k}{return} \PY{n}{f\PYZus{}fourier}
\end{Verbatim}


    \begin{Verbatim}[commandchars=\\\{\}]
{\color{incolor}In [{\color{incolor}10}]:} \PY{c+c1}{\PYZsh{} Initialising empty lists to store coefficients for both functions}
         \PY{n}{exp\PYZus{}coeff} \PY{o}{=} \PY{p}{[}\PY{p}{]}               
         \PY{n}{coscos\PYZus{}coeff} \PY{o}{=} \PY{p}{[}\PY{p}{]}
         \PY{n}{exp\PYZus{}coeff1} \PY{o}{=} \PY{p}{[}\PY{p}{]}
         \PY{n}{coscos\PYZus{}coeff1} \PY{o}{=} \PY{p}{[}\PY{p}{]}
         
         \PY{n}{exp\PYZus{}coeff1} \PY{o}{=} \PY{n}{find\PYZus{}coeff}\PY{p}{(}\PY{n}{fexp}\PY{p}{)}
         \PY{n}{coscos\PYZus{}coeff1} \PY{o}{=} \PY{n}{find\PYZus{}coeff}\PY{p}{(}\PY{n}{fcoscos}\PY{p}{)}
         
         \PY{c+c1}{\PYZsh{} to store absolute value of coefficients}
         \PY{n}{exp\PYZus{}coeff} \PY{o}{=} \PY{n}{np}\PY{o}{.}\PY{n}{abs}\PY{p}{(}\PY{n}{exp\PYZus{}coeff1}\PY{p}{)}
         \PY{n}{coscos\PYZus{}coeff} \PY{o}{=} \PY{n}{np}\PY{o}{.}\PY{n}{abs}\PY{p}{(}\PY{n}{coscos\PYZus{}coeff1}\PY{p}{)}
         
         \PY{c+c1}{\PYZsh{} Computing function using fourier coeff}
         \PY{n}{fexp\PYZus{}fourier} \PY{o}{=} \PY{n}{computeFunctionfromCoeff}\PY{p}{(}\PY{n}{exp\PYZus{}coeff1}\PY{p}{)}
         \PY{n}{fcoscos\PYZus{}fourier} \PY{o}{=} \PY{n}{computeFunctionfromCoeff}\PY{p}{(}\PY{n}{coscos\PYZus{}coeff1}\PY{p}{)}
\end{Verbatim}


    \begin{Verbatim}[commandchars=\\\{\}]
{\color{incolor}In [{\color{incolor}11}]:} \PY{c+c1}{\PYZsh{} Plotting the Function computed using Fourier Coefficients}
         \PY{n}{ax1}\PY{o}{.}\PY{n}{semilogy}\PY{p}{(}\PY{n}{x}\PY{p}{,}\PY{n}{fexp\PYZus{}fourier}\PY{p}{,}\PY{l+s+s1}{\PYZsq{}}\PY{l+s+s1}{ro}\PY{l+s+s1}{\PYZsq{}}\PY{p}{,}\PY{n}{label} \PY{o}{=} \PY{l+s+s2}{\PYZdq{}}\PY{l+s+s2}{Function using Fourier Coefficients}\PY{l+s+s2}{\PYZdq{}}\PY{p}{)}
         \PY{n}{ax1}\PY{o}{.}\PY{n}{set\PYZus{}ylim}\PY{p}{(}\PY{p}{[}\PY{n+nb}{pow}\PY{p}{(}\PY{l+m+mi}{10}\PY{p}{,}\PY{o}{\PYZhy{}}\PY{l+m+mi}{1}\PY{p}{)}\PY{p}{,}\PY{n+nb}{pow}\PY{p}{(}\PY{l+m+mi}{10}\PY{p}{,}\PY{l+m+mi}{4}\PY{p}{)}\PY{p}{]}\PY{p}{)}
         \PY{n}{ax1}\PY{o}{.}\PY{n}{legend}\PY{p}{(}\PY{p}{)}
         \PY{n}{fig1}
\end{Verbatim}

\texttt{\color{outcolor}Out[{\color{outcolor}11}]:}
    
    \begin{center}
    \adjustimage{max size={0.9\linewidth}{0.9\paperheight}}{output_13_0.pdf}
    \end{center}
    { \hspace*{\fill} \\}
    

    \begin{Verbatim}[commandchars=\\\{\}]
{\color{incolor}In [{\color{incolor}12}]:} \PY{n}{ax2}\PY{o}{.}\PY{n}{plot}\PY{p}{(}\PY{n}{x}\PY{p}{,}\PY{n}{fcoscos\PYZus{}fourier}\PY{p}{,}\PY{l+s+s1}{\PYZsq{}}\PY{l+s+s1}{ro}\PY{l+s+s1}{\PYZsq{}}\PY{p}{,}\PY{n}{label} \PY{o}{=} \PY{l+s+s2}{\PYZdq{}}\PY{l+s+s2}{Function using Fourier Coefficients}\PY{l+s+s2}{\PYZdq{}}\PY{p}{)}
         \PY{n}{ax2}\PY{o}{.}\PY{n}{legend}\PY{p}{(}\PY{n}{loc}\PY{o}{=}\PY{l+s+s1}{\PYZsq{}}\PY{l+s+s1}{upper right}\PY{l+s+s1}{\PYZsq{}}\PY{p}{)}
         \PY{n}{fig2}
\end{Verbatim}

\texttt{\color{outcolor}Out[{\color{outcolor}12}]:}
    
    \begin{center}
    \adjustimage{max size={0.9\linewidth}{0.9\paperheight}}{output_14_0.pdf}
    \end{center}
    { \hspace*{\fill} \\}
    

    \subsection{Question3}\label{question3}

\begin{itemize}
\tightlist
\item
  Two different plots for each function using ``semilogy'' and
  ``loglog'' and plot the magnitude of the coefficients vs n
\item
  And to analyse them and to discuss the observations. \#\# Plots:
\item
  For each function magnitude of \(a_{n}\) and \(b_{n}\) coefficients
  which are computed using integration are plotted in same figure in
  semilog as well as loglog plot for simpler comparisons.
\end{itemize}

    \begin{Verbatim}[commandchars=\\\{\}]
{\color{incolor}In [{\color{incolor}13}]:} \PY{c+c1}{\PYZsh{} Plotting}
         \PY{n}{fig3} \PY{o}{=} \PY{n}{figure}\PY{p}{(}\PY{p}{)}
         \PY{n}{ax3} \PY{o}{=} \PY{n}{fig3}\PY{o}{.}\PY{n}{add\PYZus{}subplot}\PY{p}{(}\PY{l+m+mi}{111}\PY{p}{)}
         \PY{c+c1}{\PYZsh{} By using array indexing methods we separate all odd indexes starting from 1 \PYZhy{}\PYZgt{} an}
         \PY{c+c1}{\PYZsh{} and all even indexes starting from 2 \PYZhy{}\PYZgt{} bn}
         \PY{n}{ax3}\PY{o}{.}\PY{n}{semilogy}\PY{p}{(}\PY{p}{(}\PY{n}{exp\PYZus{}coeff}\PY{p}{[}\PY{l+m+mi}{1}\PY{p}{:}\PY{p}{:}\PY{l+m+mi}{2}\PY{p}{]}\PY{p}{)}\PY{p}{,}\PY{l+s+s1}{\PYZsq{}}\PY{l+s+s1}{ro}\PY{l+s+s1}{\PYZsq{}}\PY{p}{,}\PY{n}{label} \PY{o}{=} \PY{l+s+s2}{\PYZdq{}}\PY{l+s+s2}{\PYZdl{}a\PYZus{}}\PY{l+s+si}{\PYZob{}n\PYZcb{}}\PY{l+s+s2}{\PYZdl{} using Integration}\PY{l+s+s2}{\PYZdq{}}\PY{p}{)}
         \PY{n}{ax3}\PY{o}{.}\PY{n}{semilogy}\PY{p}{(}\PY{p}{(}\PY{n}{exp\PYZus{}coeff}\PY{p}{[}\PY{l+m+mi}{2}\PY{p}{:}\PY{p}{:}\PY{l+m+mi}{2}\PY{p}{]}\PY{p}{)}\PY{p}{,}\PY{l+s+s1}{\PYZsq{}}\PY{l+s+s1}{ko}\PY{l+s+s1}{\PYZsq{}}\PY{p}{,}\PY{n}{label} \PY{o}{=} \PY{l+s+s2}{\PYZdq{}}\PY{l+s+s2}{\PYZdl{}b\PYZus{}}\PY{l+s+si}{\PYZob{}n\PYZcb{}}\PY{l+s+s2}{\PYZdl{} using Integration}\PY{l+s+s2}{\PYZdq{}}\PY{p}{)}
         \PY{n}{ax3}\PY{o}{.}\PY{n}{legend}\PY{p}{(}\PY{p}{)}
         \PY{n}{title}\PY{p}{(}\PY{l+s+s2}{\PYZdq{}}\PY{l+s+s2}{Fourier coefficients of \PYZdl{}e\PYZca{}}\PY{l+s+si}{\PYZob{}x\PYZcb{}}\PY{l+s+s2}{\PYZdl{} (semi\PYZhy{}log)}\PY{l+s+s2}{\PYZdq{}}\PY{p}{)}
         \PY{n}{xlabel}\PY{p}{(}\PY{l+s+s2}{\PYZdq{}}\PY{l+s+s2}{n}\PY{l+s+s2}{\PYZdq{}}\PY{p}{)}
         \PY{n}{ylabel}\PY{p}{(}\PY{l+s+s2}{\PYZdq{}}\PY{l+s+s2}{Magnitude of coeffients}\PY{l+s+s2}{\PYZdq{}}\PY{p}{)}
         \PY{n}{show}\PY{p}{(}\PY{p}{)}
\end{Verbatim}


    \begin{center}
    \adjustimage{max size={0.9\linewidth}{0.9\paperheight}}{output_16_0.pdf}
    \end{center}
    { \hspace*{\fill} \\}
    
    \begin{Verbatim}[commandchars=\\\{\}]
{\color{incolor}In [{\color{incolor}14}]:} \PY{n}{fig4} \PY{o}{=} \PY{n}{figure}\PY{p}{(}\PY{p}{)}
         \PY{n}{ax4} \PY{o}{=} \PY{n}{fig4}\PY{o}{.}\PY{n}{add\PYZus{}subplot}\PY{p}{(}\PY{l+m+mi}{111}\PY{p}{)}
         \PY{c+c1}{\PYZsh{} By using array indexing methods we separate all odd indexes starting from 1 \PYZhy{}\PYZgt{} an}
         \PY{c+c1}{\PYZsh{} and all even indexes starting from 2 \PYZhy{}\PYZgt{} bn}
         \PY{n}{ax4}\PY{o}{.}\PY{n}{loglog}\PY{p}{(}\PY{p}{(}\PY{n}{exp\PYZus{}coeff}\PY{p}{[}\PY{l+m+mi}{1}\PY{p}{:}\PY{p}{:}\PY{l+m+mi}{2}\PY{p}{]}\PY{p}{)}\PY{p}{,}\PY{l+s+s1}{\PYZsq{}}\PY{l+s+s1}{ro}\PY{l+s+s1}{\PYZsq{}}\PY{p}{,}\PY{n}{label} \PY{o}{=} \PY{l+s+s2}{\PYZdq{}}\PY{l+s+s2}{\PYZdl{}a\PYZus{}}\PY{l+s+si}{\PYZob{}n\PYZcb{}}\PY{l+s+s2}{\PYZdl{} using Integration}\PY{l+s+s2}{\PYZdq{}}\PY{p}{)}
         \PY{n}{ax4}\PY{o}{.}\PY{n}{loglog}\PY{p}{(}\PY{p}{(}\PY{n}{exp\PYZus{}coeff}\PY{p}{[}\PY{l+m+mi}{2}\PY{p}{:}\PY{p}{:}\PY{l+m+mi}{2}\PY{p}{]}\PY{p}{)}\PY{p}{,}\PY{l+s+s1}{\PYZsq{}}\PY{l+s+s1}{ko}\PY{l+s+s1}{\PYZsq{}}\PY{p}{,}\PY{n}{label} \PY{o}{=} \PY{l+s+s2}{\PYZdq{}}\PY{l+s+s2}{\PYZdl{}b\PYZus{}}\PY{l+s+si}{\PYZob{}n\PYZcb{}}\PY{l+s+s2}{\PYZdl{} using Integration}\PY{l+s+s2}{\PYZdq{}}\PY{p}{)}
         \PY{n}{ax4}\PY{o}{.}\PY{n}{legend}\PY{p}{(}\PY{n}{loc}\PY{o}{=}\PY{l+s+s1}{\PYZsq{}}\PY{l+s+s1}{upper right}\PY{l+s+s1}{\PYZsq{}}\PY{p}{)}
         \PY{n}{title}\PY{p}{(}\PY{l+s+s2}{\PYZdq{}}\PY{l+s+s2}{Fourier coefficients of \PYZdl{}e\PYZca{}}\PY{l+s+si}{\PYZob{}x\PYZcb{}}\PY{l+s+s2}{\PYZdl{} (Log\PYZhy{}Log)}\PY{l+s+s2}{\PYZdq{}}\PY{p}{)}
         \PY{n}{xlabel}\PY{p}{(}\PY{l+s+s2}{\PYZdq{}}\PY{l+s+s2}{n}\PY{l+s+s2}{\PYZdq{}}\PY{p}{)}
         \PY{n}{ylabel}\PY{p}{(}\PY{l+s+s2}{\PYZdq{}}\PY{l+s+s2}{Magnitude of coeffients}\PY{l+s+s2}{\PYZdq{}}\PY{p}{)}
         \PY{n}{show}\PY{p}{(}\PY{p}{)}
\end{Verbatim}


    \begin{center}
    \adjustimage{max size={0.9\linewidth}{0.9\paperheight}}{output_17_0.pdf}
    \end{center}
    { \hspace*{\fill} \\}
    
    \begin{Verbatim}[commandchars=\\\{\}]
{\color{incolor}In [{\color{incolor}15}]:} \PY{n}{fig5} \PY{o}{=} \PY{n}{figure}\PY{p}{(}\PY{p}{)}
         \PY{n}{ax5} \PY{o}{=} \PY{n}{fig5}\PY{o}{.}\PY{n}{add\PYZus{}subplot}\PY{p}{(}\PY{l+m+mi}{111}\PY{p}{)}
         \PY{c+c1}{\PYZsh{} By using array indexing methods we separate all odd indexes starting from 1 \PYZhy{}\PYZgt{} an}
         \PY{c+c1}{\PYZsh{} and all even indexes starting from 2 \PYZhy{}\PYZgt{} bn}
         \PY{n}{ax5}\PY{o}{.}\PY{n}{semilogy}\PY{p}{(}\PY{p}{(}\PY{n}{coscos\PYZus{}coeff}\PY{p}{[}\PY{l+m+mi}{1}\PY{p}{:}\PY{p}{:}\PY{l+m+mi}{2}\PY{p}{]}\PY{p}{)}\PY{p}{,}\PY{l+s+s1}{\PYZsq{}}\PY{l+s+s1}{ro}\PY{l+s+s1}{\PYZsq{}}\PY{p}{,}\PY{n}{label} \PY{o}{=} \PY{l+s+s2}{\PYZdq{}}\PY{l+s+s2}{\PYZdl{}a\PYZus{}}\PY{l+s+si}{\PYZob{}n\PYZcb{}}\PY{l+s+s2}{\PYZdl{} using Integration}\PY{l+s+s2}{\PYZdq{}}\PY{p}{)}
         \PY{n}{ax5}\PY{o}{.}\PY{n}{semilogy}\PY{p}{(}\PY{p}{(}\PY{n}{coscos\PYZus{}coeff}\PY{p}{[}\PY{l+m+mi}{2}\PY{p}{:}\PY{p}{:}\PY{l+m+mi}{2}\PY{p}{]}\PY{p}{)}\PY{p}{,}\PY{l+s+s1}{\PYZsq{}}\PY{l+s+s1}{ko}\PY{l+s+s1}{\PYZsq{}}\PY{p}{,}\PY{n}{label} \PY{o}{=} \PY{l+s+s2}{\PYZdq{}}\PY{l+s+s2}{\PYZdl{}b\PYZus{}}\PY{l+s+si}{\PYZob{}n\PYZcb{}}\PY{l+s+s2}{\PYZdl{} using Integration}\PY{l+s+s2}{\PYZdq{}}\PY{p}{)}
         \PY{n}{ax5}\PY{o}{.}\PY{n}{legend}\PY{p}{(}\PY{n}{loc}\PY{o}{=}\PY{l+s+s1}{\PYZsq{}}\PY{l+s+s1}{upper right}\PY{l+s+s1}{\PYZsq{}}\PY{p}{)}
         \PY{n}{title}\PY{p}{(}\PY{l+s+s2}{\PYZdq{}}\PY{l+s+s2}{Fourier coefficients of \PYZdl{}}\PY{l+s+s2}{\PYZbs{}}\PY{l+s+s2}{cos(}\PY{l+s+s2}{\PYZbs{}}\PY{l+s+s2}{cos(x))\PYZdl{} (semi\PYZhy{}log)}\PY{l+s+s2}{\PYZdq{}}\PY{p}{)}
         \PY{n}{xlabel}\PY{p}{(}\PY{l+s+s2}{\PYZdq{}}\PY{l+s+s2}{n}\PY{l+s+s2}{\PYZdq{}}\PY{p}{)}
         \PY{n}{ylabel}\PY{p}{(}\PY{l+s+s2}{\PYZdq{}}\PY{l+s+s2}{Magnitude of coeffients}\PY{l+s+s2}{\PYZdq{}}\PY{p}{)}
         \PY{n}{show}\PY{p}{(}\PY{p}{)}
\end{Verbatim}


    \begin{center}
    \adjustimage{max size={0.9\linewidth}{0.9\paperheight}}{output_18_0.pdf}
    \end{center}
    { \hspace*{\fill} \\}
    
    \begin{Verbatim}[commandchars=\\\{\}]
{\color{incolor}In [{\color{incolor}16}]:} \PY{n}{fig6} \PY{o}{=} \PY{n}{figure}\PY{p}{(}\PY{p}{)}
         \PY{n}{ax6} \PY{o}{=} \PY{n}{fig6}\PY{o}{.}\PY{n}{add\PYZus{}subplot}\PY{p}{(}\PY{l+m+mi}{111}\PY{p}{)}
         \PY{c+c1}{\PYZsh{} By using array indexing methods we separate all odd indexes starting from 1 \PYZhy{}\PYZgt{} an}
         \PY{c+c1}{\PYZsh{} and all even indexes starting from 2 \PYZhy{}\PYZgt{} bn}
         \PY{n}{ax6}\PY{o}{.}\PY{n}{loglog}\PY{p}{(}\PY{p}{(}\PY{n}{coscos\PYZus{}coeff}\PY{p}{[}\PY{l+m+mi}{1}\PY{p}{:}\PY{p}{:}\PY{l+m+mi}{2}\PY{p}{]}\PY{p}{)}\PY{p}{,}\PY{l+s+s1}{\PYZsq{}}\PY{l+s+s1}{ro}\PY{l+s+s1}{\PYZsq{}}\PY{p}{,}\PY{n}{label} \PY{o}{=} \PY{l+s+s2}{\PYZdq{}}\PY{l+s+s2}{\PYZdl{}a\PYZus{}}\PY{l+s+si}{\PYZob{}n\PYZcb{}}\PY{l+s+s2}{\PYZdl{} using Integration}\PY{l+s+s2}{\PYZdq{}}\PY{p}{)}
         \PY{n}{ax6}\PY{o}{.}\PY{n}{loglog}\PY{p}{(}\PY{p}{(}\PY{n}{coscos\PYZus{}coeff}\PY{p}{[}\PY{l+m+mi}{2}\PY{p}{:}\PY{p}{:}\PY{l+m+mi}{2}\PY{p}{]}\PY{p}{)}\PY{p}{,}\PY{l+s+s1}{\PYZsq{}}\PY{l+s+s1}{ko}\PY{l+s+s1}{\PYZsq{}}\PY{p}{,}\PY{n}{label} \PY{o}{=} \PY{l+s+s2}{\PYZdq{}}\PY{l+s+s2}{\PYZdl{}b\PYZus{}}\PY{l+s+si}{\PYZob{}n\PYZcb{}}\PY{l+s+s2}{\PYZdl{} using Integration}\PY{l+s+s2}{\PYZdq{}}\PY{p}{)}
         \PY{n}{ax6}\PY{o}{.}\PY{n}{legend}\PY{p}{(}\PY{n}{loc}\PY{o}{=}\PY{l+s+s1}{\PYZsq{}}\PY{l+s+s1}{upper right}\PY{l+s+s1}{\PYZsq{}}\PY{p}{)}
         \PY{n}{title}\PY{p}{(}\PY{l+s+s2}{\PYZdq{}}\PY{l+s+s2}{Fourier coefficients of \PYZdl{}}\PY{l+s+s2}{\PYZbs{}}\PY{l+s+s2}{cos(}\PY{l+s+s2}{\PYZbs{}}\PY{l+s+s2}{cos(x))\PYZdl{}  (Log\PYZhy{}Log)}\PY{l+s+s2}{\PYZdq{}}\PY{p}{)}
         \PY{n}{xlabel}\PY{p}{(}\PY{l+s+s2}{\PYZdq{}}\PY{l+s+s2}{n}\PY{l+s+s2}{\PYZdq{}}\PY{p}{)}
         \PY{n}{ylabel}\PY{p}{(}\PY{l+s+s2}{\PYZdq{}}\PY{l+s+s2}{Magnitude of coeffients}\PY{l+s+s2}{\PYZdq{}}\PY{p}{)}
         \PY{n}{show}\PY{p}{(}\PY{p}{)}
\end{Verbatim}


    \begin{center}
    \adjustimage{max size={0.9\linewidth}{0.9\paperheight}}{output_19_0.pdf}
    \end{center}
    { \hspace*{\fill} \\}
    
    \subsubsection{Results and Observations :}\label{results-and-observations}

\begin{enumerate}
\def\labelenumi{(\alph{enumi})}
\tightlist
\item
  The \(b_{n}\) coefficients in the second case should be nearly zero.
  Why does this happen?
\item
  In the first case, the coefficients do not decay as quickly as the
  coefficients for the second case. Why not?
\item
  Why does loglog plot in Figure 4 look linear, wheras the semilog plot
  in Figure 5 looks linear?
\end{enumerate}

    \subsection{Question 4 \& 5}\label{question-4-5}

\begin{itemize}
\tightlist
\item
  Uses least squares method approach to find the fourier coefficients of
  \(e^{x}\) and \(\cos(\cos(x))\)
\item
  Evaluate both the functions at each x values and call it b. Now this
  is approximated by
  \(a_{0} + \sum\limits_{n=1}^{\infty} {{a_{n}\cos(nx)+b_{n}\sin(nx)}}\)
\item
  such that

  \begin{equation}
  a_{0} + \sum\limits_{n=1}^{\infty} {{a_{n}\cos(nx_{i})+b_{n}\sin(nx_{i})}} \approx f(x_{i}) 
  \end{equation}
\item
  To implement this we use matrices to find the coefficients using Least
  Squares method using inbuilt python function 'lstsq'
\end{itemize}

    \begin{Verbatim}[commandchars=\\\{\}]
{\color{incolor}In [{\color{incolor}17}]:} \PY{c+c1}{\PYZsh{}Function to calculate coefficients using lstsq and by calling }
         \PY{c+c1}{\PYZsh{} function \PYZsq{}createAmatrix\PYZsq{} which was defined earlier in the code }
         \PY{c+c1}{\PYZsh{} to create \PYZsq{}A\PYZsq{} matrix with arguments as function \PYZsq{}f\PYZsq{} and lower}
         \PY{c+c1}{\PYZsh{} and upper limits of input x and no\PYZus{}of points needed}
         
         \PY{k}{def} \PY{n+nf}{getCoeffByLeastSq}\PY{p}{(}\PY{n}{f}\PY{p}{,}\PY{n}{low\PYZus{}lim}\PY{p}{,}\PY{n}{upp\PYZus{}lim}\PY{p}{,}\PY{n}{no\PYZus{}points}\PY{p}{)}\PY{p}{:}
             \PY{n}{x1} \PY{o}{=} \PY{n}{linspace}\PY{p}{(}\PY{n}{low\PYZus{}lim}\PY{p}{,}\PY{n}{upp\PYZus{}lim}\PY{p}{,}\PY{n}{no\PYZus{}points}\PY{p}{)}
             \PY{n}{b} \PY{o}{=} \PY{n}{f}\PY{p}{(}\PY{n}{x1}\PY{p}{)}
             \PY{n}{A} \PY{o}{=} \PY{n}{createAmatrix}\PY{p}{(}\PY{l+m+mi}{400}\PY{p}{,}\PY{l+m+mi}{51}\PY{p}{,}\PY{n}{x1}\PY{p}{)}
             \PY{n}{c} \PY{o}{=} \PY{p}{[}\PY{p}{]}
             \PY{n}{c}\PY{o}{=}\PY{n}{lstsq}\PY{p}{(}\PY{n}{A}\PY{p}{,}\PY{n}{b}\PY{p}{)}\PY{p}{[}\PY{l+m+mi}{0}\PY{p}{]} \PY{c+c1}{\PYZsh{} the ’[0]’ is to pull out the}
             \PY{c+c1}{\PYZsh{} best fit vector. lstsq returns a list.}
             \PY{k}{return} \PY{n}{c}
\end{Verbatim}


    \begin{Verbatim}[commandchars=\\\{\}]
{\color{incolor}In [{\color{incolor}18}]:} \PY{c+c1}{\PYZsh{} Calling function and storing them in respective vectors.}
         \PY{n}{coeff\PYZus{}exp} \PY{o}{=} \PY{n}{getCoeffByLeastSq}\PY{p}{(}\PY{n}{fexp}\PY{p}{,}\PY{l+m+mi}{0}\PY{p}{,}\PY{l+m+mi}{2}\PY{o}{*}\PY{n}{pi}\PY{p}{,}\PY{l+m+mi}{400}\PY{p}{)}
         \PY{n}{coeff\PYZus{}coscos} \PY{o}{=} \PY{n}{getCoeffByLeastSq}\PY{p}{(}\PY{n}{fcoscos}\PY{p}{,}\PY{l+m+mi}{0}\PY{p}{,}\PY{l+m+mi}{2}\PY{o}{*}\PY{n}{pi}\PY{p}{,}\PY{l+m+mi}{400}\PY{p}{)}
         
         \PY{c+c1}{\PYZsh{} To plot magnitude of coefficients this is used}
         \PY{n}{c1} \PY{o}{=} \PY{n}{np}\PY{o}{.}\PY{n}{abs}\PY{p}{(}\PY{n}{coeff\PYZus{}exp}\PY{p}{)}
         \PY{n}{c2} \PY{o}{=} \PY{n}{np}\PY{o}{.}\PY{n}{abs}\PY{p}{(}\PY{n}{coeff\PYZus{}coscos}\PY{p}{)}
\end{Verbatim}


    \begin{Verbatim}[commandchars=\\\{\}]
{\color{incolor}In [{\color{incolor}19}]:} \PY{c+c1}{\PYZsh{} Plotting in coefficients got using Lstsq in corresponding figures}
         \PY{c+c1}{\PYZsh{} 3,4,5,6 using axes.}
         \PY{n}{ax3}\PY{o}{.}\PY{n}{semilogy}\PY{p}{(}\PY{p}{(}\PY{n}{c1}\PY{p}{[}\PY{l+m+mi}{1}\PY{p}{:}\PY{p}{:}\PY{l+m+mi}{2}\PY{p}{]}\PY{p}{)}\PY{p}{,}\PY{l+s+s1}{\PYZsq{}}\PY{l+s+s1}{go}\PY{l+s+s1}{\PYZsq{}}\PY{p}{,}\PY{n}{label} \PY{o}{=} \PY{l+s+s2}{\PYZdq{}}\PY{l+s+s2}{\PYZdl{}a\PYZus{}}\PY{l+s+si}{\PYZob{}n\PYZcb{}}\PY{l+s+s2}{\PYZdl{} using Least Squares}\PY{l+s+s2}{\PYZdq{}}\PY{p}{)}
         \PY{n}{ax3}\PY{o}{.}\PY{n}{semilogy}\PY{p}{(}\PY{p}{(}\PY{n}{c1}\PY{p}{[}\PY{l+m+mi}{2}\PY{p}{:}\PY{p}{:}\PY{l+m+mi}{2}\PY{p}{]}\PY{p}{)}\PY{p}{,}\PY{l+s+s1}{\PYZsq{}}\PY{l+s+s1}{bo}\PY{l+s+s1}{\PYZsq{}}\PY{p}{,}\PY{n}{label} \PY{o}{=} \PY{l+s+s2}{\PYZdq{}}\PY{l+s+s2}{\PYZdl{}b\PYZus{}}\PY{l+s+si}{\PYZob{}n\PYZcb{}}\PY{l+s+s2}{\PYZdl{} using Least Squares}\PY{l+s+s2}{\PYZdq{}}\PY{p}{)}
         \PY{n}{ax3}\PY{o}{.}\PY{n}{legend}\PY{p}{(}\PY{n}{loc}\PY{o}{=}\PY{l+s+s1}{\PYZsq{}}\PY{l+s+s1}{upper right}\PY{l+s+s1}{\PYZsq{}}\PY{p}{)}
         \PY{n}{fig3}
\end{Verbatim}

\texttt{\color{outcolor}Out[{\color{outcolor}19}]:}
    
    \begin{center}
    \adjustimage{max size={0.9\linewidth}{0.9\paperheight}}{output_24_0.pdf}
    \end{center}
    { \hspace*{\fill} \\}
    

    \begin{Verbatim}[commandchars=\\\{\}]
{\color{incolor}In [{\color{incolor}20}]:} \PY{n}{ax4}\PY{o}{.}\PY{n}{loglog}\PY{p}{(}\PY{p}{(}\PY{n}{c1}\PY{p}{[}\PY{l+m+mi}{1}\PY{p}{:}\PY{p}{:}\PY{l+m+mi}{2}\PY{p}{]}\PY{p}{)}\PY{p}{,}\PY{l+s+s1}{\PYZsq{}}\PY{l+s+s1}{go}\PY{l+s+s1}{\PYZsq{}}\PY{p}{,}\PY{n}{label} \PY{o}{=} \PY{l+s+s2}{\PYZdq{}}\PY{l+s+s2}{\PYZdl{}a\PYZus{}}\PY{l+s+si}{\PYZob{}n\PYZcb{}}\PY{l+s+s2}{\PYZdl{} using Least Squares }\PY{l+s+s2}{\PYZdq{}}\PY{p}{)}
         \PY{n}{ax4}\PY{o}{.}\PY{n}{loglog}\PY{p}{(}\PY{p}{(}\PY{n}{c1}\PY{p}{[}\PY{l+m+mi}{2}\PY{p}{:}\PY{p}{:}\PY{l+m+mi}{2}\PY{p}{]}\PY{p}{)}\PY{p}{,}\PY{l+s+s1}{\PYZsq{}}\PY{l+s+s1}{bo}\PY{l+s+s1}{\PYZsq{}}\PY{p}{,}\PY{n}{label} \PY{o}{=} \PY{l+s+s2}{\PYZdq{}}\PY{l+s+s2}{\PYZdl{}b\PYZus{}}\PY{l+s+si}{\PYZob{}n\PYZcb{}}\PY{l+s+s2}{\PYZdl{} using Least Squares}\PY{l+s+s2}{\PYZdq{}}\PY{p}{)}
         \PY{n}{ax4}\PY{o}{.}\PY{n}{legend}\PY{p}{(}\PY{n}{loc}\PY{o}{=}\PY{l+s+s1}{\PYZsq{}}\PY{l+s+s1}{lower left}\PY{l+s+s1}{\PYZsq{}}\PY{p}{)}
         \PY{n}{fig4}
\end{Verbatim}

\texttt{\color{outcolor}Out[{\color{outcolor}20}]:}
    
    \begin{center}
    \adjustimage{max size={0.9\linewidth}{0.9\paperheight}}{output_25_0.pdf}
    \end{center}
    { \hspace*{\fill} \\}
    

    \begin{Verbatim}[commandchars=\\\{\}]
{\color{incolor}In [{\color{incolor}21}]:} \PY{n}{ax5}\PY{o}{.}\PY{n}{semilogy}\PY{p}{(}\PY{p}{(}\PY{n}{c2}\PY{p}{[}\PY{l+m+mi}{1}\PY{p}{:}\PY{p}{:}\PY{l+m+mi}{2}\PY{p}{]}\PY{p}{)}\PY{p}{,}\PY{l+s+s1}{\PYZsq{}}\PY{l+s+s1}{go}\PY{l+s+s1}{\PYZsq{}}\PY{p}{,}\PY{n}{label} \PY{o}{=} \PY{l+s+s2}{\PYZdq{}}\PY{l+s+s2}{\PYZdl{}a\PYZus{}}\PY{l+s+si}{\PYZob{}n\PYZcb{}}\PY{l+s+s2}{\PYZdl{} using Least Squares}\PY{l+s+s2}{\PYZdq{}}\PY{p}{)}
         \PY{n}{ax5}\PY{o}{.}\PY{n}{semilogy}\PY{p}{(}\PY{p}{(}\PY{n}{c2}\PY{p}{[}\PY{l+m+mi}{2}\PY{p}{:}\PY{p}{:}\PY{l+m+mi}{2}\PY{p}{]}\PY{p}{)}\PY{p}{,}\PY{l+s+s1}{\PYZsq{}}\PY{l+s+s1}{bo}\PY{l+s+s1}{\PYZsq{}}\PY{p}{,}\PY{n}{label} \PY{o}{=} \PY{l+s+s2}{\PYZdq{}}\PY{l+s+s2}{\PYZdl{}b\PYZus{}}\PY{l+s+si}{\PYZob{}n\PYZcb{}}\PY{l+s+s2}{\PYZdl{} using Least Squares}\PY{l+s+s2}{\PYZdq{}}\PY{p}{)}
         \PY{n}{ax5}\PY{o}{.}\PY{n}{legend}\PY{p}{(}\PY{n}{loc}\PY{o}{=}\PY{l+s+s1}{\PYZsq{}}\PY{l+s+s1}{upper right}\PY{l+s+s1}{\PYZsq{}}\PY{p}{)}
         \PY{n}{fig5}
\end{Verbatim}

\texttt{\color{outcolor}Out[{\color{outcolor}21}]:}
    
    \begin{center}
    \adjustimage{max size={0.9\linewidth}{0.9\paperheight}}{output_26_0.pdf}
    \end{center}
    { \hspace*{\fill} \\}
    

    \begin{Verbatim}[commandchars=\\\{\}]
{\color{incolor}In [{\color{incolor}22}]:} \PY{n}{ax6}\PY{o}{.}\PY{n}{loglog}\PY{p}{(}\PY{p}{(}\PY{n}{c2}\PY{p}{[}\PY{l+m+mi}{1}\PY{p}{:}\PY{p}{:}\PY{l+m+mi}{2}\PY{p}{]}\PY{p}{)}\PY{p}{,}\PY{l+s+s1}{\PYZsq{}}\PY{l+s+s1}{go}\PY{l+s+s1}{\PYZsq{}}\PY{p}{,}\PY{n}{label} \PY{o}{=} \PY{l+s+s2}{\PYZdq{}}\PY{l+s+s2}{\PYZdl{}a\PYZus{}}\PY{l+s+si}{\PYZob{}n\PYZcb{}}\PY{l+s+s2}{\PYZdl{} using Least Squares }\PY{l+s+s2}{\PYZdq{}}\PY{p}{)}
         \PY{n}{ax6}\PY{o}{.}\PY{n}{loglog}\PY{p}{(}\PY{p}{(}\PY{n}{c2}\PY{p}{[}\PY{l+m+mi}{2}\PY{p}{:}\PY{p}{:}\PY{l+m+mi}{2}\PY{p}{]}\PY{p}{)}\PY{p}{,}\PY{l+s+s1}{\PYZsq{}}\PY{l+s+s1}{bo}\PY{l+s+s1}{\PYZsq{}}\PY{p}{,}\PY{n}{label} \PY{o}{=} \PY{l+s+s2}{\PYZdq{}}\PY{l+s+s2}{\PYZdl{}b\PYZus{}}\PY{l+s+si}{\PYZob{}n\PYZcb{}}\PY{l+s+s2}{\PYZdl{} using Least Squares}\PY{l+s+s2}{\PYZdq{}}\PY{p}{)}
         \PY{n}{ax6}\PY{o}{.}\PY{n}{legend}\PY{p}{(}\PY{n}{loc}\PY{o}{=}\PY{l+m+mi}{0}\PY{p}{)}
         \PY{n}{fig6}
\end{Verbatim}

\texttt{\color{outcolor}Out[{\color{outcolor}22}]:}
    
    \begin{center}
    \adjustimage{max size={0.9\linewidth}{0.9\paperheight}}{output_27_0.pdf}
    \end{center}
    { \hspace*{\fill} \\}
    

    \subsection{Question 6}\label{question-6}

\begin{itemize}
\tightlist
\item
  To compare the answers got by least squares and by the direct
  integration.
\item
  And finding deviation between them and find the largest deviation
  using Vectors
\end{itemize}

    \begin{Verbatim}[commandchars=\\\{\}]
{\color{incolor}In [{\color{incolor}23}]:} \PY{c+c1}{\PYZsh{} Function to compare the coefficients got by integration and }
         \PY{c+c1}{\PYZsh{} least squares and find largest deviation using Vectorized Technique}
         \PY{c+c1}{\PYZsh{} Argument : \PYZsq{}integer f which is either 1 or .}
         \PY{c+c1}{\PYZsh{} 1 \PYZhy{}\PYZgt{} exp(x)    2 \PYZhy{}\PYZgt{} cos(cos(x))}
         \PY{k}{def} \PY{n+nf}{compareCoeff}\PY{p}{(}\PY{n}{f}\PY{p}{)}\PY{p}{:}
             \PY{n}{deviations} \PY{o}{=} \PY{p}{[}\PY{p}{]}
             \PY{n}{max\PYZus{}dev} \PY{o}{=} \PY{l+m+mi}{0}
             \PY{k}{if}\PY{p}{(}\PY{n}{f}\PY{o}{==}\PY{l+m+mi}{1}\PY{p}{)}\PY{p}{:}
                 \PY{n}{deviations} \PY{o}{=} \PY{n}{np}\PY{o}{.}\PY{n}{abs}\PY{p}{(}\PY{n}{exp\PYZus{}coeff1} \PY{o}{\PYZhy{}} \PY{n}{coeff\PYZus{}exp}\PY{p}{)}
             \PY{k}{elif}\PY{p}{(}\PY{n}{f}\PY{o}{==}\PY{l+m+mi}{2}\PY{p}{)}\PY{p}{:}
                 \PY{n}{deviations} \PY{o}{=} \PY{n}{np}\PY{o}{.}\PY{n}{abs}\PY{p}{(}\PY{n}{coscos\PYZus{}coeff1} \PY{o}{\PYZhy{}} \PY{n}{coeff\PYZus{}coscos}\PY{p}{)}
                 
             \PY{n}{max\PYZus{}dev} \PY{o}{=} \PY{n}{np}\PY{o}{.}\PY{n}{amax}\PY{p}{(}\PY{n}{deviations}\PY{p}{)}
             \PY{k}{return} \PY{n}{deviations}\PY{p}{,}\PY{n}{max\PYZus{}dev}
\end{Verbatim}


    \begin{Verbatim}[commandchars=\\\{\}]
{\color{incolor}In [{\color{incolor}24}]:} \PY{n}{dev1}\PY{p}{,}\PY{n}{maxdev1} \PY{o}{=} \PY{n}{compareCoeff}\PY{p}{(}\PY{l+m+mi}{1}\PY{p}{)}
         \PY{n}{dev2}\PY{p}{,}\PY{n}{maxdev2} \PY{o}{=} \PY{n}{compareCoeff}\PY{p}{(}\PY{l+m+mi}{2}\PY{p}{)}
         
         \PY{n+nb}{print}\PY{p}{(}\PY{l+s+s2}{\PYZdq{}}\PY{l+s+s2}{The largest deviation for exp(x) : }\PY{l+s+si}{\PYZpc{}g}\PY{l+s+s2}{\PYZdq{}} \PY{o}{\PYZpc{}}\PY{p}{(}\PY{n}{maxdev1}\PY{p}{)}\PY{p}{)}
         \PY{n+nb}{print}\PY{p}{(}\PY{l+s+s2}{\PYZdq{}}\PY{l+s+s2}{The largest deviation for cos(cos(x)) : }\PY{l+s+si}{\PYZpc{}g}\PY{l+s+s2}{\PYZdq{}} \PY{o}{\PYZpc{}}\PY{p}{(}\PY{n}{maxdev2}\PY{p}{)}\PY{p}{)}
         
         \PY{c+c1}{\PYZsh{} Plotting the deviation vs n }
         \PY{n}{plot}\PY{p}{(}\PY{n}{dev1}\PY{p}{,}\PY{l+s+s1}{\PYZsq{}}\PY{l+s+s1}{g}\PY{l+s+s1}{\PYZsq{}}\PY{p}{)}
         \PY{n}{title}\PY{p}{(}\PY{l+s+s2}{\PYZdq{}}\PY{l+s+s2}{Deviation between Coefficients by lstsq \PYZam{} by Integration method for \PYZdl{}e\PYZca{}}\PY{l+s+si}{\PYZob{}x\PYZcb{}}\PY{l+s+s2}{\PYZdl{}}\PY{l+s+s2}{\PYZdq{}}\PY{p}{)}
         \PY{n}{xlabel}\PY{p}{(}\PY{l+s+s2}{\PYZdq{}}\PY{l+s+s2}{n}\PY{l+s+s2}{\PYZdq{}}\PY{p}{)}
         \PY{n}{ylabel}\PY{p}{(}\PY{l+s+s2}{\PYZdq{}}\PY{l+s+s2}{Magnitude of Deviations}\PY{l+s+s2}{\PYZdq{}}\PY{p}{)}
         \PY{n}{show}\PY{p}{(}\PY{p}{)}
\end{Verbatim}


    \begin{Verbatim}[commandchars=\\\{\}]
The largest deviation for exp(x) : 0.0881217
The largest deviation for cos(cos(x)) : 2.60883e-15

    \end{Verbatim}

    \begin{center}
    \adjustimage{max size={0.9\linewidth}{0.9\paperheight}}{output_30_1.pdf}
    \end{center}
    { \hspace*{\fill} \\}
    
    \begin{Verbatim}[commandchars=\\\{\}]
{\color{incolor}In [{\color{incolor}25}]:} \PY{c+c1}{\PYZsh{} Plotting the deviation vs n }
         \PY{n}{plot}\PY{p}{(}\PY{n}{dev2}\PY{p}{,}\PY{l+s+s1}{\PYZsq{}}\PY{l+s+s1}{g}\PY{l+s+s1}{\PYZsq{}}\PY{p}{)}
         \PY{n}{title}\PY{p}{(}\PY{l+s+s2}{\PYZdq{}}\PY{l+s+s2}{Lstsq Vs Integration method (Deviation) for \PYZdl{}}\PY{l+s+s2}{\PYZbs{}}\PY{l+s+s2}{cos(}\PY{l+s+s2}{\PYZbs{}}\PY{l+s+s2}{cos(x))\PYZdl{}}\PY{l+s+s2}{\PYZdq{}}\PY{p}{)}
         \PY{n}{xlabel}\PY{p}{(}\PY{l+s+s2}{\PYZdq{}}\PY{l+s+s2}{n}\PY{l+s+s2}{\PYZdq{}}\PY{p}{)}
         \PY{n}{ylabel}\PY{p}{(}\PY{l+s+s2}{\PYZdq{}}\PY{l+s+s2}{Magnitude of Deviations}\PY{l+s+s2}{\PYZdq{}}\PY{p}{)}
         \PY{n}{show}\PY{p}{(}\PY{p}{)}
\end{Verbatim}


    \begin{center}
    \adjustimage{max size={0.9\linewidth}{0.9\paperheight}}{output_31_0.pdf}
    \end{center}
    { \hspace*{\fill} \\}
    
    \subsubsection{Results and Discussion :}\label{results-and-discussion}

\begin{itemize}
\tightlist
\item
  As we observe that there is a significant deviation for \(e^{x}\) as
  it has discontinuites at \(2n\pi\) which can be observed in Figure 1
  and hence there will be \textbf{Gibbs phenomenon} i.e there will be
  oscillations around the discontinuity points.
\item
  Due to this the fourier coefficients using least squares will not fit
  the curve exactly
\item
  Whereas for \(\cos(\cos(x))\) the largest deviation is in order of
  \(10^{-15}\) because the function itself is a periodic function and it
  is a continous function in entire x range so we get very negligible
  deviation.
\item
  And as we know that Fourier series is used to define periodic signals
  in frequency domain and \(e^{x}\) is a aperiodic signal so you can't
  define an aperiodic signal on an interval of finite length (if you
  try, you'll lose information about the signal), so one must use the
  Fourier transform for such a signal.
\item
  Thats why there is significant deviations are found for \(e^{x}\)
\end{itemize}

    \subsection{Question 7}\label{question-7}

\begin{itemize}
\tightlist
\item
  Computing Ac i.e multiplying Matrix A and Vector C from the estimated
  values of Coeffient Vector C by Least Squares Method.
\item
  To Plot them (with green circles) in Figures 1 and 2 respectively for
  the two functions.
\end{itemize}

    \begin{Verbatim}[commandchars=\\\{\}]
{\color{incolor}In [{\color{incolor}26}]:} \PY{c+c1}{\PYZsh{} Define vector x1 from 0 to 2pi}
         \PY{n}{x1} \PY{o}{=} \PY{n}{linspace}\PY{p}{(}\PY{l+m+mi}{0}\PY{p}{,}\PY{l+m+mi}{2}\PY{o}{*}\PY{n}{pi}\PY{p}{,}\PY{l+m+mi}{400}\PY{p}{)}
\end{Verbatim}


    \begin{Verbatim}[commandchars=\\\{\}]
{\color{incolor}In [{\color{incolor}27}]:} \PY{c+c1}{\PYZsh{} Function to reconstruct the signalfrom coefficients}
         \PY{c+c1}{\PYZsh{} computed using Least Squares.}
         \PY{c+c1}{\PYZsh{} Takes coefficient vector : \PYZsq{}c\PYZsq{} as argument}
         \PY{c+c1}{\PYZsh{} returns vector values of function at each x}
         \PY{k}{def} \PY{n+nf}{computeFunctionbyLeastSq}\PY{p}{(}\PY{n}{c}\PY{p}{)}\PY{p}{:}
             \PY{n}{f\PYZus{}lstsq} \PY{o}{=} \PY{p}{[}\PY{p}{]}
             \PY{n}{A} \PY{o}{=} \PY{n}{createAmatrix}\PY{p}{(}\PY{l+m+mi}{400}\PY{p}{,}\PY{l+m+mi}{51}\PY{p}{,}\PY{n}{x1}\PY{p}{)}
             \PY{n}{f\PYZus{}lstsq} \PY{o}{=} \PY{n}{A}\PY{o}{.}\PY{n}{dot}\PY{p}{(}\PY{n}{c}\PY{p}{)}
             \PY{k}{return} \PY{n}{f\PYZus{}lstsq}
\end{Verbatim}


    \begin{Verbatim}[commandchars=\\\{\}]
{\color{incolor}In [{\color{incolor}28}]:} \PY{n}{fexp\PYZus{}lstsq} \PY{o}{=} \PY{n}{computeFunctionbyLeastSq}\PY{p}{(}\PY{n}{coeff\PYZus{}exp}\PY{p}{)}
         \PY{n}{fcoscos\PYZus{}lstsq} \PY{o}{=} \PY{n}{computeFunctionbyLeastSq}\PY{p}{(}\PY{n}{coeff\PYZus{}coscos}\PY{p}{)}
         
         \PY{c+c1}{\PYZsh{} Plotting in Figure1 to compare the original function }
         \PY{c+c1}{\PYZsh{} and Reconstructed one using Least Squares method}
         \PY{n}{ax1}\PY{o}{.}\PY{n}{semilogy}\PY{p}{(}\PY{n}{x1}\PY{p}{,}\PY{n}{fexp\PYZus{}lstsq}\PY{p}{,}\PY{l+s+s1}{\PYZsq{}}\PY{l+s+s1}{go}\PY{l+s+s1}{\PYZsq{}}\PY{p}{,}
                      \PY{n}{label} \PY{o}{=} \PY{l+s+s2}{\PYZdq{}}\PY{l+s+s2}{Inverse Fourier Transform From Least Squares}\PY{l+s+s2}{\PYZdq{}}\PY{p}{)}
         \PY{n}{ax1}\PY{o}{.}\PY{n}{legend}\PY{p}{(}\PY{p}{)}
         \PY{n}{ax1}\PY{o}{.}\PY{n}{set\PYZus{}ylim}\PY{p}{(}\PY{p}{[}\PY{n+nb}{pow}\PY{p}{(}\PY{l+m+mi}{10}\PY{p}{,}\PY{o}{\PYZhy{}}\PY{l+m+mi}{2}\PY{p}{)}\PY{p}{,}\PY{n+nb}{pow}\PY{p}{(}\PY{l+m+mi}{10}\PY{p}{,}\PY{l+m+mi}{5}\PY{p}{)}\PY{p}{]}\PY{p}{)}
         \PY{n}{ax1}\PY{o}{.}\PY{n}{set\PYZus{}xlim}\PY{p}{(}\PY{p}{[}\PY{l+m+mi}{0}\PY{p}{,}\PY{l+m+mi}{2}\PY{o}{*}\PY{n}{pi}\PY{p}{]}\PY{p}{)}
         \PY{n}{fig1}
\end{Verbatim}

\texttt{\color{outcolor}Out[{\color{outcolor}28}]:}
    
    \begin{center}
    \adjustimage{max size={0.9\linewidth}{0.9\paperheight}}{output_36_0.pdf}
    \end{center}
    { \hspace*{\fill} \\}
    

    \begin{Verbatim}[commandchars=\\\{\}]
{\color{incolor}In [{\color{incolor}29}]:} \PY{n}{ax2}\PY{o}{.}\PY{n}{plot}\PY{p}{(}\PY{n}{x1}\PY{p}{,}\PY{n}{fcoscos\PYZus{}lstsq}\PY{p}{,}\PY{l+s+s1}{\PYZsq{}}\PY{l+s+s1}{go}\PY{l+s+s1}{\PYZsq{}}\PY{p}{,}\PY{n}{markersize}\PY{o}{=}\PY{l+m+mi}{4}\PY{p}{,}
                  \PY{n}{label} \PY{o}{=} \PY{l+s+s2}{\PYZdq{}}\PY{l+s+s2}{Inverse Fourier Transform From Least Squares}\PY{l+s+s2}{\PYZdq{}}\PY{p}{)}
         \PY{n}{ax2}\PY{o}{.}\PY{n}{set\PYZus{}ylim}\PY{p}{(}\PY{p}{[}\PY{l+m+mf}{0.5}\PY{p}{,}\PY{l+m+mf}{1.3}\PY{p}{]}\PY{p}{)}
         \PY{n}{ax2}\PY{o}{.}\PY{n}{set\PYZus{}xlim}\PY{p}{(}\PY{p}{[}\PY{l+m+mi}{0}\PY{p}{,}\PY{l+m+mi}{2}\PY{o}{*}\PY{n}{pi}\PY{p}{]}\PY{p}{)}
         \PY{n}{ax2}\PY{o}{.}\PY{n}{legend}\PY{p}{(}\PY{p}{)}
         \PY{n}{fig2}
\end{Verbatim}

\texttt{\color{outcolor}Out[{\color{outcolor}29}]:}
    
    \begin{center}
    \adjustimage{max size={0.9\linewidth}{0.9\paperheight}}{output_37_0.pdf}
    \end{center}
    { \hspace*{\fill} \\}
    

    \subsubsection{Results and Discussion :}\label{results-and-discussion}

\begin{itemize}
\tightlist
\item
  As we observe that there is a significant deviation for \(e^{x}\) as
  it has discontinuites at \(2n\pi\) which can be observed in Figure 1
  and Because there will be \textbf{Gibbs phenomenon} i.e there will be
  oscillations around the discontinuity points and their ripple
  amplitude will decrease as we go close to discontinuity. In this case
  it is at \(2\pi\) for \(e^{x}\).
\item
  As we observe that rimples are high in starting and reduces and
  oscillate with more frequency as we go towards \(2\pi\). This
  phenomenon is called \textbf{Gibbs Phenomenon}
\item
  Due to this. the orginal function and one which is reconstructed using
  least squares will not fit exactly.
\item
  And as we know that Fourier series is used to define periodic signals
  in frequency domain and \(e^{x}\) is a aperiodic signal so you can't
  define an aperiodic signal on an interval of finite length (if you
  try, you'll lose information about the signal), so one must use the
  Fourier transform for such a signal.
\item
  Thats why there are significant deviations for \(e^{x}\) from original
  function.
\item
  Whereas for \(\cos(\cos(x))\) the curves fit almost perfectly because
  the function itself is a periodic function and it is a continous
  function in entire x range so we get very negligible deviation and
  able to reconstruct the signal with just the fourier coefficients.
\end{itemize}


    % Add a bibliography block to the postdoc
    
    
    
    \end{document}
