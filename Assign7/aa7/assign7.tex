
% Default to the notebook output style

    


% Inherit from the specified cell style.




    
\documentclass[11pt]{article}

    
    
    \usepackage[T1]{fontenc}
    % Nicer default font (+ math font) than Computer Modern for most use cases
    \usepackage{mathpazo}

    % Basic figure setup, for now with no caption control since it's done
    % automatically by Pandoc (which extracts ![](path) syntax from Markdown).
    \usepackage{graphicx}
    % We will generate all images so they have a width \maxwidth. This means
    % that they will get their normal width if they fit onto the page, but
    % are scaled down if they would overflow the margins.
    \makeatletter
    \def\maxwidth{\ifdim\Gin@nat@width>\linewidth\linewidth
    \else\Gin@nat@width\fi}
    \makeatother
    \let\Oldincludegraphics\includegraphics
    % Set max figure width to be 80% of text width, for now hardcoded.
    \renewcommand{\includegraphics}[1]{\Oldincludegraphics[width=.8\maxwidth]{#1}}
    % Ensure that by default, figures have no caption (until we provide a
    % proper Figure object with a Caption API and a way to capture that
    % in the conversion process - todo).
    \usepackage{caption}
    \DeclareCaptionLabelFormat{nolabel}{}
    \captionsetup{labelformat=nolabel}

    \usepackage{adjustbox} % Used to constrain images to a maximum size 
    \usepackage{xcolor} % Allow colors to be defined
    \usepackage{enumerate} % Needed for markdown enumerations to work
    \usepackage{geometry} % Used to adjust the document margins
    \usepackage{amsmath} % Equations
    \usepackage{amssymb} % Equations
    \usepackage{textcomp} % defines textquotesingle
    % Hack from http://tex.stackexchange.com/a/47451/13684:
    \AtBeginDocument{%
        \def\PYZsq{\textquotesingle}% Upright quotes in Pygmentized code
    }
    \usepackage{upquote} % Upright quotes for verbatim code
    \usepackage{eurosym} % defines \euro
    \usepackage[mathletters]{ucs} % Extended unicode (utf-8) support
    \usepackage[utf8x]{inputenc} % Allow utf-8 characters in the tex document
    \usepackage{fancyvrb} % verbatim replacement that allows latex
    \usepackage{grffile} % extends the file name processing of package graphics 
                         % to support a larger range 
    % The hyperref package gives us a pdf with properly built
    % internal navigation ('pdf bookmarks' for the table of contents,
    % internal cross-reference links, web links for URLs, etc.)
    \usepackage{hyperref}
    \usepackage{longtable} % longtable support required by pandoc >1.10
    \usepackage{booktabs}  % table support for pandoc > 1.12.2
    \usepackage[inline]{enumitem} % IRkernel/repr support (it uses the enumerate* environment)
    \usepackage[normalem]{ulem} % ulem is needed to support strikethroughs (\sout)
                                % normalem makes italics be italics, not underlines
    

    
    
    % Colors for the hyperref package
    \definecolor{urlcolor}{rgb}{0,.145,.698}
    \definecolor{linkcolor}{rgb}{.71,0.21,0.01}
    \definecolor{citecolor}{rgb}{.12,.54,.11}

    % ANSI colors
    \definecolor{ansi-black}{HTML}{3E424D}
    \definecolor{ansi-black-intense}{HTML}{282C36}
    \definecolor{ansi-red}{HTML}{E75C58}
    \definecolor{ansi-red-intense}{HTML}{B22B31}
    \definecolor{ansi-green}{HTML}{00A250}
    \definecolor{ansi-green-intense}{HTML}{007427}
    \definecolor{ansi-yellow}{HTML}{DDB62B}
    \definecolor{ansi-yellow-intense}{HTML}{B27D12}
    \definecolor{ansi-blue}{HTML}{208FFB}
    \definecolor{ansi-blue-intense}{HTML}{0065CA}
    \definecolor{ansi-magenta}{HTML}{D160C4}
    \definecolor{ansi-magenta-intense}{HTML}{A03196}
    \definecolor{ansi-cyan}{HTML}{60C6C8}
    \definecolor{ansi-cyan-intense}{HTML}{258F8F}
    \definecolor{ansi-white}{HTML}{C5C1B4}
    \definecolor{ansi-white-intense}{HTML}{A1A6B2}

    % commands and environments needed by pandoc snippets
    % extracted from the output of `pandoc -s`
    \providecommand{\tightlist}{%
      \setlength{\itemsep}{0pt}\setlength{\parskip}{0pt}}
    \DefineVerbatimEnvironment{Highlighting}{Verbatim}{commandchars=\\\{\}}
    % Add ',fontsize=\small' for more characters per line
    \newenvironment{Shaded}{}{}
    \newcommand{\KeywordTok}[1]{\textcolor[rgb]{0.00,0.44,0.13}{\textbf{{#1}}}}
    \newcommand{\DataTypeTok}[1]{\textcolor[rgb]{0.56,0.13,0.00}{{#1}}}
    \newcommand{\DecValTok}[1]{\textcolor[rgb]{0.25,0.63,0.44}{{#1}}}
    \newcommand{\BaseNTok}[1]{\textcolor[rgb]{0.25,0.63,0.44}{{#1}}}
    \newcommand{\FloatTok}[1]{\textcolor[rgb]{0.25,0.63,0.44}{{#1}}}
    \newcommand{\CharTok}[1]{\textcolor[rgb]{0.25,0.44,0.63}{{#1}}}
    \newcommand{\StringTok}[1]{\textcolor[rgb]{0.25,0.44,0.63}{{#1}}}
    \newcommand{\CommentTok}[1]{\textcolor[rgb]{0.38,0.63,0.69}{\textit{{#1}}}}
    \newcommand{\OtherTok}[1]{\textcolor[rgb]{0.00,0.44,0.13}{{#1}}}
    \newcommand{\AlertTok}[1]{\textcolor[rgb]{1.00,0.00,0.00}{\textbf{{#1}}}}
    \newcommand{\FunctionTok}[1]{\textcolor[rgb]{0.02,0.16,0.49}{{#1}}}
    \newcommand{\RegionMarkerTok}[1]{{#1}}
    \newcommand{\ErrorTok}[1]{\textcolor[rgb]{1.00,0.00,0.00}{\textbf{{#1}}}}
    \newcommand{\NormalTok}[1]{{#1}}
    
    % Additional commands for more recent versions of Pandoc
    \newcommand{\ConstantTok}[1]{\textcolor[rgb]{0.53,0.00,0.00}{{#1}}}
    \newcommand{\SpecialCharTok}[1]{\textcolor[rgb]{0.25,0.44,0.63}{{#1}}}
    \newcommand{\VerbatimStringTok}[1]{\textcolor[rgb]{0.25,0.44,0.63}{{#1}}}
    \newcommand{\SpecialStringTok}[1]{\textcolor[rgb]{0.73,0.40,0.53}{{#1}}}
    \newcommand{\ImportTok}[1]{{#1}}
    \newcommand{\DocumentationTok}[1]{\textcolor[rgb]{0.73,0.13,0.13}{\textit{{#1}}}}
    \newcommand{\AnnotationTok}[1]{\textcolor[rgb]{0.38,0.63,0.69}{\textbf{\textit{{#1}}}}}
    \newcommand{\CommentVarTok}[1]{\textcolor[rgb]{0.38,0.63,0.69}{\textbf{\textit{{#1}}}}}
    \newcommand{\VariableTok}[1]{\textcolor[rgb]{0.10,0.09,0.49}{{#1}}}
    \newcommand{\ControlFlowTok}[1]{\textcolor[rgb]{0.00,0.44,0.13}{\textbf{{#1}}}}
    \newcommand{\OperatorTok}[1]{\textcolor[rgb]{0.40,0.40,0.40}{{#1}}}
    \newcommand{\BuiltInTok}[1]{{#1}}
    \newcommand{\ExtensionTok}[1]{{#1}}
    \newcommand{\PreprocessorTok}[1]{\textcolor[rgb]{0.74,0.48,0.00}{{#1}}}
    \newcommand{\AttributeTok}[1]{\textcolor[rgb]{0.49,0.56,0.16}{{#1}}}
    \newcommand{\InformationTok}[1]{\textcolor[rgb]{0.38,0.63,0.69}{\textbf{\textit{{#1}}}}}
    \newcommand{\WarningTok}[1]{\textcolor[rgb]{0.38,0.63,0.69}{\textbf{\textit{{#1}}}}}
    
    
    % Define a nice break command that doesn't care if a line doesn't already
    % exist.
    \def\br{\hspace*{\fill} \\* }
    % Math Jax compatability definitions
    \def\gt{>}
    \def\lt{<}
    % Document parameters
    \title{assign7}
    
    
    

    % Pygments definitions
    
\makeatletter
\def\PY@reset{\let\PY@it=\relax \let\PY@bf=\relax%
    \let\PY@ul=\relax \let\PY@tc=\relax%
    \let\PY@bc=\relax \let\PY@ff=\relax}
\def\PY@tok#1{\csname PY@tok@#1\endcsname}
\def\PY@toks#1+{\ifx\relax#1\empty\else%
    \PY@tok{#1}\expandafter\PY@toks\fi}
\def\PY@do#1{\PY@bc{\PY@tc{\PY@ul{%
    \PY@it{\PY@bf{\PY@ff{#1}}}}}}}
\def\PY#1#2{\PY@reset\PY@toks#1+\relax+\PY@do{#2}}

\expandafter\def\csname PY@tok@w\endcsname{\def\PY@tc##1{\textcolor[rgb]{0.73,0.73,0.73}{##1}}}
\expandafter\def\csname PY@tok@c\endcsname{\let\PY@it=\textit\def\PY@tc##1{\textcolor[rgb]{0.25,0.50,0.50}{##1}}}
\expandafter\def\csname PY@tok@cp\endcsname{\def\PY@tc##1{\textcolor[rgb]{0.74,0.48,0.00}{##1}}}
\expandafter\def\csname PY@tok@k\endcsname{\let\PY@bf=\textbf\def\PY@tc##1{\textcolor[rgb]{0.00,0.50,0.00}{##1}}}
\expandafter\def\csname PY@tok@kp\endcsname{\def\PY@tc##1{\textcolor[rgb]{0.00,0.50,0.00}{##1}}}
\expandafter\def\csname PY@tok@kt\endcsname{\def\PY@tc##1{\textcolor[rgb]{0.69,0.00,0.25}{##1}}}
\expandafter\def\csname PY@tok@o\endcsname{\def\PY@tc##1{\textcolor[rgb]{0.40,0.40,0.40}{##1}}}
\expandafter\def\csname PY@tok@ow\endcsname{\let\PY@bf=\textbf\def\PY@tc##1{\textcolor[rgb]{0.67,0.13,1.00}{##1}}}
\expandafter\def\csname PY@tok@nb\endcsname{\def\PY@tc##1{\textcolor[rgb]{0.00,0.50,0.00}{##1}}}
\expandafter\def\csname PY@tok@nf\endcsname{\def\PY@tc##1{\textcolor[rgb]{0.00,0.00,1.00}{##1}}}
\expandafter\def\csname PY@tok@nc\endcsname{\let\PY@bf=\textbf\def\PY@tc##1{\textcolor[rgb]{0.00,0.00,1.00}{##1}}}
\expandafter\def\csname PY@tok@nn\endcsname{\let\PY@bf=\textbf\def\PY@tc##1{\textcolor[rgb]{0.00,0.00,1.00}{##1}}}
\expandafter\def\csname PY@tok@ne\endcsname{\let\PY@bf=\textbf\def\PY@tc##1{\textcolor[rgb]{0.82,0.25,0.23}{##1}}}
\expandafter\def\csname PY@tok@nv\endcsname{\def\PY@tc##1{\textcolor[rgb]{0.10,0.09,0.49}{##1}}}
\expandafter\def\csname PY@tok@no\endcsname{\def\PY@tc##1{\textcolor[rgb]{0.53,0.00,0.00}{##1}}}
\expandafter\def\csname PY@tok@nl\endcsname{\def\PY@tc##1{\textcolor[rgb]{0.63,0.63,0.00}{##1}}}
\expandafter\def\csname PY@tok@ni\endcsname{\let\PY@bf=\textbf\def\PY@tc##1{\textcolor[rgb]{0.60,0.60,0.60}{##1}}}
\expandafter\def\csname PY@tok@na\endcsname{\def\PY@tc##1{\textcolor[rgb]{0.49,0.56,0.16}{##1}}}
\expandafter\def\csname PY@tok@nt\endcsname{\let\PY@bf=\textbf\def\PY@tc##1{\textcolor[rgb]{0.00,0.50,0.00}{##1}}}
\expandafter\def\csname PY@tok@nd\endcsname{\def\PY@tc##1{\textcolor[rgb]{0.67,0.13,1.00}{##1}}}
\expandafter\def\csname PY@tok@s\endcsname{\def\PY@tc##1{\textcolor[rgb]{0.73,0.13,0.13}{##1}}}
\expandafter\def\csname PY@tok@sd\endcsname{\let\PY@it=\textit\def\PY@tc##1{\textcolor[rgb]{0.73,0.13,0.13}{##1}}}
\expandafter\def\csname PY@tok@si\endcsname{\let\PY@bf=\textbf\def\PY@tc##1{\textcolor[rgb]{0.73,0.40,0.53}{##1}}}
\expandafter\def\csname PY@tok@se\endcsname{\let\PY@bf=\textbf\def\PY@tc##1{\textcolor[rgb]{0.73,0.40,0.13}{##1}}}
\expandafter\def\csname PY@tok@sr\endcsname{\def\PY@tc##1{\textcolor[rgb]{0.73,0.40,0.53}{##1}}}
\expandafter\def\csname PY@tok@ss\endcsname{\def\PY@tc##1{\textcolor[rgb]{0.10,0.09,0.49}{##1}}}
\expandafter\def\csname PY@tok@sx\endcsname{\def\PY@tc##1{\textcolor[rgb]{0.00,0.50,0.00}{##1}}}
\expandafter\def\csname PY@tok@m\endcsname{\def\PY@tc##1{\textcolor[rgb]{0.40,0.40,0.40}{##1}}}
\expandafter\def\csname PY@tok@gh\endcsname{\let\PY@bf=\textbf\def\PY@tc##1{\textcolor[rgb]{0.00,0.00,0.50}{##1}}}
\expandafter\def\csname PY@tok@gu\endcsname{\let\PY@bf=\textbf\def\PY@tc##1{\textcolor[rgb]{0.50,0.00,0.50}{##1}}}
\expandafter\def\csname PY@tok@gd\endcsname{\def\PY@tc##1{\textcolor[rgb]{0.63,0.00,0.00}{##1}}}
\expandafter\def\csname PY@tok@gi\endcsname{\def\PY@tc##1{\textcolor[rgb]{0.00,0.63,0.00}{##1}}}
\expandafter\def\csname PY@tok@gr\endcsname{\def\PY@tc##1{\textcolor[rgb]{1.00,0.00,0.00}{##1}}}
\expandafter\def\csname PY@tok@ge\endcsname{\let\PY@it=\textit}
\expandafter\def\csname PY@tok@gs\endcsname{\let\PY@bf=\textbf}
\expandafter\def\csname PY@tok@gp\endcsname{\let\PY@bf=\textbf\def\PY@tc##1{\textcolor[rgb]{0.00,0.00,0.50}{##1}}}
\expandafter\def\csname PY@tok@go\endcsname{\def\PY@tc##1{\textcolor[rgb]{0.53,0.53,0.53}{##1}}}
\expandafter\def\csname PY@tok@gt\endcsname{\def\PY@tc##1{\textcolor[rgb]{0.00,0.27,0.87}{##1}}}
\expandafter\def\csname PY@tok@err\endcsname{\def\PY@bc##1{\setlength{\fboxsep}{0pt}\fcolorbox[rgb]{1.00,0.00,0.00}{1,1,1}{\strut ##1}}}
\expandafter\def\csname PY@tok@kc\endcsname{\let\PY@bf=\textbf\def\PY@tc##1{\textcolor[rgb]{0.00,0.50,0.00}{##1}}}
\expandafter\def\csname PY@tok@kd\endcsname{\let\PY@bf=\textbf\def\PY@tc##1{\textcolor[rgb]{0.00,0.50,0.00}{##1}}}
\expandafter\def\csname PY@tok@kn\endcsname{\let\PY@bf=\textbf\def\PY@tc##1{\textcolor[rgb]{0.00,0.50,0.00}{##1}}}
\expandafter\def\csname PY@tok@kr\endcsname{\let\PY@bf=\textbf\def\PY@tc##1{\textcolor[rgb]{0.00,0.50,0.00}{##1}}}
\expandafter\def\csname PY@tok@bp\endcsname{\def\PY@tc##1{\textcolor[rgb]{0.00,0.50,0.00}{##1}}}
\expandafter\def\csname PY@tok@fm\endcsname{\def\PY@tc##1{\textcolor[rgb]{0.00,0.00,1.00}{##1}}}
\expandafter\def\csname PY@tok@vc\endcsname{\def\PY@tc##1{\textcolor[rgb]{0.10,0.09,0.49}{##1}}}
\expandafter\def\csname PY@tok@vg\endcsname{\def\PY@tc##1{\textcolor[rgb]{0.10,0.09,0.49}{##1}}}
\expandafter\def\csname PY@tok@vi\endcsname{\def\PY@tc##1{\textcolor[rgb]{0.10,0.09,0.49}{##1}}}
\expandafter\def\csname PY@tok@vm\endcsname{\def\PY@tc##1{\textcolor[rgb]{0.10,0.09,0.49}{##1}}}
\expandafter\def\csname PY@tok@sa\endcsname{\def\PY@tc##1{\textcolor[rgb]{0.73,0.13,0.13}{##1}}}
\expandafter\def\csname PY@tok@sb\endcsname{\def\PY@tc##1{\textcolor[rgb]{0.73,0.13,0.13}{##1}}}
\expandafter\def\csname PY@tok@sc\endcsname{\def\PY@tc##1{\textcolor[rgb]{0.73,0.13,0.13}{##1}}}
\expandafter\def\csname PY@tok@dl\endcsname{\def\PY@tc##1{\textcolor[rgb]{0.73,0.13,0.13}{##1}}}
\expandafter\def\csname PY@tok@s2\endcsname{\def\PY@tc##1{\textcolor[rgb]{0.73,0.13,0.13}{##1}}}
\expandafter\def\csname PY@tok@sh\endcsname{\def\PY@tc##1{\textcolor[rgb]{0.73,0.13,0.13}{##1}}}
\expandafter\def\csname PY@tok@s1\endcsname{\def\PY@tc##1{\textcolor[rgb]{0.73,0.13,0.13}{##1}}}
\expandafter\def\csname PY@tok@mb\endcsname{\def\PY@tc##1{\textcolor[rgb]{0.40,0.40,0.40}{##1}}}
\expandafter\def\csname PY@tok@mf\endcsname{\def\PY@tc##1{\textcolor[rgb]{0.40,0.40,0.40}{##1}}}
\expandafter\def\csname PY@tok@mh\endcsname{\def\PY@tc##1{\textcolor[rgb]{0.40,0.40,0.40}{##1}}}
\expandafter\def\csname PY@tok@mi\endcsname{\def\PY@tc##1{\textcolor[rgb]{0.40,0.40,0.40}{##1}}}
\expandafter\def\csname PY@tok@il\endcsname{\def\PY@tc##1{\textcolor[rgb]{0.40,0.40,0.40}{##1}}}
\expandafter\def\csname PY@tok@mo\endcsname{\def\PY@tc##1{\textcolor[rgb]{0.40,0.40,0.40}{##1}}}
\expandafter\def\csname PY@tok@ch\endcsname{\let\PY@it=\textit\def\PY@tc##1{\textcolor[rgb]{0.25,0.50,0.50}{##1}}}
\expandafter\def\csname PY@tok@cm\endcsname{\let\PY@it=\textit\def\PY@tc##1{\textcolor[rgb]{0.25,0.50,0.50}{##1}}}
\expandafter\def\csname PY@tok@cpf\endcsname{\let\PY@it=\textit\def\PY@tc##1{\textcolor[rgb]{0.25,0.50,0.50}{##1}}}
\expandafter\def\csname PY@tok@c1\endcsname{\let\PY@it=\textit\def\PY@tc##1{\textcolor[rgb]{0.25,0.50,0.50}{##1}}}
\expandafter\def\csname PY@tok@cs\endcsname{\let\PY@it=\textit\def\PY@tc##1{\textcolor[rgb]{0.25,0.50,0.50}{##1}}}

\def\PYZbs{\char`\\}
\def\PYZus{\char`\_}
\def\PYZob{\char`\{}
\def\PYZcb{\char`\}}
\def\PYZca{\char`\^}
\def\PYZam{\char`\&}
\def\PYZlt{\char`\<}
\def\PYZgt{\char`\>}
\def\PYZsh{\char`\#}
\def\PYZpc{\char`\%}
\def\PYZdl{\char`\$}
\def\PYZhy{\char`\-}
\def\PYZsq{\char`\'}
\def\PYZdq{\char`\"}
\def\PYZti{\char`\~}
% for compatibility with earlier versions
\def\PYZat{@}
\def\PYZlb{[}
\def\PYZrb{]}
\makeatother


    % Exact colors from NB
    \definecolor{incolor}{rgb}{0.0, 0.0, 0.5}
    \definecolor{outcolor}{rgb}{0.545, 0.0, 0.0}



    
    % Prevent overflowing lines due to hard-to-break entities
    \sloppy 
    % Setup hyperref package
    \hypersetup{
      breaklinks=true,  % so long urls are correctly broken across lines
      colorlinks=true,
      urlcolor=urlcolor,
      linkcolor=linkcolor,
      citecolor=citecolor,
      }
    % Slightly bigger margins than the latex defaults
    
    \geometry{verbose,tmargin=1in,bmargin=1in,lmargin=1in,rmargin=1in}
    
    

    \begin{document}
    
    
    \maketitle
    
    

    
    Abstract

 This report will discuss about how to analyse "Linear Time-invariant"
\textbf{LTI} Systems with numerical tools in Python.It uses Laplace
transform to solve the differential equations of the system rather than
time domain analysis,which is very effective as Laplace transform
changes the problem into polynomial equations which are otherwise
differential equations in time domain.So it focuses on how to analyse
the \(LTI\) system when governing system equations with proper initial
conditions given using Laplace transform and convolution.We later
discuss about steady state solutions of electrical circuits and analyse
their response to step inputs.

    \section{Introduction}\label{introduction}

\begin{itemize}
\item
  We analyse the infamous LTI systems in continuous time using Laplace
  transform to find the solutions to the equations governing the system
  with the help of python tools such as Signal toolbox\\
\item
  \(system.impulse \to\) Computes the impulse response of the transfer
  function
\item
  \(sp.lsim \to\) This simulates \(y=u(t)*h(t)\) taking \(u(t)\) and
  \(\mathcal{H(s)}\) as arguments
\item
  \(sp.lti \to\) defines a transfer function from polynomial
  coefficients of numerator and denominator as inputs.
\item
  \$bode() \to \$ It's used to find the magnitude and phase response of
  transfer function
\item
  We use following method to find the Laplace transform of a time domain
  signal, here we use these methods to find laplace of system governing
  differential coefficients
\item
  Some of the equations to follow while finding laplace transform
\end{itemize}

\begin{equation}
    \mathscr{L}\{x(t)\} \to \mathcal{X(s)}
\end{equation}

\begin{equation}
    \mathscr{L}\{\frac{dx(t)}{dt}\} \to \mathcal{sX(s)-x(0^{-})}
\end{equation}

\begin{equation}
    \mathscr{L}\{\frac{d^{2}x(t)}{dt^{2}}\} \to \mathcal{s^{2}X(s)-sx(0^{-})-\dot x(0^{-})}
\end{equation}

\begin{itemize}
\tightlist
\item
  Combining the above equations above, we find the laplace transform of
  a differential equation and analyse them.
\end{itemize}

    \begin{Verbatim}[commandchars=\\\{\}]
{\color{incolor}In [{\color{incolor}22}]:} \PY{c+c1}{\PYZsh{} load libraries and set plot parameters}
         \PY{k+kn}{from} \PY{n+nn}{pylab} \PY{k}{import} \PY{o}{*}
         \PY{o}{\PYZpc{}}\PY{k}{matplotlib} inline
         \PY{k+kn}{from}  \PY{n+nn}{tabulate} \PY{k}{import} \PY{n}{tabulate}
         \PY{k+kn}{import} \PY{n+nn}{scipy}\PY{n+nn}{.}\PY{n+nn}{signal} \PY{k}{as} \PY{n+nn}{sp}
         
         \PY{k+kn}{from} \PY{n+nn}{IPython}\PY{n+nn}{.}\PY{n+nn}{display} \PY{k}{import} \PY{n}{set\PYZus{}matplotlib\PYZus{}formats}
         \PY{n}{set\PYZus{}matplotlib\PYZus{}formats}\PY{p}{(}\PY{l+s+s1}{\PYZsq{}}\PY{l+s+s1}{pdf}\PY{l+s+s1}{\PYZsq{}}\PY{p}{,} \PY{l+s+s1}{\PYZsq{}}\PY{l+s+s1}{png}\PY{l+s+s1}{\PYZsq{}}\PY{p}{)}
         \PY{n}{plt}\PY{o}{.}\PY{n}{rcParams}\PY{p}{[}\PY{l+s+s1}{\PYZsq{}}\PY{l+s+s1}{savefig.dpi}\PY{l+s+s1}{\PYZsq{}}\PY{p}{]} \PY{o}{=} \PY{l+m+mi}{75}
         
         \PY{n}{plt}\PY{o}{.}\PY{n}{rcParams}\PY{p}{[}\PY{l+s+s1}{\PYZsq{}}\PY{l+s+s1}{figure.autolayout}\PY{l+s+s1}{\PYZsq{}}\PY{p}{]} \PY{o}{=} \PY{k+kc}{False}
         \PY{n}{plt}\PY{o}{.}\PY{n}{rcParams}\PY{p}{[}\PY{l+s+s1}{\PYZsq{}}\PY{l+s+s1}{figure.figsize}\PY{l+s+s1}{\PYZsq{}}\PY{p}{]} \PY{o}{=} \PY{l+m+mi}{12}\PY{p}{,} \PY{l+m+mi}{9}
         \PY{n}{plt}\PY{o}{.}\PY{n}{rcParams}\PY{p}{[}\PY{l+s+s1}{\PYZsq{}}\PY{l+s+s1}{axes.labelsize}\PY{l+s+s1}{\PYZsq{}}\PY{p}{]} \PY{o}{=} \PY{l+m+mi}{18}
         \PY{n}{plt}\PY{o}{.}\PY{n}{rcParams}\PY{p}{[}\PY{l+s+s1}{\PYZsq{}}\PY{l+s+s1}{axes.titlesize}\PY{l+s+s1}{\PYZsq{}}\PY{p}{]} \PY{o}{=} \PY{l+m+mi}{20}
         \PY{n}{plt}\PY{o}{.}\PY{n}{rcParams}\PY{p}{[}\PY{l+s+s1}{\PYZsq{}}\PY{l+s+s1}{font.size}\PY{l+s+s1}{\PYZsq{}}\PY{p}{]} \PY{o}{=} \PY{l+m+mi}{16}
         \PY{n}{plt}\PY{o}{.}\PY{n}{rcParams}\PY{p}{[}\PY{l+s+s1}{\PYZsq{}}\PY{l+s+s1}{lines.linewidth}\PY{l+s+s1}{\PYZsq{}}\PY{p}{]} \PY{o}{=} \PY{l+m+mf}{2.0}
         \PY{n}{plt}\PY{o}{.}\PY{n}{rcParams}\PY{p}{[}\PY{l+s+s1}{\PYZsq{}}\PY{l+s+s1}{lines.markersize}\PY{l+s+s1}{\PYZsq{}}\PY{p}{]} \PY{o}{=} \PY{l+m+mi}{6}
         \PY{n}{plt}\PY{o}{.}\PY{n}{rcParams}\PY{p}{[}\PY{l+s+s1}{\PYZsq{}}\PY{l+s+s1}{legend.fontsize}\PY{l+s+s1}{\PYZsq{}}\PY{p}{]} \PY{o}{=} \PY{l+m+mi}{14}
         \PY{n}{plt}\PY{o}{.}\PY{n}{rcParams}\PY{p}{[}\PY{l+s+s1}{\PYZsq{}}\PY{l+s+s1}{legend.numpoints}\PY{l+s+s1}{\PYZsq{}}\PY{p}{]} \PY{o}{=} \PY{l+m+mi}{2}
         \PY{n}{plt}\PY{o}{.}\PY{n}{rcParams}\PY{p}{[}\PY{l+s+s1}{\PYZsq{}}\PY{l+s+s1}{legend.loc}\PY{l+s+s1}{\PYZsq{}}\PY{p}{]} \PY{o}{=} \PY{l+s+s1}{\PYZsq{}}\PY{l+s+s1}{best}\PY{l+s+s1}{\PYZsq{}}
         \PY{n}{plt}\PY{o}{.}\PY{n}{rcParams}\PY{p}{[}\PY{l+s+s1}{\PYZsq{}}\PY{l+s+s1}{legend.fancybox}\PY{l+s+s1}{\PYZsq{}}\PY{p}{]} \PY{o}{=} \PY{k+kc}{True}
         \PY{n}{plt}\PY{o}{.}\PY{n}{rcParams}\PY{p}{[}\PY{l+s+s1}{\PYZsq{}}\PY{l+s+s1}{legend.shadow}\PY{l+s+s1}{\PYZsq{}}\PY{p}{]} \PY{o}{=} \PY{k+kc}{True}
         \PY{n}{plt}\PY{o}{.}\PY{n}{rcParams}\PY{p}{[}\PY{l+s+s1}{\PYZsq{}}\PY{l+s+s1}{text.usetex}\PY{l+s+s1}{\PYZsq{}}\PY{p}{]} \PY{o}{=} \PY{k+kc}{True}
         \PY{n}{plt}\PY{o}{.}\PY{n}{rcParams}\PY{p}{[}\PY{l+s+s1}{\PYZsq{}}\PY{l+s+s1}{font.family}\PY{l+s+s1}{\PYZsq{}}\PY{p}{]} \PY{o}{=} \PY{l+s+s2}{\PYZdq{}}\PY{l+s+s2}{serif}\PY{l+s+s2}{\PYZdq{}}
         \PY{n}{plt}\PY{o}{.}\PY{n}{rcParams}\PY{p}{[}\PY{l+s+s1}{\PYZsq{}}\PY{l+s+s1}{font.serif}\PY{l+s+s1}{\PYZsq{}}\PY{p}{]} \PY{o}{=} \PY{l+s+s2}{\PYZdq{}}\PY{l+s+s2}{cm}\PY{l+s+s2}{\PYZdq{}}
         \PY{n}{plt}\PY{o}{.}\PY{n}{rcParams}\PY{p}{[}\PY{l+s+s1}{\PYZsq{}}\PY{l+s+s1}{text.latex.preamble}\PY{l+s+s1}{\PYZsq{}}\PY{p}{]} \PY{o}{=} \PY{l+s+sa}{r}\PY{l+s+s2}{\PYZdq{}}\PY{l+s+s2}{\PYZbs{}}\PY{l+s+s2}{usepackage}\PY{l+s+si}{\PYZob{}subdepth\PYZcb{}}\PY{l+s+s2}{, }\PY{l+s+s2}{\PYZbs{}}\PY{l+s+s2}{usepackage}\PY{l+s+si}{\PYZob{}type1cm\PYZcb{}}\PY{l+s+s2}{\PYZdq{}}
\end{Verbatim}


    \subsection{Question 1 \& 2:}\label{question-1-2}

\begin{itemize}
\tightlist
\item
  To solve for the time repsonse of the spring mass system,whose driving
  force varies as \(f(t)\) given as
\end{itemize}

\begin{equation}
f(t) = \cos(1.5t) e^{-0.5t}u_0(t)
\end{equation}

\begin{itemize}
\tightlist
\item
  Laplace transform of \(f(t)\) using equations (1),(2) \& (3) given
  above
\end{itemize}

\begin{equation}
    \mathcal{F(s)} = \frac{s+0.5}{(s+0.5)^2 + 2.25}
\end{equation}

\begin{itemize}
\tightlist
\item
  Spring satisfies the below equation with \(x(0)\) \(=\) \(0\) and
  \(\dot x(0)\) \(=\) \(0\) for \$ 0 \leq t \leq 50s\$.
\end{itemize}

\begin{equation}
\ddot x + 2.25x = f(t)
\end{equation}

\begin{itemize}
\item
  So we take laplace transform of the equation given above with given
  intial conditions and in generalised form considering
  \$w\_\{0\}\^{}\{2\} = 2.25 \$ in the differential equation above,
  natural frequency of the system is \(w_0 = 1.5 rads^{-1}\),and decay
  factor of \(f(t)\) as \(d = 0.5\) and frequency of the input as
  \(w = 1.5 rads^{-1}\) in this question.
\item
  In general we get
\end{itemize}

\begin{equation}
    \mathcal{X(s)} = \frac{s+d}{((s+d)^2 + w^2)(s^2 + w_{0}^{2})}
\end{equation}

\begin{itemize}
\tightlist
\item
  In question 1 we get with given values and \(d = 0.5\)
\end{itemize}

\begin{equation}
    \mathcal{X(s)} = \frac{s+0.5}{((s+0.5)^2 + 2.25)(s^2 + 2.25)}
\end{equation}

\begin{itemize}
\tightlist
\item
  Solve the above problem with much smaller decay with same initial
  conditions, now \(f(t)\) is as follows
\end{itemize}

\begin{equation}
f(t) = \cos(1.5t) e^{-0.05t}u_0(t)
\end{equation}

\begin{itemize}
\tightlist
\item
  So in question 2 we get with \(d = 0.05\)
\end{itemize}

\begin{equation}
    \mathcal{X(s)} = \frac{s+0.05}{((s+0.05)^2 + 2.25)(s^2 + 2.25)}
\end{equation}

\begin{itemize}
\tightlist
\item
  To solve for \(x(t)\) displacement for each of the cases using Laplace
  transform with python tools such as \(system.impulse\) and plot them.
\item
  To analyse the plots obtained and discuss the effect of decay on
  \(x(t)\).
\end{itemize}

    \begin{Verbatim}[commandchars=\\\{\}]
{\color{incolor}In [{\color{incolor}23}]:} \PY{l+s+sd}{\PYZsq{}\PYZsq{}\PYZsq{}}
         \PY{l+s+sd}{function to solve for x(t)}
         \PY{l+s+sd}{Arguments : x0    \PYZhy{} x(0)}
         \PY{l+s+sd}{            x0dot \PYZhy{} derivative of x at x=0}
         \PY{l+s+sd}{            decay \PYZhy{} decay factor}
         \PY{l+s+sd}{            freq  \PYZhy{} frequency at which spring operates(resonant case)}
         \PY{l+s+sd}{Returns   : t and x(t)}
         \PY{l+s+sd}{\PYZsq{}\PYZsq{}\PYZsq{}}
         \PY{k}{def} \PY{n+nf}{laplaceSolver}\PY{p}{(}\PY{n}{x0}\PY{p}{,}\PY{n}{x0dot}\PY{p}{,}\PY{n}{decay}\PY{p}{,}\PY{n}{freq}\PY{p}{)}\PY{p}{:}    
             \PY{n}{Xnum} \PY{o}{=} \PY{n}{poly1d}\PY{p}{(}\PY{p}{[}\PY{l+m+mi}{1}\PY{p}{,}\PY{n}{decay}\PY{p}{]}\PY{p}{)}\PY{o}{+}\PY{n}{polymul}\PY{p}{(}\PY{p}{[}\PY{n}{x0}\PY{p}{,}\PY{n}{x0dot}\PY{p}{]}\PY{p}{,}\PY{p}{[}\PY{l+m+mi}{1}\PY{p}{,}\PY{l+m+mi}{2}\PY{o}{*}\PY{n}{decay}\PY{p}{,}\PY{p}{(}\PY{n+nb}{pow}\PY{p}{(}\PY{n}{freq}\PY{p}{,}\PY{l+m+mi}{2}\PY{p}{)}\PY{o}{+}\PY{n+nb}{pow}\PY{p}{(}\PY{n}{decay}\PY{p}{,}\PY{l+m+mi}{2}\PY{p}{)}\PY{p}{)}\PY{p}{]}\PY{p}{)}
             \PY{n}{Xden} \PY{o}{=} \PY{n}{polymul}\PY{p}{(}\PY{p}{[}\PY{l+m+mi}{1}\PY{p}{,}\PY{l+m+mi}{0}\PY{p}{,}\PY{n+nb}{pow}\PY{p}{(}\PY{n}{freq}\PY{p}{,}\PY{l+m+mi}{2}\PY{p}{)}\PY{p}{]}\PY{p}{,}\PY{p}{[}\PY{l+m+mi}{1}\PY{p}{,}\PY{l+m+mi}{2}\PY{o}{*}\PY{n}{decay}\PY{p}{,}\PY{p}{(}\PY{n+nb}{pow}\PY{p}{(}\PY{n}{freq}\PY{p}{,}\PY{l+m+mi}{2}\PY{p}{)}\PY{o}{+}\PY{n+nb}{pow}\PY{p}{(}\PY{n}{decay}\PY{p}{,}\PY{l+m+mi}{2}\PY{p}{)}\PY{p}{)}\PY{p}{]}\PY{p}{)}
             
             \PY{c+c1}{\PYZsh{}Computes the impulse response of the transfer function}
             \PY{n}{Xs} \PY{o}{=} \PY{n}{sp}\PY{o}{.}\PY{n}{lti}\PY{p}{(}\PY{n}{Xnum}\PY{p}{,}\PY{n}{Xden}\PY{p}{)}
             \PY{n}{t}\PY{p}{,}\PY{n}{x}\PY{o}{=}\PY{n}{sp}\PY{o}{.}\PY{n}{impulse}\PY{p}{(}\PY{n}{Xs}\PY{p}{,}\PY{k+kc}{None}\PY{p}{,}\PY{n}{linspace}\PY{p}{(}\PY{l+m+mi}{0}\PY{p}{,}\PY{l+m+mi}{100}\PY{p}{,}\PY{l+m+mi}{10000}\PY{p}{)}\PY{p}{)}
             \PY{k}{return} \PY{n}{t}\PY{p}{,}\PY{n}{x}
\end{Verbatim}


    \begin{Verbatim}[commandchars=\\\{\}]
{\color{incolor}In [{\color{incolor}24}]:} \PY{c+c1}{\PYZsh{} solving for two cases with decay of 0.5 and 0.05}
         
         \PY{n}{t1}\PY{p}{,}\PY{n}{x1} \PY{o}{=} \PY{n}{laplaceSolver}\PY{p}{(}\PY{l+m+mi}{0}\PY{p}{,}\PY{l+m+mi}{0}\PY{p}{,}\PY{l+m+mf}{0.5}\PY{p}{,}\PY{l+m+mf}{1.5}\PY{p}{)}
         \PY{n}{t2}\PY{p}{,}\PY{n}{x2} \PY{o}{=} \PY{n}{laplaceSolver}\PY{p}{(}\PY{l+m+mi}{0}\PY{p}{,}\PY{l+m+mi}{0}\PY{p}{,}\PY{l+m+mf}{0.05}\PY{p}{,}\PY{l+m+mf}{1.5}\PY{p}{)}
         \PY{n}{t3}\PY{p}{,}\PY{n}{x3} \PY{o}{=} \PY{n}{laplaceSolver}\PY{p}{(}\PY{l+m+mi}{0}\PY{p}{,}\PY{l+m+mi}{0}\PY{p}{,}\PY{l+m+mf}{0.005}\PY{p}{,}\PY{l+m+mf}{1.5}\PY{p}{)}
         \PY{n}{t4}\PY{p}{,}\PY{n}{x4} \PY{o}{=} \PY{n}{laplaceSolver}\PY{p}{(}\PY{l+m+mi}{0}\PY{p}{,}\PY{l+m+mi}{0}\PY{p}{,}\PY{l+m+mi}{5}\PY{p}{,}\PY{l+m+mf}{1.5}\PY{p}{)}
\end{Verbatim}


    \begin{Verbatim}[commandchars=\\\{\}]
{\color{incolor}In [{\color{incolor}25}]:} \PY{c+c1}{\PYZsh{}plot of x(t) with decay of 0.5}
         
         \PY{n}{fig1a} \PY{o}{=} \PY{n}{figure}\PY{p}{(}\PY{p}{)}
         \PY{n}{ax1a} \PY{o}{=} \PY{n}{fig1a}\PY{o}{.}\PY{n}{add\PYZus{}subplot}\PY{p}{(}\PY{l+m+mi}{111}\PY{p}{)}
         \PY{n}{ax1a}\PY{o}{.}\PY{n}{plot}\PY{p}{(}\PY{n}{t1}\PY{p}{,}\PY{n}{x1}\PY{p}{,}\PY{l+s+s1}{\PYZsq{}}\PY{l+s+s1}{b}\PY{l+s+s1}{\PYZsq{}}\PY{p}{,}\PY{n}{label}\PY{o}{=}\PY{l+s+s2}{\PYZdq{}}\PY{l+s+s2}{decay = 0.5}\PY{l+s+s2}{\PYZdq{}}\PY{p}{)}
         \PY{n}{ax1a}\PY{o}{.}\PY{n}{legend}\PY{p}{(}\PY{p}{)}
         \PY{n}{title}\PY{p}{(}\PY{l+s+sa}{r}\PY{l+s+s2}{\PYZdq{}}\PY{l+s+s2}{Figure 1a: \PYZdl{}x(t)\PYZdl{} of spring system}\PY{l+s+s2}{\PYZdq{}}\PY{p}{)}
         \PY{n}{xlabel}\PY{p}{(}\PY{l+s+sa}{r}\PY{l+s+s2}{\PYZdq{}}\PY{l+s+s2}{\PYZdl{}t }\PY{l+s+s2}{\PYZbs{}}\PY{l+s+s2}{to \PYZdl{}}\PY{l+s+s2}{\PYZdq{}}\PY{p}{)}
         \PY{n}{ylabel}\PY{p}{(}\PY{l+s+sa}{r}\PY{l+s+s2}{\PYZdq{}}\PY{l+s+s2}{\PYZdl{}x(t) }\PY{l+s+s2}{\PYZbs{}}\PY{l+s+s2}{to \PYZdl{}}\PY{l+s+s2}{\PYZdq{}}\PY{p}{)}
         \PY{n}{grid}\PY{p}{(}\PY{p}{)}
         \PY{n}{savefig}\PY{p}{(}\PY{l+s+s2}{\PYZdq{}}\PY{l+s+s2}{Figure1a.jpg}\PY{l+s+s2}{\PYZdq{}}\PY{p}{)}
\end{Verbatim}


    \begin{center}
    \adjustimage{max size={0.9\linewidth}{0.9\paperheight}}{output_6_0.pdf}
    \end{center}
    { \hspace*{\fill} \\}
    
    \begin{Verbatim}[commandchars=\\\{\}]
{\color{incolor}In [{\color{incolor}26}]:} \PY{c+c1}{\PYZsh{}plot of x(t) with decay values of 0.5 and 0.05}
         \PY{n}{fig1} \PY{o}{=} \PY{n}{figure}\PY{p}{(}\PY{p}{)}
         \PY{n}{ax1} \PY{o}{=} \PY{n}{fig1}\PY{o}{.}\PY{n}{add\PYZus{}subplot}\PY{p}{(}\PY{l+m+mi}{111}\PY{p}{)}
         \PY{n}{ax1}\PY{o}{.}\PY{n}{plot}\PY{p}{(}\PY{n}{t1}\PY{p}{,}\PY{n}{x1}\PY{p}{,}\PY{l+s+s1}{\PYZsq{}}\PY{l+s+s1}{b}\PY{l+s+s1}{\PYZsq{}}\PY{p}{,}\PY{n}{label}\PY{o}{=}\PY{l+s+s2}{\PYZdq{}}\PY{l+s+s2}{decay = 0.5}\PY{l+s+s2}{\PYZdq{}}\PY{p}{)}
         \PY{n}{ax1}\PY{o}{.}\PY{n}{plot}\PY{p}{(}\PY{n}{t2}\PY{p}{,}\PY{n}{x2}\PY{p}{,}\PY{l+s+s1}{\PYZsq{}}\PY{l+s+s1}{r}\PY{l+s+s1}{\PYZsq{}}\PY{p}{,}\PY{n}{label}\PY{o}{=}\PY{l+s+s2}{\PYZdq{}}\PY{l+s+s2}{decay = 0.05}\PY{l+s+s2}{\PYZdq{}}\PY{p}{)}
         \PY{n}{ax1}\PY{o}{.}\PY{n}{plot}\PY{p}{(}\PY{n}{t3}\PY{p}{,}\PY{n}{x3}\PY{p}{,}\PY{l+s+s1}{\PYZsq{}}\PY{l+s+s1}{g}\PY{l+s+s1}{\PYZsq{}}\PY{p}{,}\PY{n}{label}\PY{o}{=}\PY{l+s+s2}{\PYZdq{}}\PY{l+s+s2}{decay = 0.005}\PY{l+s+s2}{\PYZdq{}}\PY{p}{)}
         \PY{n}{ax1}\PY{o}{.}\PY{n}{plot}\PY{p}{(}\PY{n}{t4}\PY{p}{,}\PY{n}{x4}\PY{p}{,}\PY{l+s+s1}{\PYZsq{}}\PY{l+s+s1}{k}\PY{l+s+s1}{\PYZsq{}}\PY{p}{,}\PY{n}{label}\PY{o}{=}\PY{l+s+s2}{\PYZdq{}}\PY{l+s+s2}{decay = 5}\PY{l+s+s2}{\PYZdq{}}\PY{p}{)}
         
         \PY{n}{ax1}\PY{o}{.}\PY{n}{legend}\PY{p}{(}\PY{p}{)}
         \PY{n}{title}\PY{p}{(}\PY{l+s+sa}{r}\PY{l+s+s2}{\PYZdq{}}\PY{l+s+s2}{Figure 1b: \PYZdl{}x(t)\PYZdl{} of spring system as function of decay value of \PYZdl{}f(t)\PYZdl{}}\PY{l+s+s2}{\PYZdq{}}\PY{p}{)}
         \PY{n}{xlabel}\PY{p}{(}\PY{l+s+sa}{r}\PY{l+s+s2}{\PYZdq{}}\PY{l+s+s2}{\PYZdl{}t }\PY{l+s+s2}{\PYZbs{}}\PY{l+s+s2}{to \PYZdl{}}\PY{l+s+s2}{\PYZdq{}}\PY{p}{)}
         \PY{n}{ylabel}\PY{p}{(}\PY{l+s+sa}{r}\PY{l+s+s2}{\PYZdq{}}\PY{l+s+s2}{\PYZdl{}x(t) }\PY{l+s+s2}{\PYZbs{}}\PY{l+s+s2}{to \PYZdl{}}\PY{l+s+s2}{\PYZdq{}}\PY{p}{)}
         \PY{n}{grid}\PY{p}{(}\PY{p}{)}
         \PY{n}{savefig}\PY{p}{(}\PY{l+s+s2}{\PYZdq{}}\PY{l+s+s2}{Figure1b.jpg}\PY{l+s+s2}{\PYZdq{}}\PY{p}{)}
\end{Verbatim}


    \begin{center}
    \adjustimage{max size={0.9\linewidth}{0.9\paperheight}}{output_7_0.pdf}
    \end{center}
    { \hspace*{\fill} \\}
    
    \subsubsection{Results and Discussion :}\label{results-and-discussion}

\begin{itemize}
\item
  As we observe the plot that for smaller decay of \(e^{-0.05t}\) and
  \(e^{-0.005t}\),etc.\(x(t)\) has large amplitude and its growing as
  time increases and oscillates.
\item
  Whereas the for higher decay values the amplitude of \(x(t)\) is very
  small but since its second order system it oscillates.
\item
  And If we observe the Figure 1a, amplitude growth stopped and settles
  quicker than for smaller decay values whereas for small decay values
  the \(x(t)\) amplitude settling time is larger because as given below:
\item
  Our input \(f(t)\) to the system has natural frequency that is
  \(w=w_0\), so its a resonant case,so the solution of differentail
  equation for sinusoidal inputs from observing the plot can be of the
  form \(te^{-dt}\cos (w_0 t)\) so for smaller decay values the graph
  takes more time neutralise the growing effect of \(t\) in the
  solution.
\item
  So to conclude for small decay , \(x(t)\) has large amplitude and the
  time required for it settle or saturate to a certain maximum amplitude
  is higher compared to large decay case
\end{itemize}

    \subsection{Question 3:}\label{question-3}

\begin{itemize}
\item
  In an LTI system. \(f(t)\) is the input, and \(x(t)\) is the output.
\item
  To Obtain the system transfer function
  \(\frac{\mathcal {X(s)}}{ \mathcal {F(s)}}\)
\item
  we use \(signal.lsim\) to simulate the problem.
\item
  Here we try to plot the system response for different values of
  excitation frequecies i.e input frequencies with natural frequency of
  the system as \(w_0 = 1.5 rads^{-1}\)
\item
  So using a for loop, we sweep the frequency \(w\) of the \(f(t)\) from
  \(1.4\) to \(1.6\) in steps of \(0.05\) keeping the exponent as
  \(e^{−0.05t}\) that is \(d=0.05\) and plot the resulting responses.
\item
  So with above conditions laplace transform of \(x(t)\) is
\end{itemize}

\begin{equation}
    \mathcal{X(s)} = \frac{s+0.05}{((s+0.05)^2 + w^2)(s^2 + 2.25)}
\end{equation}

\begin{itemize}
\tightlist
\item
  So we transfer function of the system is
\end{itemize}

\begin{equation}
    \mathcal{H(s)} = \frac{s+0.05}{((s+0.05)^2 + w^2)(s^2 + 2.25)}
\end{equation}

\begin{equation}
    \mathcal{H(s)} = \frac{s+0.05}{((s+0.05)^2 + w^2)(s^2 + 2.25)}
\end{equation}

\begin{itemize}
\tightlist
\item
  Using this we analyse the results.
\end{itemize}

    \begin{Verbatim}[commandchars=\\\{\}]
{\color{incolor}In [{\color{incolor}27}]:} \PY{l+s+sd}{\PYZsq{}\PYZsq{}\PYZsq{}}
         \PY{l+s+sd}{function to return f(t) for various parameters}
         \PY{l+s+sd}{Arguments : t     \PYZhy{} time}
         \PY{l+s+sd}{            freq  \PYZhy{} frequency at system is excited}
         \PY{l+s+sd}{            decay \PYZhy{} decay factor}
         \PY{l+s+sd}{Returns   : t and x(t)}
         \PY{l+s+sd}{\PYZsq{}\PYZsq{}\PYZsq{}}
         
         \PY{k}{def} \PY{n+nf}{f}\PY{p}{(}\PY{n}{t}\PY{p}{,}\PY{n}{freq}\PY{p}{,}\PY{n}{decay}\PY{p}{)}\PY{p}{:}
             \PY{k}{return} \PY{n}{cos}\PY{p}{(}\PY{n}{freq}\PY{o}{*}\PY{n}{t}\PY{p}{)}\PY{o}{*}\PY{n}{exp}\PY{p}{(}\PY{o}{\PYZhy{}}\PY{n}{decay}\PY{o}{*}\PY{n}{t}\PY{p}{)}
\end{Verbatim}


    \begin{Verbatim}[commandchars=\\\{\}]
{\color{incolor}In [{\color{incolor}28}]:} \PY{l+s+sd}{\PYZsq{}\PYZsq{}\PYZsq{}}
         \PY{l+s+sd}{function to solve for Transfer function H(s)}
         \PY{l+s+sd}{Arguments : x0    \PYZhy{} x(0)}
         \PY{l+s+sd}{            x0dot \PYZhy{} derivative of x at x=0}
         \PY{l+s+sd}{            decay \PYZhy{} decay factor}
         \PY{l+s+sd}{            freq  \PYZhy{} frequency at which system is excited}
         \PY{l+s+sd}{Returns   : Hs    \PYZhy{} transfer function of the system}
         \PY{l+s+sd}{\PYZsq{}\PYZsq{}\PYZsq{}}
         
         \PY{k}{def} \PY{n+nf}{getTransferfunc}\PY{p}{(}\PY{n}{x0}\PY{p}{,}\PY{n}{x0dot}\PY{p}{,}\PY{n}{decay}\PY{p}{,}\PY{n}{freq}\PY{p}{)}\PY{p}{:}    
             
             \PY{c+c1}{\PYZsh{}natural frequency is 1.5rad/s}
             \PY{n}{nat\PYZus{}freq} \PY{o}{=} \PY{l+m+mf}{1.5}
             \PY{n}{Hnum} \PY{o}{=} \PY{n}{poly1d}\PY{p}{(}\PY{p}{[}\PY{l+m+mi}{1}\PY{p}{,}\PY{n}{decay}\PY{p}{]}\PY{p}{)}\PY{o}{+}\PY{n}{polymul}\PY{p}{(}\PY{p}{[}\PY{n}{x0}\PY{p}{,}\PY{n}{x0dot}\PY{p}{]}\PY{p}{,}
                                              \PY{p}{[}\PY{l+m+mi}{1}\PY{p}{,}\PY{l+m+mi}{2}\PY{o}{*}\PY{n}{decay}\PY{p}{,}\PY{p}{(}\PY{n+nb}{pow}\PY{p}{(}\PY{n}{freq}\PY{p}{,}\PY{l+m+mi}{2}\PY{p}{)}\PY{o}{+}\PY{n+nb}{pow}\PY{p}{(}\PY{n}{decay}\PY{p}{,}\PY{l+m+mi}{2}\PY{p}{)}\PY{p}{)}\PY{p}{]}\PY{p}{)}\PY{o}{+}\PY{n}{poly1d}\PY{p}{(}\PY{p}{[}\PY{l+m+mi}{1}\PY{p}{]}\PY{p}{)}
             \PY{n}{Hden} \PY{o}{=} \PY{n}{polymul}\PY{p}{(}\PY{p}{[}\PY{l+m+mi}{1}\PY{p}{,}\PY{l+m+mi}{0}\PY{p}{,}\PY{n+nb}{pow}\PY{p}{(}\PY{n}{nat\PYZus{}freq}\PY{p}{,}\PY{l+m+mi}{2}\PY{p}{)}\PY{p}{]}\PY{p}{,}\PY{p}{[}\PY{l+m+mi}{1}\PY{p}{,}\PY{n}{decay}\PY{p}{]}\PY{p}{)}
             
             \PY{c+c1}{\PYZsh{}Computes the impulse response of the transfer function}
             \PY{n}{Hs} \PY{o}{=} \PY{n}{sp}\PY{o}{.}\PY{n}{lti}\PY{p}{(}\PY{n}{Hnum}\PY{p}{,}\PY{n}{Hden}\PY{p}{)}
             \PY{k}{return} \PY{n}{Hs}
\end{Verbatim}


    \begin{Verbatim}[commandchars=\\\{\}]
{\color{incolor}In [{\color{incolor}47}]:} \PY{c+c1}{\PYZsh{}Plot of x(t) with different input frequencies}
         \PY{n}{fig2} \PY{o}{=} \PY{n}{figure}\PY{p}{(}\PY{p}{)}
         \PY{n}{ax2} \PY{o}{=} \PY{n}{fig2}\PY{o}{.}\PY{n}{add\PYZus{}subplot}\PY{p}{(}\PY{l+m+mi}{111}\PY{p}{)}
         \PY{n}{title}\PY{p}{(}\PY{l+s+sa}{r}\PY{l+s+s2}{\PYZdq{}}\PY{l+s+s2}{Figure 2: \PYZdl{}x(t)\PYZdl{} of spring system with varying Frequency}\PY{l+s+s2}{\PYZdq{}}\PY{p}{)}
         
         \PY{c+c1}{\PYZsh{}For loop  to plot x(t) for different values of freq}
         \PY{k}{for} \PY{n}{w} \PY{o+ow}{in} \PY{n}{arange}\PY{p}{(}\PY{l+m+mf}{1.4}\PY{p}{,}\PY{l+m+mf}{1.6}\PY{p}{,}\PY{l+m+mf}{0.05}\PY{p}{)}\PY{p}{:}
             \PY{n}{decay} \PY{o}{=} \PY{l+m+mf}{0.05}
             \PY{n}{H} \PY{o}{=} \PY{n}{getTransferfunc}\PY{p}{(}\PY{l+m+mi}{0}\PY{p}{,}\PY{l+m+mi}{0}\PY{p}{,}\PY{n}{decay}\PY{p}{,}\PY{n}{w}\PY{p}{)}
             \PY{n}{t} \PY{o}{=} \PY{n}{linspace}\PY{p}{(}\PY{l+m+mi}{0}\PY{p}{,}\PY{l+m+mi}{200}\PY{p}{,}\PY{l+m+mi}{10000}\PY{p}{)}
             \PY{n}{t}\PY{p}{,}\PY{n}{y}\PY{p}{,}\PY{n}{svec}\PY{o}{=}\PY{n}{sp}\PY{o}{.}\PY{n}{lsim}\PY{p}{(}\PY{n}{H}\PY{p}{,}\PY{n}{f}\PY{p}{(}\PY{n}{t}\PY{p}{,}\PY{n}{w}\PY{p}{,}\PY{n}{decay}\PY{p}{)}\PY{p}{,}\PY{n}{t}\PY{p}{)}
             \PY{n}{legnd} \PY{o}{=} \PY{l+s+s2}{\PYZdq{}}\PY{l+s+s2}{\PYZdl{}w\PYZdl{} = }\PY{l+s+si}{\PYZpc{}g}\PY{l+s+s2}{ rad/s}\PY{l+s+s2}{\PYZdq{}}\PY{o}{\PYZpc{}}\PY{p}{(}\PY{n}{w}\PY{p}{)} 
             \PY{n}{ax2}\PY{o}{.}\PY{n}{plot}\PY{p}{(}\PY{n}{t}\PY{p}{,}\PY{n}{y}\PY{p}{,}\PY{n}{label}\PY{o}{=}\PY{n}{legnd}\PY{p}{)} 
             \PY{n}{ax2}\PY{o}{.}\PY{n}{legend}\PY{p}{(}\PY{p}{)}
             
         \PY{n}{xlabel}\PY{p}{(}\PY{l+s+sa}{r}\PY{l+s+s2}{\PYZdq{}}\PY{l+s+s2}{\PYZdl{}t }\PY{l+s+s2}{\PYZbs{}}\PY{l+s+s2}{to \PYZdl{}}\PY{l+s+s2}{\PYZdq{}}\PY{p}{)}
         \PY{n}{ylabel}\PY{p}{(}\PY{l+s+sa}{r}\PY{l+s+s2}{\PYZdq{}}\PY{l+s+s2}{\PYZdl{}x(t) }\PY{l+s+s2}{\PYZbs{}}\PY{l+s+s2}{to \PYZdl{}}\PY{l+s+s2}{\PYZdq{}}\PY{p}{)}
         \PY{n}{grid}\PY{p}{(}\PY{p}{)}
         \PY{n}{savefig}\PY{p}{(}\PY{l+s+s2}{\PYZdq{}}\PY{l+s+s2}{Figure2.jpg}\PY{l+s+s2}{\PYZdq{}}\PY{p}{)}
         \PY{n}{show}\PY{p}{(}\PY{p}{)}
\end{Verbatim}


    \begin{center}
    \adjustimage{max size={0.9\linewidth}{0.9\paperheight}}{output_12_0.pdf}
    \end{center}
    { \hspace*{\fill} \\}
    
    \begin{Verbatim}[commandchars=\\\{\}]
{\color{incolor}In [{\color{incolor}51}]:} \PY{c+c1}{\PYZsh{}Calculating bode plot for transfer function H}
         \PY{n}{w1}\PY{p}{,}\PY{n}{mag}\PY{p}{,}\PY{n}{phi}\PY{o}{=}\PY{n}{H}\PY{o}{.}\PY{n}{bode}\PY{p}{(}\PY{p}{)}
         
         \PY{c+c1}{\PYZsh{}Plot of x(t) with different input frequencies}
         \PY{n}{fig3} \PY{o}{=} \PY{n}{figure}\PY{p}{(}\PY{p}{)}
         \PY{n}{ax3} \PY{o}{=} \PY{n}{fig3}\PY{o}{.}\PY{n}{add\PYZus{}subplot}\PY{p}{(}\PY{l+m+mi}{111}\PY{p}{)}
         \PY{n}{title}\PY{p}{(}\PY{l+s+s2}{\PYZdq{}}\PY{l+s+s2}{Figure 3: \PYZdl{}|H(j}\PY{l+s+s2}{\PYZbs{}}\PY{l+s+s2}{omega)|\PYZdl{} \PYZhy{} Bode plot of the transfer function}\PY{l+s+s2}{\PYZdq{}}\PY{p}{)}
         
         \PY{n}{ax3}\PY{o}{.}\PY{n}{semilogx}\PY{p}{(}\PY{n}{w1}\PY{p}{,}\PY{n}{mag}\PY{p}{)} 
         \PY{n}{xlabel}\PY{p}{(}\PY{l+s+sa}{r}\PY{l+s+s2}{\PYZdq{}}\PY{l+s+s2}{\PYZdl{}}\PY{l+s+s2}{\PYZbs{}}\PY{l+s+s2}{omega\PYZdl{}}\PY{l+s+s2}{\PYZdq{}}\PY{p}{)}
         \PY{n}{ylabel}\PY{p}{(}\PY{l+s+sa}{r}\PY{l+s+s2}{\PYZdq{}}\PY{l+s+s2}{\PYZdl{} 20}\PY{l+s+s2}{\PYZbs{}}\PY{l+s+s2}{log|H(jw)| }\PY{l+s+s2}{\PYZbs{}}\PY{l+s+s2}{to \PYZdl{}}\PY{l+s+s2}{\PYZdq{}}\PY{p}{)}
         \PY{n}{grid}\PY{p}{(}\PY{p}{)}
         \PY{n}{savefig}\PY{p}{(}\PY{l+s+s2}{\PYZdq{}}\PY{l+s+s2}{Figure3.jpg}\PY{l+s+s2}{\PYZdq{}}\PY{p}{)}
         \PY{n}{show}\PY{p}{(}\PY{p}{)}
\end{Verbatim}


    \begin{center}
    \adjustimage{max size={0.9\linewidth}{0.9\paperheight}}{output_13_0.pdf}
    \end{center}
    { \hspace*{\fill} \\}
    
    \subsubsection{Results and Discussion:}\label{results-and-discussion}

\begin{itemize}
\tightlist
\item
  As we observe the magnitude response of the transfer function of the
  system in plot above, we observe that it has sharp peak for frequency
  \(\omega = 1.5 rads^{-1}\) because it's the case of resonance as
  driving frequency and resonant frequency matches.So the gain increases
  sharply at that frequency and decreases narrowly for closer
  frequencies.That's why we see large amplitude for \(x(t)\) when
  \(\omega = 1.5\).
\item
  As we observe figure 2 that,when we force (we are basically exciting
  the spring mass system with \(f(t)\)) the system with \(w \neq w_0\)
  where \(w_0 = 1.5\) and \(w \approx w_0\) i.e around 1.5
  \(rads^{-1}\),since intially the system is at rest,so when f(t) is
  forced upon it,they are in phase,so constructive superposition occurs
  so the amplitude of \(x(t)\) increases in same fashion for \(w\) very
  close to \(w_0\) at starting , after that since the forced and natural
  frequency are not same,it is detuned so amplitude after peak amplitude
  starts falling slightly but at steady state all forced response dies
  out,so basically \(x(t)\) will vary only with natural mode since
  forced response died out eventually.
\item
  Whereas for \(w=w_0\) case resonant occurs,so system and forcing input
  have same frequency so,the forced response adds up to natural response
  basically 'tuned',so the amplitude is very high in steady state.
\end{itemize}

    \subsection{Question 4:}\label{question-4}

\begin{itemize}
\item
  To Solve for a coupled spring problem:
\item
  System satisfies the below equation with \(x(0) = 1\) and
  \(\dot x(0) = y(0) = \dot y(0) = 0\).
\end{itemize}

\begin{equation}
\ddot x + (x-y) = 0
\end{equation}

\begin{equation}
\ddot y + 2(y-x)= 0
\end{equation}

\begin{itemize}
\item
  Solve for \(x(t)\) and \(y(t)\) for \$ 0 \leq t \leq 20s\$ by taking
  laplace transform of both equations given above and solve for
  \(\mathcal {X(s)}\) and \(\mathcal{Y(s)}\) using substitution method.
\item
  Now from \(\mathcal {X(s)}\) and \(\mathcal{Y(s)}\) find \(x(t)\) and
  \(y(t)\) using \(system.impulse\).
\item
  Plot \(x(t)\) and \(y(t)\) in the same graph and analyse them
\end{itemize}

    \begin{Verbatim}[commandchars=\\\{\}]
{\color{incolor}In [{\color{incolor}30}]:} \PY{l+s+sd}{\PYZsq{}\PYZsq{}\PYZsq{}}
         \PY{l+s+sd}{function to solve for Transfer function H(s)}
         \PY{l+s+sd}{Arguments : num\PYZus{}coeff   \PYZhy{} array of coefficients of denominator polynomial}
         \PY{l+s+sd}{            den\PYZus{}coeff   \PYZhy{} array of coefficients of denominator polynomial}
         \PY{l+s+sd}{Returns   : t,h         \PYZhy{} time and response of the system}
         \PY{l+s+sd}{\PYZsq{}\PYZsq{}\PYZsq{}}   
         
         \PY{k}{def} \PY{n+nf}{coupledSysSolver}\PY{p}{(}\PY{n}{num\PYZus{}coeff}\PY{p}{,}\PY{n}{den\PYZus{}coeff}\PY{p}{)}\PY{p}{:}
             \PY{n}{H\PYZus{}num} \PY{o}{=} \PY{n}{poly1d}\PY{p}{(}\PY{n}{num\PYZus{}coeff}\PY{p}{)}
             \PY{n}{H\PYZus{}den} \PY{o}{=} \PY{n}{poly1d}\PY{p}{(}\PY{n}{den\PYZus{}coeff}\PY{p}{)}
             
             \PY{n}{Hs} \PY{o}{=} \PY{n}{sp}\PY{o}{.}\PY{n}{lti}\PY{p}{(}\PY{n}{H\PYZus{}num}\PY{p}{,}\PY{n}{H\PYZus{}den}\PY{p}{)}
             \PY{n}{t}\PY{p}{,}\PY{n}{h}\PY{o}{=}\PY{n}{sp}\PY{o}{.}\PY{n}{impulse}\PY{p}{(}\PY{n}{Hs}\PY{p}{,}\PY{k+kc}{None}\PY{p}{,}\PY{n}{linspace}\PY{p}{(}\PY{l+m+mi}{0}\PY{p}{,}\PY{l+m+mi}{20}\PY{p}{,}\PY{l+m+mi}{1000}\PY{p}{)}\PY{p}{)}
             \PY{k}{return} \PY{n}{t}\PY{p}{,}\PY{n}{h}
\end{Verbatim}


    \begin{Verbatim}[commandchars=\\\{\}]
{\color{incolor}In [{\color{incolor}31}]:} \PY{c+c1}{\PYZsh{}find x and y using above function}
         \PY{n}{t1}\PY{p}{,}\PY{n}{x}  \PY{o}{=} \PY{n}{coupledSysSolver}\PY{p}{(}\PY{p}{[}\PY{l+m+mi}{1}\PY{p}{,}\PY{l+m+mi}{0}\PY{p}{,}\PY{l+m+mi}{2}\PY{p}{]}\PY{p}{,}\PY{p}{[}\PY{l+m+mi}{1}\PY{p}{,}\PY{l+m+mi}{0}\PY{p}{,}\PY{l+m+mi}{3}\PY{p}{,}\PY{l+m+mi}{0}\PY{p}{]}\PY{p}{)}
         \PY{n}{t2}\PY{p}{,}\PY{n}{y} \PY{o}{=} \PY{n}{coupledSysSolver}\PY{p}{(}\PY{p}{[}\PY{l+m+mi}{2}\PY{p}{]}\PY{p}{,}\PY{p}{[}\PY{l+m+mi}{1}\PY{p}{,}\PY{l+m+mi}{0}\PY{p}{,}\PY{l+m+mi}{3}\PY{p}{,}\PY{l+m+mi}{0}\PY{p}{]}\PY{p}{)}
\end{Verbatim}


    \begin{Verbatim}[commandchars=\\\{\}]
{\color{incolor}In [{\color{incolor}32}]:} \PY{c+c1}{\PYZsh{}plot x(t) and y(t)}
         \PY{n}{fig4} \PY{o}{=} \PY{n}{figure}\PY{p}{(}\PY{p}{)}
         \PY{n}{ax4} \PY{o}{=} \PY{n}{fig4}\PY{o}{.}\PY{n}{add\PYZus{}subplot}\PY{p}{(}\PY{l+m+mi}{111}\PY{p}{)}
         \PY{n}{ax4}\PY{o}{.}\PY{n}{plot}\PY{p}{(}\PY{n}{t1}\PY{p}{,}\PY{n}{x}\PY{p}{,}\PY{l+s+s1}{\PYZsq{}}\PY{l+s+s1}{b}\PY{l+s+s1}{\PYZsq{}}\PY{p}{,}\PY{n}{label}\PY{o}{=}\PY{l+s+s2}{\PYZdq{}}\PY{l+s+s2}{\PYZdl{}x(t)\PYZdl{}}\PY{l+s+s2}{\PYZdq{}}\PY{p}{)}
         \PY{n}{ax4}\PY{o}{.}\PY{n}{plot}\PY{p}{(}\PY{n}{t2}\PY{p}{,}\PY{n}{y}\PY{p}{,}\PY{l+s+s1}{\PYZsq{}}\PY{l+s+s1}{r}\PY{l+s+s1}{\PYZsq{}}\PY{p}{,}\PY{n}{label}\PY{o}{=}\PY{l+s+s2}{\PYZdq{}}\PY{l+s+s2}{\PYZdl{}y(t)\PYZdl{}}\PY{l+s+s2}{\PYZdq{}}\PY{p}{)}
         \PY{n}{ax4}\PY{o}{.}\PY{n}{legend}\PY{p}{(}\PY{p}{)}
         \PY{n}{title}\PY{p}{(}\PY{l+s+sa}{r}\PY{l+s+s2}{\PYZdq{}}\PY{l+s+s2}{Figure 4: Time evolution of \PYZdl{}x(t)\PYZdl{} and \PYZdl{}y(t)\PYZdl{} for \PYZdl{}0 }\PY{l+s+s2}{\PYZbs{}}\PY{l+s+s2}{leq t }\PY{l+s+s2}{\PYZbs{}}\PY{l+s+s2}{leq 20\PYZdl{}. of Coupled spring system }\PY{l+s+s2}{\PYZdq{}}\PY{p}{)}
         \PY{n}{xlabel}\PY{p}{(}\PY{l+s+sa}{r}\PY{l+s+s2}{\PYZdq{}}\PY{l+s+s2}{\PYZdl{}t }\PY{l+s+s2}{\PYZbs{}}\PY{l+s+s2}{to \PYZdl{}}\PY{l+s+s2}{\PYZdq{}}\PY{p}{)}
         \PY{n}{ylabel}\PY{p}{(}\PY{l+s+sa}{r}\PY{l+s+s2}{\PYZdq{}}\PY{l+s+s2}{\PYZdl{}x(t),y(t) }\PY{l+s+s2}{\PYZbs{}}\PY{l+s+s2}{to \PYZdl{}}\PY{l+s+s2}{\PYZdq{}}\PY{p}{)}
         \PY{n}{grid}\PY{p}{(}\PY{p}{)}
         \PY{n}{savefig}\PY{p}{(}\PY{l+s+s2}{\PYZdq{}}\PY{l+s+s2}{Figure4.jpg}\PY{l+s+s2}{\PYZdq{}}\PY{p}{)}
\end{Verbatim}


    \begin{center}
    \adjustimage{max size={0.9\linewidth}{0.9\paperheight}}{output_18_0.pdf}
    \end{center}
    { \hspace*{\fill} \\}
    
    \subsubsection{Results and Discussion:}\label{results-and-discussion}

\begin{itemize}
\tightlist
\item
  As we observe figure 4 that, the \(x(t)\) and \(y(t)\) obtained
  satisfies the given initial conditions,and oscillating sinusoidally
  with \(180^{o}\) out of phase
\end{itemize}

    \subsection{Question 5:}\label{question-5}

\begin{itemize}
\item
  To Obtain the magnitude and phase response of the Steady State
  Transfer function of the following two-port network
\item
  Transfer function of the RLC network from input to voltage at
  capacitor in general for given Network is
\end{itemize}

\begin{equation}
    \frac{\mathcal{V_{0}(s)}}{\mathcal{V_{i}(s)}} = \mathcal{H(s)} = \frac{1}{s^{2}LC + sRC + 1}
\end{equation}

\begin{itemize}
\item
  For the given values of \$ R = 100 \Omega\(,\) L = 1\mu H\(,\)C=
  1\mu F\$
\item
  We get
\end{itemize}

\begin{equation}
    \mathcal{H(s)} = \frac{1}{s^{2}10^{-12} + s10^{-4} + 1}
\end{equation}

\begin{itemize}
\tightlist
\item
  can be written as
\end{itemize}

\begin{equation}
    \mathcal{H(s)} = \frac{1}{(1 + \frac{s}{10^{8}})(1 + \frac{s}{10^{4}})}
\end{equation}

\begin{itemize}
\item
  So system has poles on left half s plane, that too real poles with \$
  s = -10\textsuperscript{4,-10}8 rads\^{}\{-1\}\$, we will observe the
  effect of poles on magnitude and phase response by analysing plots of
  them
\item
  Magnitude Response is \(|H(s)|\) evaluated at any point on imaginary
  axis so basically \(|H(j\omega)|\)
\item
  Phase response is \(\angle H(j\omega)\)
\item
  Using \(\mathcal{H(s)}\) we calculate magnitude and phase response of
  it using \(Bode()\) and plot them and analyse.
\end{itemize}

    \begin{Verbatim}[commandchars=\\\{\}]
{\color{incolor}In [{\color{incolor}33}]:} \PY{l+s+sd}{\PYZsq{}\PYZsq{}\PYZsq{}}
         \PY{l+s+sd}{function to solve for Transfer function H(s)}
         \PY{l+s+sd}{Arguments : H         \PYZhy{} Transfer function.}
         \PY{l+s+sd}{Returns   : w,mag,phi}
         \PY{l+s+sd}{\PYZsq{}\PYZsq{}\PYZsq{}}   
         
         \PY{k}{def} \PY{n+nf}{CalcMagPhase}\PY{p}{(}\PY{n}{H}\PY{p}{)}\PY{p}{:}
             \PY{n}{w}\PY{p}{,}\PY{n}{mag}\PY{p}{,}\PY{n}{phi}\PY{o}{=}\PY{n}{H}\PY{o}{.}\PY{n}{bode}\PY{p}{(}\PY{p}{)}
             \PY{k}{return} \PY{n}{w}\PY{p}{,}\PY{n}{mag}\PY{p}{,}\PY{n}{phi}
\end{Verbatim}


    \begin{Verbatim}[commandchars=\\\{\}]
{\color{incolor}In [{\color{incolor}34}]:} \PY{l+s+sd}{\PYZsq{}\PYZsq{}\PYZsq{}}
         \PY{l+s+sd}{function to solve given RLC network for any R,L,C values}
         \PY{l+s+sd}{Returns   : w,mag,phi,Hs}
         \PY{l+s+sd}{\PYZsq{}\PYZsq{}\PYZsq{}}  
         
         \PY{k}{def} \PY{n+nf}{RLCnetwork}\PY{p}{(}\PY{n}{R}\PY{p}{,}\PY{n}{C}\PY{p}{,}\PY{n}{L}\PY{p}{)}\PY{p}{:}
             \PY{n}{Hnum} \PY{o}{=} \PY{n}{poly1d}\PY{p}{(}\PY{p}{[}\PY{l+m+mi}{1}\PY{p}{]}\PY{p}{)}
             \PY{n}{Hden} \PY{o}{=} \PY{n}{poly1d}\PY{p}{(}\PY{p}{[}\PY{n}{L}\PY{o}{*}\PY{n}{C}\PY{p}{,}\PY{n}{R}\PY{o}{*}\PY{n}{C}\PY{p}{,}\PY{l+m+mi}{1}\PY{p}{]}\PY{p}{)}
             
             \PY{c+c1}{\PYZsh{}Computes the impulse response of the transfer function}
             \PY{n}{Hs} \PY{o}{=} \PY{n}{sp}\PY{o}{.}\PY{n}{lti}\PY{p}{(}\PY{n}{Hnum}\PY{p}{,}\PY{n}{Hden}\PY{p}{)}
             \PY{c+c1}{\PYZsh{}Calculates magnitude and phase response}
             \PY{n}{w}\PY{p}{,}\PY{n}{mag}\PY{p}{,}\PY{n}{phi} \PY{o}{=} \PY{n}{CalcMagPhase}\PY{p}{(}\PY{n}{Hs}\PY{p}{)}
             \PY{k}{return} \PY{n}{w}\PY{p}{,}\PY{n}{mag}\PY{p}{,}\PY{n}{phi}\PY{p}{,}\PY{n}{Hs}
\end{Verbatim}


    \begin{Verbatim}[commandchars=\\\{\}]
{\color{incolor}In [{\color{incolor}35}]:} \PY{c+c1}{\PYZsh{}Finds magnitude and phase response of Transfer function}
         \PY{n}{R} \PY{o}{=} \PY{l+m+mi}{100}
         \PY{n}{L} \PY{o}{=} \PY{l+m+mf}{1e\PYZhy{}6}
         \PY{n}{C} \PY{o}{=} \PY{l+m+mf}{1e\PYZhy{}6}
         \PY{n}{w}\PY{p}{,}\PY{n}{mag}\PY{p}{,}\PY{n}{phi}\PY{p}{,}\PY{n}{Hrlc} \PY{o}{=} \PY{n}{RLCnetwork}\PY{p}{(}\PY{n}{R}\PY{p}{,}\PY{n}{L}\PY{p}{,}\PY{n}{C}\PY{p}{)}
         
         \PY{c+c1}{\PYZsh{}plot Magnitude Response }
         \PY{n}{fig5} \PY{o}{=} \PY{n}{figure}\PY{p}{(}\PY{p}{)}
         \PY{n}{ax5} \PY{o}{=} \PY{n}{fig5}\PY{o}{.}\PY{n}{add\PYZus{}subplot}\PY{p}{(}\PY{l+m+mi}{111}\PY{p}{)}
         \PY{n}{ax5}\PY{o}{.}\PY{n}{semilogx}\PY{p}{(}\PY{n}{w}\PY{p}{,}\PY{n}{mag}\PY{p}{,}\PY{l+s+s1}{\PYZsq{}}\PY{l+s+s1}{b}\PY{l+s+s1}{\PYZsq{}}\PY{p}{,}\PY{n}{label}\PY{o}{=}\PY{l+s+s2}{\PYZdq{}}\PY{l+s+s2}{\PYZdl{}Mag Response\PYZdl{}}\PY{l+s+s2}{\PYZdq{}}\PY{p}{)}
         \PY{c+c1}{\PYZsh{} ax5.semilogx(w,phi,\PYZsq{}r\PYZsq{},label=\PYZdq{}\PYZdl{}Phase Response\PYZdl{}\PYZdq{})}
         \PY{n}{ax5}\PY{o}{.}\PY{n}{legend}\PY{p}{(}\PY{p}{)}
         \PY{n}{title}\PY{p}{(}\PY{l+s+sa}{r}\PY{l+s+s2}{\PYZdq{}}\PY{l+s+s2}{Figure 5: Magnitude Response of \PYZdl{}H(jw)\PYZdl{} of Series RLC network}\PY{l+s+s2}{\PYZdq{}}\PY{p}{)}
         \PY{n}{xlabel}\PY{p}{(}\PY{l+s+sa}{r}\PY{l+s+s2}{\PYZdq{}}\PY{l+s+s2}{\PYZdl{} }\PY{l+s+s2}{\PYZbs{}}\PY{l+s+s2}{log w }\PY{l+s+s2}{\PYZbs{}}\PY{l+s+s2}{to \PYZdl{}}\PY{l+s+s2}{\PYZdq{}}\PY{p}{)}
         \PY{n}{ylabel}\PY{p}{(}\PY{l+s+sa}{r}\PY{l+s+s2}{\PYZdq{}}\PY{l+s+s2}{\PYZdl{} 20}\PY{l+s+s2}{\PYZbs{}}\PY{l+s+s2}{log|H(jw)|  }\PY{l+s+s2}{\PYZbs{}}\PY{l+s+s2}{to \PYZdl{}}\PY{l+s+s2}{\PYZdq{}}\PY{p}{)}
         \PY{n}{grid}\PY{p}{(}\PY{p}{)}
         \PY{n}{savefig}\PY{p}{(}\PY{l+s+s2}{\PYZdq{}}\PY{l+s+s2}{Figure5.jpg}\PY{l+s+s2}{\PYZdq{}}\PY{p}{)}
\end{Verbatim}


    \begin{center}
    \adjustimage{max size={0.9\linewidth}{0.9\paperheight}}{output_23_0.pdf}
    \end{center}
    { \hspace*{\fill} \\}
    
    \subsubsection{Results and Discussion:}\label{results-and-discussion}

\begin{itemize}
\tightlist
\item
  As we observe figure 5 that, its a low pass filter with dominant pole
  at \(\omega = 10^4 rads^{-1}\) and since the Gain i.e \(|H(j\omega)|\)
  at smaller frequencies is larger and as the frequency increases the
  gain reduces 20dB/dec after the dominant pole which is located
  \(10^4 \leq \omega \leq 10^8\) after that it reduces more steeply with
  40dB/dec.
\item
  Since its a second order system it has two poles,as we observe from
  the plot that it has 2 poles at \$ \omega =
  -10\textsuperscript{4,-10}8 rads\^{}\{-1\}\$
\end{itemize}

    \begin{Verbatim}[commandchars=\\\{\}]
{\color{incolor}In [{\color{incolor}36}]:} \PY{c+c1}{\PYZsh{}Plot of phase response}
         \PY{n}{fig6} \PY{o}{=} \PY{n}{figure}\PY{p}{(}\PY{p}{)}
         \PY{n}{ax6} \PY{o}{=} \PY{n}{fig6}\PY{o}{.}\PY{n}{add\PYZus{}subplot}\PY{p}{(}\PY{l+m+mi}{111}\PY{p}{)}
         \PY{n}{ax6}\PY{o}{.}\PY{n}{semilogx}\PY{p}{(}\PY{n}{w}\PY{p}{,}\PY{n}{phi}\PY{p}{,}\PY{l+s+s1}{\PYZsq{}}\PY{l+s+s1}{r}\PY{l+s+s1}{\PYZsq{}}\PY{p}{,}\PY{n}{label}\PY{o}{=}\PY{l+s+s2}{\PYZdq{}}\PY{l+s+s2}{\PYZdl{}Phase Response\PYZdl{}}\PY{l+s+s2}{\PYZdq{}}\PY{p}{)}
         \PY{n}{ax6}\PY{o}{.}\PY{n}{legend}\PY{p}{(}\PY{p}{)}
         \PY{n}{title}\PY{p}{(}\PY{l+s+sa}{r}\PY{l+s+s2}{\PYZdq{}}\PY{l+s+s2}{Figure 6: phase response of the \PYZdl{}H(jw)\PYZdl{} of Series RLC networkfor}\PY{l+s+s2}{\PYZdq{}}\PY{p}{)}
         \PY{n}{xlabel}\PY{p}{(}\PY{l+s+sa}{r}\PY{l+s+s2}{\PYZdq{}}\PY{l+s+s2}{\PYZdl{} }\PY{l+s+s2}{\PYZbs{}}\PY{l+s+s2}{log w }\PY{l+s+s2}{\PYZbs{}}\PY{l+s+s2}{to \PYZdl{}}\PY{l+s+s2}{\PYZdq{}}\PY{p}{)}
         \PY{n}{ylabel}\PY{p}{(}\PY{l+s+sa}{r}\PY{l+s+s2}{\PYZdq{}}\PY{l+s+s2}{\PYZdl{} }\PY{l+s+s2}{\PYZbs{}}\PY{l+s+s2}{angle H(j}\PY{l+s+s2}{\PYZbs{}}\PY{l+s+s2}{omega)\PYZdl{} \PYZdl{}}\PY{l+s+s2}{\PYZbs{}}\PY{l+s+s2}{to \PYZdl{}}\PY{l+s+s2}{\PYZdq{}}\PY{p}{)}
         \PY{n}{grid}\PY{p}{(}\PY{p}{)}
         \PY{n}{savefig}\PY{p}{(}\PY{l+s+s2}{\PYZdq{}}\PY{l+s+s2}{Figure6.jpg}\PY{l+s+s2}{\PYZdq{}}\PY{p}{)}
\end{Verbatim}


    \begin{center}
    \adjustimage{max size={0.9\linewidth}{0.9\paperheight}}{output_25_0.pdf}
    \end{center}
    { \hspace*{\fill} \\}
    
    \subsubsection{Results and Discussion:}\label{results-and-discussion}

\begin{itemize}
\tightlist
\item
  As we observe figure 6 that,\(0 \leq \angle H(j\omega) < 180^{o}\).
\item
  So the system is unconditionally stable since phase does not go to
  \(180^{o}\).
\item
  Since each pole adds \(90^{o}\) to the phase its clear that the system
  is second order because it has two poles hence phase go till
  \(180^{o}\).
\item
  And since all the poles are in left half s plane RLC Network given is
  unconditionally stable for given values.
\end{itemize}

    \subsection{Question 6:}\label{question-6}

\begin{itemize}
\tightlist
\item
  Consider the problem in \(Q5\).If the input signal \(v_i(t)\) is given
  by
\end{itemize}

\begin{equation}
 v_i(t) = \cos (10^{3}t) u(t) − \cos (10^{6}t) u(t)
 \end{equation}

\begin{itemize}
\tightlist
\item
  Obtain the output voltage \(v_0(t)\) using the transfer function of
  the system obtained.
\item
  To explain the output signal for \$ 0 \textless{} t \textless{} 30
  \mu s \$
\item
  And explain the long term response on the \(msec\) timescale
\end{itemize}

    \begin{Verbatim}[commandchars=\\\{\}]
{\color{incolor}In [{\color{incolor}37}]:} \PY{l+s+sd}{\PYZsq{}\PYZsq{}\PYZsq{}}
         \PY{l+s+sd}{function to return vi(t)}
         \PY{l+s+sd}{arguments : t   \PYZhy{} time variable}
         \PY{l+s+sd}{            w1  \PYZhy{} frequency of 1st cos term}
         \PY{l+s+sd}{            w2  \PYZhy{} frequency of 2nd cos term}
         \PY{l+s+sd}{Returns   : vi(t)}
         \PY{l+s+sd}{\PYZsq{}\PYZsq{}\PYZsq{}}  
         
         \PY{k}{def} \PY{n+nf}{vi}\PY{p}{(}\PY{n}{t}\PY{p}{,}\PY{n}{w1}\PY{p}{,}\PY{n}{w2}\PY{p}{)}\PY{p}{:}
             \PY{k}{return} \PY{n}{cos}\PY{p}{(}\PY{n}{w1}\PY{o}{*}\PY{n}{t}\PY{p}{)}\PY{o}{\PYZhy{}}\PY{n}{cos}\PY{p}{(}\PY{n}{w2}\PY{o}{*}\PY{n}{t}\PY{p}{)}
\end{Verbatim}


    \begin{Verbatim}[commandchars=\\\{\}]
{\color{incolor}In [{\color{incolor}38}]:} \PY{c+c1}{\PYZsh{}Defines time from 0 to 90 msec}
         \PY{n}{t}  \PY{o}{=} \PY{n}{linspace}\PY{p}{(}\PY{l+m+mi}{0}\PY{p}{,}\PY{l+m+mi}{90}\PY{o}{*}\PY{n+nb}{pow}\PY{p}{(}\PY{l+m+mi}{10}\PY{p}{,}\PY{o}{\PYZhy{}}\PY{l+m+mi}{3}\PY{p}{)}\PY{p}{,}\PY{n+nb}{pow}\PY{p}{(}\PY{l+m+mi}{10}\PY{p}{,}\PY{l+m+mi}{6}\PY{p}{)}\PY{p}{)}
         \PY{c+c1}{\PYZsh{}finding vi(t) using above function}
         \PY{n}{Vi} \PY{o}{=} \PY{n}{vi}\PY{p}{(}\PY{n}{t}\PY{p}{,}\PY{n+nb}{pow}\PY{p}{(}\PY{l+m+mi}{10}\PY{p}{,}\PY{l+m+mi}{3}\PY{p}{)}\PY{p}{,}\PY{n+nb}{pow}\PY{p}{(}\PY{l+m+mi}{10}\PY{p}{,}\PY{l+m+mi}{6}\PY{p}{)}\PY{p}{)}
         
         \PY{c+c1}{\PYZsh{}finds Vo(t) using lsim}
         \PY{n}{t}\PY{p}{,}\PY{n}{Vo}\PY{p}{,}\PY{n}{svec}\PY{o}{=}\PY{n}{sp}\PY{o}{.}\PY{n}{lsim}\PY{p}{(}\PY{n}{Hrlc}\PY{p}{,}\PY{n}{Vi}\PY{p}{,}\PY{n}{t}\PY{p}{)}
         \PY{n}{vo\PYZus{}ideal} \PY{o}{=} \PY{n}{cos}\PY{p}{(}\PY{l+m+mf}{1e3}\PY{o}{*}\PY{n}{t}\PY{p}{)}
\end{Verbatim}


    \begin{Verbatim}[commandchars=\\\{\}]
{\color{incolor}In [{\color{incolor}39}]:} \PY{c+c1}{\PYZsh{}plot of Vo(t) for large time i.e at steady state}
         \PY{c+c1}{\PYZsh{}Long term response}
         \PY{n}{fig7a} \PY{o}{=} \PY{n}{figure}\PY{p}{(}\PY{p}{)}
         \PY{n}{ax7a} \PY{o}{=} \PY{n}{fig7a}\PY{o}{.}\PY{n}{add\PYZus{}subplot}\PY{p}{(}\PY{l+m+mi}{111}\PY{p}{)}
         \PY{n}{ax7a}\PY{o}{.}\PY{n}{plot}\PY{p}{(}\PY{n}{t}\PY{p}{,}\PY{n}{Vo}\PY{p}{,}\PY{l+s+s1}{\PYZsq{}}\PY{l+s+s1}{r}\PY{l+s+s1}{\PYZsq{}}\PY{p}{,}\PY{n}{label}\PY{o}{=}\PY{l+s+s2}{\PYZdq{}}\PY{l+s+s2}{Output Voltage \PYZdl{}v\PYZus{}0(t)\PYZdl{} for large time}\PY{l+s+s2}{\PYZdq{}}\PY{p}{)}
         \PY{n}{ax7a}\PY{o}{.}\PY{n}{legend}\PY{p}{(}\PY{p}{)}
         \PY{n}{title}\PY{p}{(}\PY{l+s+sa}{r}\PY{l+s+s2}{\PYZdq{}}\PY{l+s+s2}{Figure 7a: Output Voltage \PYZdl{}v\PYZus{}0(t)\PYZdl{}  of series RLC network for given \PYZdl{}v\PYZus{}i(t)\PYZdl{} at Steady State}\PY{l+s+s2}{\PYZdq{}}\PY{p}{)}
         \PY{n}{xlabel}\PY{p}{(}\PY{l+s+sa}{r}\PY{l+s+s2}{\PYZdq{}}\PY{l+s+s2}{\PYZdl{} t }\PY{l+s+s2}{\PYZbs{}}\PY{l+s+s2}{to \PYZdl{}}\PY{l+s+s2}{\PYZdq{}}\PY{p}{)}
         \PY{n}{ylabel}\PY{p}{(}\PY{l+s+sa}{r}\PY{l+s+s2}{\PYZdq{}}\PY{l+s+s2}{\PYZdl{} y(t) }\PY{l+s+s2}{\PYZbs{}}\PY{l+s+s2}{to \PYZdl{}}\PY{l+s+s2}{\PYZdq{}}\PY{p}{)}
         \PY{n}{grid}\PY{p}{(}\PY{p}{)}
         \PY{n}{savefig}\PY{p}{(}\PY{l+s+s2}{\PYZdq{}}\PY{l+s+s2}{Figure7a.jpg}\PY{l+s+s2}{\PYZdq{}}\PY{p}{)}
\end{Verbatim}


    \begin{center}
    \adjustimage{max size={0.9\linewidth}{0.9\paperheight}}{output_30_0.pdf}
    \end{center}
    { \hspace*{\fill} \\}
    
    \subsubsection{Results and Discussion:}\label{results-and-discussion}

\begin{itemize}
\tightlist
\item
  As we observe the plot and the circuit that we know it is a Low pass
  filter with bandwidth \(0< \omega < 10^4\).
\item
  So when the circuit will only pass input with frequencies which are in
  range of bandwidth only. But since its not a ideal low pass filter as
  its gain doesn't drop abruptly at \(10^4\) rather gradual decrease
  which is observed from magnitude response plot.
\item
  So the output \(V_o(t)\) will be mainly of \(\cos(10^{3}t)\) with
  higher frequencies riding over it in long term response i.e Steady
  state solution.
\item
  This behaviour is observed in the plot that,the
  \(v_o(t) \approx \cos(10^{3}t)\) with higher frequencies riding over
  it for large time.
\item
  The curve is very flickery or not smooth because of the high frequency
  components only.
\item
  In the next plot we'll zoom large enough to observe those components.
\end{itemize}

    \begin{Verbatim}[commandchars=\\\{\}]
{\color{incolor}In [{\color{incolor}40}]:} \PY{c+c1}{\PYZsh{}plot of Vo(t) for large time i.e at steady state}
         \PY{c+c1}{\PYZsh{}Long term response}
         \PY{n}{fig7} \PY{o}{=} \PY{n}{figure}\PY{p}{(}\PY{p}{)}
         \PY{n}{ax7} \PY{o}{=} \PY{n}{fig7}\PY{o}{.}\PY{n}{add\PYZus{}subplot}\PY{p}{(}\PY{l+m+mi}{111}\PY{p}{)}
         \PY{n}{ax7}\PY{o}{.}\PY{n}{plot}\PY{p}{(}\PY{n}{t}\PY{p}{,}\PY{n}{Vo}\PY{p}{,}\PY{l+s+s1}{\PYZsq{}}\PY{l+s+s1}{r}\PY{l+s+s1}{\PYZsq{}}\PY{p}{,}\PY{n}{label}\PY{o}{=}\PY{l+s+s2}{\PYZdq{}}\PY{l+s+s2}{Output Voltage \PYZdl{}v\PYZus{}0(t)\PYZdl{} \PYZhy{} zoomed in }\PY{l+s+s2}{\PYZdq{}}\PY{p}{)}
         \PY{n}{ax7}\PY{o}{.}\PY{n}{plot}\PY{p}{(}\PY{n}{t}\PY{p}{,}\PY{n}{vo\PYZus{}ideal}\PY{p}{,}\PY{l+s+s1}{\PYZsq{}}\PY{l+s+s1}{g}\PY{l+s+s1}{\PYZsq{}}\PY{p}{,}\PY{n}{label}\PY{o}{=}\PY{l+s+s2}{\PYZdq{}}\PY{l+s+s2}{Ideal Low Pass filter Output with cutoff at \PYZdl{}10\PYZca{}4\PYZdl{}}\PY{l+s+s2}{\PYZdq{}}\PY{p}{)}
         \PY{n}{xlim}\PY{p}{(}\PY{l+m+mf}{0.0505}\PY{p}{,}\PY{l+m+mf}{0.051}\PY{p}{)}
         \PY{n}{ylim}\PY{p}{(}\PY{l+m+mf}{0.75}\PY{p}{,}\PY{l+m+mf}{1.1}\PY{p}{)}
         \PY{n}{ax7}\PY{o}{.}\PY{n}{legend}\PY{p}{(}\PY{p}{)}
         \PY{n}{title}\PY{p}{(}\PY{l+s+sa}{r}\PY{l+s+s2}{\PYZdq{}}\PY{l+s+s2}{Figure 7b: Output Voltage \PYZdl{}v\PYZus{}0(t)\PYZdl{}  Vs Ideal Low pass filter Output}\PY{l+s+s2}{\PYZdq{}}\PY{p}{)}
         \PY{n}{xlabel}\PY{p}{(}\PY{l+s+sa}{r}\PY{l+s+s2}{\PYZdq{}}\PY{l+s+s2}{\PYZdl{} t }\PY{l+s+s2}{\PYZbs{}}\PY{l+s+s2}{to \PYZdl{}}\PY{l+s+s2}{\PYZdq{}}\PY{p}{)}
         \PY{n}{ylabel}\PY{p}{(}\PY{l+s+sa}{r}\PY{l+s+s2}{\PYZdq{}}\PY{l+s+s2}{\PYZdl{} y(t) }\PY{l+s+s2}{\PYZbs{}}\PY{l+s+s2}{to \PYZdl{}}\PY{l+s+s2}{\PYZdq{}}\PY{p}{)}
         \PY{n}{grid}\PY{p}{(}\PY{p}{)}
         \PY{n}{savefig}\PY{p}{(}\PY{l+s+s2}{\PYZdq{}}\PY{l+s+s2}{Figure7b.jpg}\PY{l+s+s2}{\PYZdq{}}\PY{p}{)}
\end{Verbatim}


    \begin{center}
    \adjustimage{max size={0.9\linewidth}{0.9\paperheight}}{output_32_0.pdf}
    \end{center}
    { \hspace*{\fill} \\}
    
    \subsubsection{Results and Discussion:}\label{results-and-discussion}

\begin{itemize}
\tightlist
\item
  The oscillatory behaviour in the graph is because of high frequency
  components riding over the main signal which is \(\cos(10^{3}t)\)
  since gain of the system \(\approx 1\) for lower frequencies than
  \(\omega < 10^4\) and gradually decreases as 20dB/dec which is
  observed from magnitude response plot and for very high frequencies
  40dB/dec.
\item
  So there will be some higher frequencies but the gain will be very
  less since its low pass filter ,that's why we can see there are very
  small sinusoidal oscillations over the main output signal compared to
  ideal low pass filter output which is \(\cos(10^{3}t)\)
\end{itemize}

    \begin{Verbatim}[commandchars=\\\{\}]
{\color{incolor}In [{\color{incolor}41}]:} \PY{c+c1}{\PYZsh{}Plot of Vo(t) for 0\PYZlt{}t\PYZlt{}30usec}
         \PY{n}{fig8} \PY{o}{=} \PY{n}{figure}\PY{p}{(}\PY{p}{)}
         \PY{n}{ax8} \PY{o}{=} \PY{n}{fig8}\PY{o}{.}\PY{n}{add\PYZus{}subplot}\PY{p}{(}\PY{l+m+mi}{111}\PY{p}{)}
         \PY{n}{ax8}\PY{o}{.}\PY{n}{plot}\PY{p}{(}\PY{n}{t}\PY{p}{,}\PY{n}{Vo}\PY{p}{,}\PY{l+s+s1}{\PYZsq{}}\PY{l+s+s1}{r}\PY{l+s+s1}{\PYZsq{}}\PY{p}{,}\PY{n}{label}\PY{o}{=}\PY{l+s+s2}{\PYZdq{}}\PY{l+s+s2}{Output Voltage \PYZdl{}v\PYZus{}0(t)\PYZdl{} : \PYZdl{}0\PYZlt{}t\PYZlt{}30}\PY{l+s+s2}{\PYZbs{}}\PY{l+s+s2}{mu sec\PYZdl{}}\PY{l+s+s2}{\PYZdq{}}\PY{p}{)}
         \PY{n}{ax8}\PY{o}{.}\PY{n}{legend}\PY{p}{(}\PY{p}{)}
         \PY{n}{title}\PY{p}{(}\PY{l+s+sa}{r}\PY{l+s+s2}{\PYZdq{}}\PY{l+s+s2}{Figure 8: Output Voltage \PYZdl{}v\PYZus{}0(t)\PYZdl{} for \PYZdl{}0\PYZlt{}t\PYZlt{}30}\PY{l+s+s2}{\PYZbs{}}\PY{l+s+s2}{mu sec\PYZdl{}}\PY{l+s+s2}{\PYZdq{}}\PY{p}{)}
         \PY{n}{xlim}\PY{p}{(}\PY{l+m+mi}{0}\PY{p}{,}\PY{l+m+mf}{3e\PYZhy{}5}\PY{p}{)}
         \PY{n}{ylim}\PY{p}{(}\PY{o}{\PYZhy{}}\PY{l+m+mf}{1e\PYZhy{}5}\PY{p}{,}\PY{l+m+mf}{0.3}\PY{p}{)}
         \PY{n}{xlabel}\PY{p}{(}\PY{l+s+sa}{r}\PY{l+s+s2}{\PYZdq{}}\PY{l+s+s2}{\PYZdl{} t }\PY{l+s+s2}{\PYZbs{}}\PY{l+s+s2}{to \PYZdl{}}\PY{l+s+s2}{\PYZdq{}}\PY{p}{)}
         \PY{n}{ylabel}\PY{p}{(}\PY{l+s+sa}{r}\PY{l+s+s2}{\PYZdq{}}\PY{l+s+s2}{\PYZdl{} v\PYZus{}0(t) }\PY{l+s+s2}{\PYZbs{}}\PY{l+s+s2}{to \PYZdl{}}\PY{l+s+s2}{\PYZdq{}}\PY{p}{)}
         \PY{n}{grid}\PY{p}{(}\PY{p}{)}
         \PY{n}{savefig}\PY{p}{(}\PY{l+s+s2}{\PYZdq{}}\PY{l+s+s2}{Figure8.jpg}\PY{l+s+s2}{\PYZdq{}}\PY{p}{)}
\end{Verbatim}


    \begin{center}
    \adjustimage{max size={0.9\linewidth}{0.9\paperheight}}{output_34_0.pdf}
    \end{center}
    { \hspace*{\fill} \\}
    
    \subsubsection{Results and Discussion:}\label{results-and-discussion}

\begin{itemize}
\tightlist
\item
  As we observe the plot of Figure 8 , for \(v_0(t)\) from
  \(0<t<30\mu sec\) increases very fast from 0 to 0.3 in just
  \(30\mu s\) because of transients that is we apply sinusoidal step
  input to the system i.e the input is zero for \(t<0\), so when
  abruptly the input becomes non zero for \(t \geq 0\), the system
  output jumps or raises suddenly from 0 to non-zero values in less
  time.
\item
  That's why we observe a sharp rise in output at the start.
\item
  And the oscillatory behaviour in the graph is because of high
  frequency components riding over the main signal which is
  \(\cos(10^{3}t)\) since gain of the system \(\approx 1\) for lower
  frequencies than \(\omega < 10^4\).
\end{itemize}

    \subsection{Conclusion:}\label{conclusion}

\begin{itemize}
\tightlist
\item
  So to conclude we analysed a way to find the solution of continuous
  time LTI systems using laplace transform with help of python signals
  toolbox and got familiarised with solving of differential equations by
  taking laplace transform instead of doing arduous time domain
  analysis.
\end{itemize}


    % Add a bibliography block to the postdoc
    
    
    
    \end{document}
