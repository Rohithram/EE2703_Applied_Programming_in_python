
% Default to the notebook output style

    


% Inherit from the specified cell style.




    
\documentclass[a4paper]{article}

    
    \usepackage[T1]{fontenc}
    % Nicer default font (+ math font) than Computer Modern for most use cases
    \usepackage{mathpazo}
    
    \usepackage{float}
	\restylefloat{table}
		
	\usepackage{titlesec}
	\titleformat{\chapter}[display]{\normalfont\huge\bfseries}{Report}{12pt}{}

    % Basic figure setup, for now with no caption control since it's done
    % automatically by Pandoc (which extracts ![](path) syntax from Markdown).
    \usepackage{graphicx}
    % We will generate all images so they have a width \maxwidth. This means
    % that they will get their normal width if they fit onto the page, but
    % are scaled down if they would overflow the margins.
    \makeatletter
    \def\maxwidth{\ifdim\Gin@nat@width>\linewidth\linewidth
    \else\Gin@nat@width\fi}
    \makeatother
    \let\Oldincludegraphics\includegraphics
    % Set max figure width to be 80% of text width, for now hardcoded.
    \renewcommand{\includegraphics}[1]{\Oldincludegraphics[width=.8\maxwidth]{#1}}
    % Ensure that by default, figures have no caption (until we provide a
    % proper Figure object with a Caption API and a way to capture that
    % in the conversion process - todo).
    \usepackage{caption}
    \DeclareCaptionLabelFormat{nolabel}{}
    \captionsetup{labelformat=nolabel}

    \usepackage{adjustbox} % Used to constrain images to a maximum size 
    \usepackage{xcolor} % Allow colors to be defined
    \usepackage{enumerate} % Needed for markdown enumerations to work
    \usepackage{geometry} % Used to adjust the document margins
    \usepackage{amsmath} % Equations
    \usepackage{amssymb} % Equations
    \usepackage{textcomp} % defines textquotesingle
    % Hack from http://tex.stackexchange.com/a/47451/13684:
    \AtBeginDocument{%
        \def\PYZsq{\textquotesingle}% Upright quotes in Pygmentized code
    }
    \usepackage{upquote} % Upright quotes for verbatim code
    \usepackage{eurosym} % defines \euro
    \usepackage[mathletters]{ucs} % Extended unicode (utf-8) support
    \usepackage[utf8x]{inputenc} % Allow utf-8 characters in the tex document
    \usepackage{fancyvrb} % verbatim replacement that allows latex
    \usepackage{grffile} % extends the file name processing of package graphics 
                         % to support a larger range 
    % The hyperref package gives us a pdf with properly built
    % internal navigation ('pdf bookmarks' for the table of contents,
    % internal cross-reference links, web links for URLs, etc.)
    \usepackage{hyperref}
    \usepackage{longtable} % longtable support required by pandoc >1.10
    \usepackage{booktabs}  % table support for pandoc > 1.12.2
    \usepackage[inline]{enumitem} % IRkernel/repr support (it uses the enumerate* environment)
    \usepackage[normalem]{ulem} % ulem is needed to support strikethroughs (\sout)
                                % normalem makes italics be italics, not underlines
    

    
    
    % Colors for the hyperref package
    \definecolor{urlcolor}{rgb}{0,.145,.698}
    \definecolor{linkcolor}{rgb}{.71,0.21,0.01}
    \definecolor{citecolor}{rgb}{.12,.54,.11}

    % ANSI colors
    \definecolor{ansi-black}{HTML}{3E424D}
    \definecolor{ansi-black-intense}{HTML}{282C36}
    \definecolor{ansi-red}{HTML}{E75C58}
    \definecolor{ansi-red-intense}{HTML}{B22B31}
    \definecolor{ansi-green}{HTML}{00A250}
    \definecolor{ansi-green-intense}{HTML}{007427}
    \definecolor{ansi-yellow}{HTML}{DDB62B}
    \definecolor{ansi-yellow-intense}{HTML}{B27D12}
    \definecolor{ansi-blue}{HTML}{208FFB}
    \definecolor{ansi-blue-intense}{HTML}{0065CA}
    \definecolor{ansi-magenta}{HTML}{D160C4}
    \definecolor{ansi-magenta-intense}{HTML}{A03196}
    \definecolor{ansi-cyan}{HTML}{60C6C8}
    \definecolor{ansi-cyan-intense}{HTML}{258F8F}
    \definecolor{ansi-white}{HTML}{C5C1B4}
    \definecolor{ansi-white-intense}{HTML}{A1A6B2}

    % commands and environments needed by pandoc snippets
    % extracted from the output of `pandoc -s`
    \providecommand{\tightlist}{%
      \setlength{\itemsep}{0pt}\setlength{\parskip}{0pt}}
    \DefineVerbatimEnvironment{Highlighting}{Verbatim}{commandchars=\\\{\}}
    % Add ',fontsize=\small' for more characters per line
    \newenvironment{Shaded}{}{}
    \newcommand{\KeywordTok}[1]{\textcolor[rgb]{0.00,0.44,0.13}{\textbf{{#1}}}}
    \newcommand{\DataTypeTok}[1]{\textcolor[rgb]{0.56,0.13,0.00}{{#1}}}
    \newcommand{\DecValTok}[1]{\textcolor[rgb]{0.25,0.63,0.44}{{#1}}}
    \newcommand{\BaseNTok}[1]{\textcolor[rgb]{0.25,0.63,0.44}{{#1}}}
    \newcommand{\FloatTok}[1]{\textcolor[rgb]{0.25,0.63,0.44}{{#1}}}
    \newcommand{\CharTok}[1]{\textcolor[rgb]{0.25,0.44,0.63}{{#1}}}
    \newcommand{\StringTok}[1]{\textcolor[rgb]{0.25,0.44,0.63}{{#1}}}
    \newcommand{\CommentTok}[1]{\textcolor[rgb]{0.38,0.63,0.69}{\textit{{#1}}}}
    \newcommand{\OtherTok}[1]{\textcolor[rgb]{0.00,0.44,0.13}{{#1}}}
    \newcommand{\AlertTok}[1]{\textcolor[rgb]{1.00,0.00,0.00}{\textbf{{#1}}}}
    \newcommand{\FunctionTok}[1]{\textcolor[rgb]{0.02,0.16,0.49}{{#1}}}
    \newcommand{\RegionMarkerTok}[1]{{#1}}
    \newcommand{\ErrorTok}[1]{\textcolor[rgb]{1.00,0.00,0.00}{\textbf{{#1}}}}
    \newcommand{\NormalTok}[1]{{#1}}
    
    % Additional commands for more recent versions of Pandoc
    \newcommand{\ConstantTok}[1]{\textcolor[rgb]{0.53,0.00,0.00}{{#1}}}
    \newcommand{\SpecialCharTok}[1]{\textcolor[rgb]{0.25,0.44,0.63}{{#1}}}
    \newcommand{\VerbatimStringTok}[1]{\textcolor[rgb]{0.25,0.44,0.63}{{#1}}}
    \newcommand{\SpecialStringTok}[1]{\textcolor[rgb]{0.73,0.40,0.53}{{#1}}}
    \newcommand{\ImportTok}[1]{{#1}}
    \newcommand{\DocumentationTok}[1]{\textcolor[rgb]{0.73,0.13,0.13}{\textit{{#1}}}}
    \newcommand{\AnnotationTok}[1]{\textcolor[rgb]{0.38,0.63,0.69}{\textbf{\textit{{#1}}}}}
    \newcommand{\CommentVarTok}[1]{\textcolor[rgb]{0.38,0.63,0.69}{\textbf{\textit{{#1}}}}}
    \newcommand{\VariableTok}[1]{\textcolor[rgb]{0.10,0.09,0.49}{{#1}}}
    \newcommand{\ControlFlowTok}[1]{\textcolor[rgb]{0.00,0.44,0.13}{\textbf{{#1}}}}
    \newcommand{\OperatorTok}[1]{\textcolor[rgb]{0.40,0.40,0.40}{{#1}}}
    \newcommand{\BuiltInTok}[1]{{#1}}
    \newcommand{\ExtensionTok}[1]{{#1}}
    \newcommand{\PreprocessorTok}[1]{\textcolor[rgb]{0.74,0.48,0.00}{{#1}}}
    \newcommand{\AttributeTok}[1]{\textcolor[rgb]{0.49,0.56,0.16}{{#1}}}
    \newcommand{\InformationTok}[1]{\textcolor[rgb]{0.38,0.63,0.69}{\textbf{\textit{{#1}}}}}
    \newcommand{\WarningTok}[1]{\textcolor[rgb]{0.38,0.63,0.69}{\textbf{\textit{{#1}}}}}
    
    
    % Define a nice break command that doesn't care if a line doesn't already
    % exist.
    \def\br{\hspace*{\fill} \\* }
    % Math Jax compatability definitions
    \def\gt{>}
    \def\lt{<}
    % Document parameters
    \title{Vector Operations and Functions in Python \\ Assignment 2}
    \author{Rohithram R, EE16B031 \\ B.Tech Electrical Engineering, IIT Madras}
    \date{\today \\ First created on January 31,2018}	
    
    
    

    % Pygments definitions
    
\makeatletter
\def\PY@reset{\let\PY@it=\relax \let\PY@bf=\relax%
    \let\PY@ul=\relax \let\PY@tc=\relax%
    \let\PY@bc=\relax \let\PY@ff=\relax}
\def\PY@tok#1{\csname PY@tok@#1\endcsname}
\def\PY@toks#1+{\ifx\relax#1\empty\else%
    \PY@tok{#1}\expandafter\PY@toks\fi}
\def\PY@do#1{\PY@bc{\PY@tc{\PY@ul{%
    \PY@it{\PY@bf{\PY@ff{#1}}}}}}}
\def\PY#1#2{\PY@reset\PY@toks#1+\relax+\PY@do{#2}}

\expandafter\def\csname PY@tok@gd\endcsname{\def\PY@tc##1{\textcolor[rgb]{0.63,0.00,0.00}{##1}}}
\expandafter\def\csname PY@tok@gu\endcsname{\let\PY@bf=\textbf\def\PY@tc##1{\textcolor[rgb]{0.50,0.00,0.50}{##1}}}
\expandafter\def\csname PY@tok@gt\endcsname{\def\PY@tc##1{\textcolor[rgb]{0.00,0.27,0.87}{##1}}}
\expandafter\def\csname PY@tok@gs\endcsname{\let\PY@bf=\textbf}
\expandafter\def\csname PY@tok@gr\endcsname{\def\PY@tc##1{\textcolor[rgb]{1.00,0.00,0.00}{##1}}}
\expandafter\def\csname PY@tok@cm\endcsname{\let\PY@it=\textit\def\PY@tc##1{\textcolor[rgb]{0.25,0.50,0.50}{##1}}}
\expandafter\def\csname PY@tok@vg\endcsname{\def\PY@tc##1{\textcolor[rgb]{0.10,0.09,0.49}{##1}}}
\expandafter\def\csname PY@tok@vi\endcsname{\def\PY@tc##1{\textcolor[rgb]{0.10,0.09,0.49}{##1}}}
\expandafter\def\csname PY@tok@vm\endcsname{\def\PY@tc##1{\textcolor[rgb]{0.10,0.09,0.49}{##1}}}
\expandafter\def\csname PY@tok@mh\endcsname{\def\PY@tc##1{\textcolor[rgb]{0.40,0.40,0.40}{##1}}}
\expandafter\def\csname PY@tok@cs\endcsname{\let\PY@it=\textit\def\PY@tc##1{\textcolor[rgb]{0.25,0.50,0.50}{##1}}}
\expandafter\def\csname PY@tok@ge\endcsname{\let\PY@it=\textit}
\expandafter\def\csname PY@tok@vc\endcsname{\def\PY@tc##1{\textcolor[rgb]{0.10,0.09,0.49}{##1}}}
\expandafter\def\csname PY@tok@il\endcsname{\def\PY@tc##1{\textcolor[rgb]{0.40,0.40,0.40}{##1}}}
\expandafter\def\csname PY@tok@go\endcsname{\def\PY@tc##1{\textcolor[rgb]{0.53,0.53,0.53}{##1}}}
\expandafter\def\csname PY@tok@cp\endcsname{\def\PY@tc##1{\textcolor[rgb]{0.74,0.48,0.00}{##1}}}
\expandafter\def\csname PY@tok@gi\endcsname{\def\PY@tc##1{\textcolor[rgb]{0.00,0.63,0.00}{##1}}}
\expandafter\def\csname PY@tok@gh\endcsname{\let\PY@bf=\textbf\def\PY@tc##1{\textcolor[rgb]{0.00,0.00,0.50}{##1}}}
\expandafter\def\csname PY@tok@ni\endcsname{\let\PY@bf=\textbf\def\PY@tc##1{\textcolor[rgb]{0.60,0.60,0.60}{##1}}}
\expandafter\def\csname PY@tok@nl\endcsname{\def\PY@tc##1{\textcolor[rgb]{0.63,0.63,0.00}{##1}}}
\expandafter\def\csname PY@tok@nn\endcsname{\let\PY@bf=\textbf\def\PY@tc##1{\textcolor[rgb]{0.00,0.00,1.00}{##1}}}
\expandafter\def\csname PY@tok@no\endcsname{\def\PY@tc##1{\textcolor[rgb]{0.53,0.00,0.00}{##1}}}
\expandafter\def\csname PY@tok@na\endcsname{\def\PY@tc##1{\textcolor[rgb]{0.49,0.56,0.16}{##1}}}
\expandafter\def\csname PY@tok@nb\endcsname{\def\PY@tc##1{\textcolor[rgb]{0.00,0.50,0.00}{##1}}}
\expandafter\def\csname PY@tok@nc\endcsname{\let\PY@bf=\textbf\def\PY@tc##1{\textcolor[rgb]{0.00,0.00,1.00}{##1}}}
\expandafter\def\csname PY@tok@nd\endcsname{\def\PY@tc##1{\textcolor[rgb]{0.67,0.13,1.00}{##1}}}
\expandafter\def\csname PY@tok@ne\endcsname{\let\PY@bf=\textbf\def\PY@tc##1{\textcolor[rgb]{0.82,0.25,0.23}{##1}}}
\expandafter\def\csname PY@tok@nf\endcsname{\def\PY@tc##1{\textcolor[rgb]{0.00,0.00,1.00}{##1}}}
\expandafter\def\csname PY@tok@si\endcsname{\let\PY@bf=\textbf\def\PY@tc##1{\textcolor[rgb]{0.73,0.40,0.53}{##1}}}
\expandafter\def\csname PY@tok@s2\endcsname{\def\PY@tc##1{\textcolor[rgb]{0.73,0.13,0.13}{##1}}}
\expandafter\def\csname PY@tok@nt\endcsname{\let\PY@bf=\textbf\def\PY@tc##1{\textcolor[rgb]{0.00,0.50,0.00}{##1}}}
\expandafter\def\csname PY@tok@nv\endcsname{\def\PY@tc##1{\textcolor[rgb]{0.10,0.09,0.49}{##1}}}
\expandafter\def\csname PY@tok@s1\endcsname{\def\PY@tc##1{\textcolor[rgb]{0.73,0.13,0.13}{##1}}}
\expandafter\def\csname PY@tok@dl\endcsname{\def\PY@tc##1{\textcolor[rgb]{0.73,0.13,0.13}{##1}}}
\expandafter\def\csname PY@tok@ch\endcsname{\let\PY@it=\textit\def\PY@tc##1{\textcolor[rgb]{0.25,0.50,0.50}{##1}}}
\expandafter\def\csname PY@tok@m\endcsname{\def\PY@tc##1{\textcolor[rgb]{0.40,0.40,0.40}{##1}}}
\expandafter\def\csname PY@tok@gp\endcsname{\let\PY@bf=\textbf\def\PY@tc##1{\textcolor[rgb]{0.00,0.00,0.50}{##1}}}
\expandafter\def\csname PY@tok@sh\endcsname{\def\PY@tc##1{\textcolor[rgb]{0.73,0.13,0.13}{##1}}}
\expandafter\def\csname PY@tok@ow\endcsname{\let\PY@bf=\textbf\def\PY@tc##1{\textcolor[rgb]{0.67,0.13,1.00}{##1}}}
\expandafter\def\csname PY@tok@sx\endcsname{\def\PY@tc##1{\textcolor[rgb]{0.00,0.50,0.00}{##1}}}
\expandafter\def\csname PY@tok@bp\endcsname{\def\PY@tc##1{\textcolor[rgb]{0.00,0.50,0.00}{##1}}}
\expandafter\def\csname PY@tok@c1\endcsname{\let\PY@it=\textit\def\PY@tc##1{\textcolor[rgb]{0.25,0.50,0.50}{##1}}}
\expandafter\def\csname PY@tok@fm\endcsname{\def\PY@tc##1{\textcolor[rgb]{0.00,0.00,1.00}{##1}}}
\expandafter\def\csname PY@tok@o\endcsname{\def\PY@tc##1{\textcolor[rgb]{0.40,0.40,0.40}{##1}}}
\expandafter\def\csname PY@tok@kc\endcsname{\let\PY@bf=\textbf\def\PY@tc##1{\textcolor[rgb]{0.00,0.50,0.00}{##1}}}
\expandafter\def\csname PY@tok@c\endcsname{\let\PY@it=\textit\def\PY@tc##1{\textcolor[rgb]{0.25,0.50,0.50}{##1}}}
\expandafter\def\csname PY@tok@mf\endcsname{\def\PY@tc##1{\textcolor[rgb]{0.40,0.40,0.40}{##1}}}
\expandafter\def\csname PY@tok@err\endcsname{\def\PY@bc##1{\setlength{\fboxsep}{0pt}\fcolorbox[rgb]{1.00,0.00,0.00}{1,1,1}{\strut ##1}}}
\expandafter\def\csname PY@tok@mb\endcsname{\def\PY@tc##1{\textcolor[rgb]{0.40,0.40,0.40}{##1}}}
\expandafter\def\csname PY@tok@ss\endcsname{\def\PY@tc##1{\textcolor[rgb]{0.10,0.09,0.49}{##1}}}
\expandafter\def\csname PY@tok@sr\endcsname{\def\PY@tc##1{\textcolor[rgb]{0.73,0.40,0.53}{##1}}}
\expandafter\def\csname PY@tok@mo\endcsname{\def\PY@tc##1{\textcolor[rgb]{0.40,0.40,0.40}{##1}}}
\expandafter\def\csname PY@tok@kd\endcsname{\let\PY@bf=\textbf\def\PY@tc##1{\textcolor[rgb]{0.00,0.50,0.00}{##1}}}
\expandafter\def\csname PY@tok@mi\endcsname{\def\PY@tc##1{\textcolor[rgb]{0.40,0.40,0.40}{##1}}}
\expandafter\def\csname PY@tok@kn\endcsname{\let\PY@bf=\textbf\def\PY@tc##1{\textcolor[rgb]{0.00,0.50,0.00}{##1}}}
\expandafter\def\csname PY@tok@cpf\endcsname{\let\PY@it=\textit\def\PY@tc##1{\textcolor[rgb]{0.25,0.50,0.50}{##1}}}
\expandafter\def\csname PY@tok@kr\endcsname{\let\PY@bf=\textbf\def\PY@tc##1{\textcolor[rgb]{0.00,0.50,0.00}{##1}}}
\expandafter\def\csname PY@tok@s\endcsname{\def\PY@tc##1{\textcolor[rgb]{0.73,0.13,0.13}{##1}}}
\expandafter\def\csname PY@tok@kp\endcsname{\def\PY@tc##1{\textcolor[rgb]{0.00,0.50,0.00}{##1}}}
\expandafter\def\csname PY@tok@w\endcsname{\def\PY@tc##1{\textcolor[rgb]{0.73,0.73,0.73}{##1}}}
\expandafter\def\csname PY@tok@kt\endcsname{\def\PY@tc##1{\textcolor[rgb]{0.69,0.00,0.25}{##1}}}
\expandafter\def\csname PY@tok@sc\endcsname{\def\PY@tc##1{\textcolor[rgb]{0.73,0.13,0.13}{##1}}}
\expandafter\def\csname PY@tok@sb\endcsname{\def\PY@tc##1{\textcolor[rgb]{0.73,0.13,0.13}{##1}}}
\expandafter\def\csname PY@tok@sa\endcsname{\def\PY@tc##1{\textcolor[rgb]{0.73,0.13,0.13}{##1}}}
\expandafter\def\csname PY@tok@k\endcsname{\let\PY@bf=\textbf\def\PY@tc##1{\textcolor[rgb]{0.00,0.50,0.00}{##1}}}
\expandafter\def\csname PY@tok@se\endcsname{\let\PY@bf=\textbf\def\PY@tc##1{\textcolor[rgb]{0.73,0.40,0.13}{##1}}}
\expandafter\def\csname PY@tok@sd\endcsname{\let\PY@it=\textit\def\PY@tc##1{\textcolor[rgb]{0.73,0.13,0.13}{##1}}}

\def\PYZbs{\char`\\}
\def\PYZus{\char`\_}
\def\PYZob{\char`\{}
\def\PYZcb{\char`\}}
\def\PYZca{\char`\^}
\def\PYZam{\char`\&}
\def\PYZlt{\char`\<}
\def\PYZgt{\char`\>}
\def\PYZsh{\char`\#}
\def\PYZpc{\char`\%}
\def\PYZdl{\char`\$}
\def\PYZhy{\char`\-}
\def\PYZsq{\char`\'}
\def\PYZdq{\char`\"}
\def\PYZti{\char`\~}
% for compatibility with earlier versions
\def\PYZat{@}
\def\PYZlb{[}
\def\PYZrb{]}
\makeatother


    % Exact colors from NB
    \definecolor{incolor}{rgb}{0.0, 0.0, 0.5}
    \definecolor{outcolor}{rgb}{0.545, 0.0, 0.0}



    
    % Prevent overflowing lines due to hard-to-break entities
    \sloppy 
    % Setup hyperref package
    \hypersetup{
      breaklinks=true,  % so long urls are correctly broken across lines
      colorlinks=true,
      urlcolor=urlcolor,
      linkcolor=linkcolor,
      citecolor=citecolor,
      }
    % Slightly bigger margins than the latex defaults
    
    \geometry{verbose,tmargin=1in,bmargin=1in,lmargin=1in,rmargin=1in}
    
    

    \begin{document}
    
    
    \maketitle
    %\chapter{}
    
	\begin{abstract}    
	\end{abstract}
	This report presents a study of different methods of creating a  $tan^{-1}(x)$ from its integral definition of $\int_{0}^{x} dx/(1+t^{2})$ using scipy's quad function to integrate and other method is of numerical integration using trapezoidal rule,which can be used even for non-integrable functions to integrate it.
	And it also discusses the advantage of Vectorization of code compared to for loops and also on finding estimate errors when actual error is unknown i.e function is non-integrable by halving the stepsize till it reaches certain tolerance!.	
	\section{Introduction}
	This report discusses 5 tasks in python to create  $tan^{-1}(x)$ from its integral definition in stepwise manner.

	\begin{itemize}
	\item Define f(t) = $1/(1+t^{2})$ as a function in Python that takes a
	vector argument
	\item Define a vector x that covers the region 0 ≤ x ≤ 5 in steps of 0.1 
	 \item Plot f(x) vs x using the Python function and the already defined vector x. 
	\item Integrate f(x) using quad function and compare it with arctan(X)
	\item  Use one of the Numerical methods i.e Trapezoidal rule to integrate f(x) and to find estimate error and compare it with actual error  

	\end{itemize}
	
    \section{Python code}
    \subsection{Code to create a tan inverse function from its integral definition}
    \subsubsection{Integral definition of  $tan^{-1}(x)$ and plotting it}
    
    \begin{Verbatim}[commandchars=\\\{\},fontsize=\small]
{\color{incolor}In [{\color{incolor}34}]:} \PY{c+c1}{\PYZsh{}Importing libraries needed}
         \PY{k+kn}{from} \PY{n+nn}{pylab} \PY{k+kn}{import} \PY{o}{*}
         \PY{k+kn}{from} \PY{n+nn}{scipy.integrate} \PY{k+kn}{import} \PY{n}{quad}
         \PY{k+kn}{from} \PY{n+nn}{math} \PY{k+kn}{import} \PY{n}{pi}
         \PY{k+kn}{from}  \PY{n+nn}{tabulate} \PY{k+kn}{import} \PY{n}{tabulate}
         \PY{c+c1}{\PYZsh{}Function which takes vector x as argument used in calculation of tan inverse(x)}
         	
         \PY{k}{def} \PY{n+nf}{f}\PY{p}{(}\PY{n}{x}\PY{p}{)}\PY{p}{:}
             \PY{k}{return} \PY{l+m+mf}{1.0}\PY{o}{/}\PY{p}{(}\PY{l+m+mi}{1}\PY{o}{+}\PY{n}{np}\PY{o}{.}\PY{n}{square}\PY{p}{(}\PY{n}{x}\PY{p}{)}\PY{p}{)}
         
         \PY{c+c1}{\PYZsh{}end of function}
         
         \PY{c+c1}{\PYZsh{}Function to integrate f(x) from 0 to x[i](upper limit) using quad function for}
         \PY{c+c1}{\PYZsh{}all elements in vector x, resulting answer is tan inverse(x[i]) using for loop}
         \PY{c+c1}{\PYZsh{}method}
         
         \PY{k}{def} \PY{n+nf}{tan\PYZus{}inv}\PY{p}{(}\PY{n}{x}\PY{p}{)}\PY{p}{:}  
             
             \PY{n}{ans} \PY{o}{=} \PY{n}{np}\PY{o}{.}\PY{n}{zeros}\PY{p}{(}\PY{n+nb}{len}\PY{p}{(}\PY{n}{x}\PY{p}{)}\PY{p}{)}            \PY{c+c1}{\PYZsh{}initialising vector answer and error with zeros}
             \PY{n}{err} \PY{o}{=} \PY{n}{np}\PY{o}{.}\PY{n}{zeros}\PY{p}{(}\PY{n+nb}{len}\PY{p}{(}\PY{n}{x}\PY{p}{)}\PY{p}{)}            \PY{c+c1}{\PYZsh{}with length that of input vector x}
             
             \PY{k}{for} \PY{n}{i} \PY{o+ow}{in} \PY{n+nb}{range}\PY{p}{(}\PY{n+nb}{len}\PY{p}{(}\PY{n}{x}\PY{p}{)}\PY{p}{)}\PY{p}{:}           \PY{c+c1}{\PYZsh{}loop to calculate integral for all values of x}
                 \PY{n}{ans}\PY{p}{[}\PY{n}{i}\PY{p}{]}\PY{p}{,}\PY{n}{err}\PY{p}{[}\PY{n}{i}\PY{p}{]} \PY{o}{=} \PY{n}{quad}\PY{p}{(}\PY{n}{f}\PY{p}{,}\PY{l+m+mi}{0}\PY{p}{,}\PY{n}{x}\PY{p}{[}\PY{n}{i}\PY{p}{]}\PY{p}{)}
             \PY{k}{return} \PY{n}{ans}\PY{p}{,}\PY{n}{err}
             
         \PY{c+c1}{\PYZsh{}end of function tan\PYZus{}inv}
         
         \PY{c+c1}{\PYZsh{}declaring vector x}
         \PY{n}{x} \PY{o}{=} \PY{n}{arange}\PY{p}{(}\PY{l+m+mi}{0}\PY{p}{,}\PY{l+m+mi}{5}\PY{p}{,}\PY{l+m+mf}{0.1}\PY{p}{)}
         \PY{n}{y} \PY{o}{=} \PY{n}{f}\PY{p}{(}\PY{n}{x}\PY{p}{)}                        \PY{c+c1}{\PYZsh{} y is another vecotr which stores vector returned by f(x)}
         
         \PY{c+c1}{\PYZsh{}plotting f(x) vs x}
         \PY{n}{fig1} \PY{o}{=} \PY{n}{figure}\PY{p}{(}\PY{p}{)}
         \PY{n}{plot}\PY{p}{(}\PY{n}{x}\PY{p}{,}\PY{n}{y}\PY{p}{)}
         \PY{n}{fig1}\PY{o}{.}\PY{n}{suptitle}\PY{p}{(}\PY{l+s+sa}{r}\PY{l+s+s2}{\PYZdq{}}\PY{l+s+s2}{Plot of \PYZdl{}1/(1+t\PYZca{}\PYZob{}2\PYZcb{})\PYZdl{}}\PY{l+s+s2}{\PYZdq{}}\PY{p}{,} \PY{n}{fontsize}\PY{o}{=}\PY{l+m+mi}{20}\PY{p}{)}
         \PY{n}{xlabel}\PY{p}{(}\PY{l+s+s2}{\PYZdq{}}\PY{l+s+s2}{x}\PY{l+s+s2}{\PYZdq{}}\PY{p}{)}
         \PY{n}{fig1}\PY{o}{.}\PY{n}{savefig}\PY{p}{(}\PY{l+s+s1}{\PYZsq{}}\PY{l+s+s1}{1.jpg}\PY{l+s+s1}{\PYZsq{}}\PY{p}{)}
         
         \PY{c+c1}{\PYZsh{}calculating tan inverse of all elements in x by arctan function}
         \PY{n}{tan\PYZus{}inv\PYZus{}exact} \PY{o}{=} \PY{n}{np}\PY{o}{.}\PY{n}{arctan}\PY{p}{(}\PY{n}{x}\PY{p}{)}
         
         \PY{c+c1}{\PYZsh{}plotting tan inverse vs x}
         \PY{n}{fig2} \PY{o}{=} \PY{n}{figure}\PY{p}{(}\PY{p}{)}
         \PY{n}{plot}\PY{p}{(}\PY{n}{x}\PY{p}{,}\PY{n}{tan\PYZus{}inv\PYZus{}exact}\PY{p}{)}
         
         \PY{c+c1}{\PYZsh{}calculating tan\PYZus{}inverse through quad function and storing error associated}
         \PY{n}{I\PYZus{}quad}\PY{p}{,}\PY{n}{err} \PY{o}{=} \PY{n}{tan\PYZus{}inv}\PY{p}{(}\PY{n}{x}\PY{p}{)}
             
         \PY{n}{table} \PY{o}{=} \PY{n+nb}{zip}\PY{p}{(}\PY{n}{tan\PYZus{}inv\PYZus{}exact}\PY{p}{,}\PY{n}{I\PYZus{}quad}\PY{p}{)}
         \PY{n}{headers} \PY{o}{=} \PY{p}{[}\PY{l+s+s2}{\PYZdq{}}\PY{l+s+s2}{arctan(x)}\PY{l+s+s2}{\PYZdq{}}\PY{p}{,}\PY{l+s+s2}{\PYZdq{}}\PY{l+s+s2}{quad\PYZus{}fn:integral}\PY{l+s+s2}{\PYZdq{}}\PY{p}{]}
         \PY{c+c1}{\PYZsh{}tabulating arctan values vs quad function values}
         \PY{k}{print} \PY{n}{tabulate}\PY{p}{(}\PY{n}{table}\PY{p}{,}\PY{n}{tablefmt}\PY{o}{=}\PY{l+s+s2}{\PYZdq{}}\PY{l+s+s2}{fancy\PYZus{}grid}\PY{l+s+s2}{\PYZdq{}}\PY{p}{,}\PY{n}{headers}\PY{o}{=}\PY{n}{headers}\PY{p}{)}              
         
         \PY{c+c1}{\PYZsh{}plotting tan\PYZus{}inverse calculated using quad in same plot of arctan}
         \PY{n}{plot}\PY{p}{(}\PY{n}{x}\PY{p}{,}\PY{n}{I\PYZus{}quad}\PY{p}{,}\PY{l+s+s1}{\PYZsq{}}\PY{l+s+s1}{ro}\PY{l+s+s1}{\PYZsq{}}\PY{p}{)}
         \PY{n}{legend}\PY{p}{(} \PY{p}{(}\PY{l+s+sa}{r}\PY{l+s+s2}{\PYZdq{}}\PY{l+s+s2}{\PYZdl{}tan\PYZca{}\PYZob{}\PYZhy{}1\PYZcb{}x\PYZdl{}}\PY{l+s+s2}{\PYZdq{}}\PY{p}{,}\PY{l+s+s2}{\PYZdq{}}\PY{l+s+s2}{quad fn}\PY{l+s+s2}{\PYZdq{}}\PY{p}{)}\PY{p}{)}
         \PY{n}{fig2}\PY{o}{.}\PY{n}{suptitle}\PY{p}{(}\PY{l+s+sa}{r}\PY{l+s+s2}{\PYZdq{}}\PY{l+s+s2}{Plot of \PYZdl{}tan\PYZca{}\PYZob{}\PYZhy{}1\PYZcb{}x\PYZdl{}}\PY{l+s+s2}{\PYZdq{}}\PY{p}{,} \PY{n}{fontsize}\PY{o}{=}\PY{l+m+mi}{20}\PY{p}{)}
         \PY{n}{xlabel}\PY{p}{(}\PY{l+s+s2}{\PYZdq{}}\PY{l+s+s2}{x}\PY{l+s+s2}{\PYZdq{}}\PY{p}{)}
         \PY{n}{ylabel}\PY{p}{(}\PY{l+s+s2}{\PYZdq{}}\PY{l+s+s2}{\PYZdl{}}\PY{l+s+s2}{\PYZbs{}}\PY{l+s+s2}{int\PYZus{}\PYZob{}0\PYZcb{}\PYZca{}\PYZob{}x\PYZcb{} du/(1+u\PYZca{}\PYZob{}2\PYZcb{})\PYZdl{}}\PY{l+s+s2}{\PYZdq{}}\PY{p}{)}
         \PY{n}{fig2}\PY{o}{.}\PY{n}{savefig}\PY{p}{(}\PY{l+s+s1}{\PYZsq{}}\PY{l+s+s1}{2.jpg}\PY{l+s+s1}{\PYZsq{}}\PY{p}{)}
         
         \PY{c+c1}{\PYZsh{}plotting error associated with quad function while calulating tan\PYZus{}inverse}
         
         \PY{n}{fig3} \PY{o}{=} \PY{n}{figure}\PY{p}{(}\PY{p}{)}
         \PY{n}{semilogy}\PY{p}{(}\PY{n}{x}\PY{p}{,}\PY{n+nb}{abs}\PY{p}{(}\PY{p}{(}\PY{n}{tan\PYZus{}inv\PYZus{}exact}\PY{o}{\PYZhy{}}\PY{n}{I\PYZus{}quad}\PY{p}{)}\PY{p}{)}\PY{p}{,}\PY{l+s+s1}{\PYZsq{}}\PY{l+s+s1}{r.}\PY{l+s+s1}{\PYZsq{}}\PY{p}{)}
         \PY{n}{fig3}\PY{o}{.}\PY{n}{suptitle}\PY{p}{(}\PY{l+s+sa}{r}\PY{l+s+s2}{\PYZdq{}}\PY{l+s+s2}{Error in  \PYZdl{}}\PY{l+s+s2}{\PYZbs{}}\PY{l+s+s2}{int\PYZus{}\PYZob{}0\PYZcb{}\PYZca{}\PYZob{}x\PYZcb{} dx/(1+t\PYZca{}\PYZob{}2\PYZcb{}) \PYZdl{}}\PY{l+s+s2}{\PYZdq{}}\PY{p}{,} \PY{n}{fontsize}\PY{o}{=}\PY{l+m+mi}{12}\PY{p}{)}
         \PY{n}{xlabel}\PY{p}{(}\PY{l+s+s2}{\PYZdq{}}\PY{l+s+s2}{x}\PY{l+s+s2}{\PYZdq{}}\PY{p}{)}
         \PY{n}{ylabel}\PY{p}{(}\PY{l+s+s2}{\PYZdq{}}\PY{l+s+s2}{Error}\PY{l+s+s2}{\PYZdq{}}\PY{p}{)}
         \PY{n}{fig3}\PY{o}{.}\PY{n}{savefig}\PY{p}{(}\PY{l+s+s1}{\PYZsq{}}\PY{l+s+s1}{3.jpg}\PY{l+s+s1}{\PYZsq{}}\PY{p}{)}
         
         \PY{n}{show}\PY{p}{(}\PY{p}{)}
         
             
\end{Verbatim}

\begin{figure}[H]
    \centering
    \caption{Figure 1: Graph of f(x) vs x}
    \adjustimage{max size={0.7\linewidth}{0.7\paperheight}}{output_0_1.png}
	\end{figure}
	
	\begin{figure}[H]
    \centering
    \caption{Figure 2: Comparison of $tan^{-1}(x)$ and quad(f(x)) in same plot}
    \adjustimage{max size={0.7\linewidth}{0.7\paperheight}}{output_0_2.png}
	\end{figure}
	

\begin{table}[H]
\textbf{Tabulating values of arctan(x) vs quad(f(x))}\par\medskip
\begin{tabular}{|c|c|}
\hline
arctan(x) & quad : $\int_{0}^{x} dx/(1+t^{2}) $  \\
\hline
0.00000 & 0.00000 \\
 0.09967 & 0.09967 \\
 0.19740 & 0.19740 \\
 0.29146 & 0.29146 \\
 0.38051 & 0.38051 \\
 0.46365 & 0.46365 \\
 0.54042 & 0.54042 \\
 0.61073 & 0.61073 \\
 0.67474 & 0.67474 \\
 0.73282 & 0.73282 \\
 0.78540 & 0.78540 \\
 0.83298 & 0.83298 \\
 0.87606 & 0.87606 \\
 0.91510 & 0.91510 \\
 0.95055 & 0.95055 \\
 0.98279 & 0.98279 \\
 1.01220 & 1.01220 \\
 1.03907 & 1.03907 \\
 1.06370 & 1.06370 \\
 1.08632 & 1.08632 \\
 1.10715 & 1.10715 \\
 1.12638 & 1.12638 \\
 1.14417 & 1.14417 \\
 1.16067 & 1.16067 \\
 1.17601 & 1.17601 \\
 1.19029 & 1.19029 \\
 1.20362 & 1.20362 \\
 1.21609 & 1.21609 \\
 1.22777 & 1.22777 \\
 1.23874 & 1.23874 \\
 1.24905 & 1.24905 \\
 1.25875 & 1.25875 \\
 1.26791 & 1.26791 \\
 1.27656 & 1.27656 \\
 1.28474 & 1.28474 \\
 1.29250 & 1.29250 \\
 1.29985 & 1.29985 \\
 1.30683 & 1.30683 \\
 1.31347 & 1.31347 \\
 1.31979 & 1.31979 \\
 1.32582 & 1.32582 \\
 1.33156 & 1.33156 \\
 1.33705 & 1.33705 \\
 1.34230 & 1.34230 \\
 1.34732 & 1.34732 \\
 1.35213 & 1.35213 \\
 1.35674 & 1.35674 \\
 1.36116 & 1.36116 \\
 1.36540 & 1.36540 \\
 1.36948 & 1.36948 \\
    \hline
    \end{tabular}
    \end{table}
    
	
	\begin{figure}[H]
    \centering
    \caption{Figure 3: Error associated with quad function compared to arctan(x)}
    \adjustimage{max size={0.7\linewidth}{0.7\paperheight}}{output_0_3.png}
	\end{figure}
    
     \subsection{Integration using Trapezoidal rule}
     \subsubsection{Implementing with for loops without vectorization}
        
    \begin{Verbatim}[commandchars=\\\{\},fontsize=\small]
{\color{incolor}In [{\color{incolor}38}]:} \PY{c+c1}{\PYZsh{}Now we use numerical methods to calculate integral of f(x) using trapezoidal rule}
         \PY{c+c1}{\PYZsh{}instead of quad function with for loops without vectorizing it}
         \PY{k+kn}{import} \PY{n+nn}{time} \PY{k+kn}{as} \PY{n+nn}{t}
         
         \PY{c+c1}{\PYZsh{}I is the vector which stores the integral values of f(x) using trapezoidal rule}
         \PY{n}{I} \PY{o}{=} \PY{p}{[}\PY{p}{]}
         \PY{n}{h}\PY{o}{=}\PY{l+m+mf}{0.1}             \PY{c+c1}{\PYZsh{}h is stepsize }
         \PY{n}{x}\PY{o}{=}\PY{n}{arange}\PY{p}{(}\PY{l+m+mi}{0}\PY{p}{,}\PY{l+m+mi}{5}\PY{p}{,}\PY{n}{h}\PY{p}{)}   \PY{c+c1}{\PYZsh{}x is input vector from 0 to 5 with stepsize 0.1}
         
         \PY{c+c1}{\PYZsh{}Function which takes index of lower limit and upperlimit and stepsize as arguments}
         \PY{c+c1}{\PYZsh{}and calulates using trapezoidal rule}
         
         \PY{k}{def} \PY{n+nf}{trapez}\PY{p}{(}\PY{n}{lower\PYZus{}index}\PY{p}{,}\PY{n}{i}\PY{p}{,}\PY{n}{h}\PY{p}{)}\PY{p}{:}
             \PY{n}{Ii} \PY{o}{=} \PY{n}{h}\PY{o}{*}\PY{p}{(}\PY{p}{(}\PY{n}{cumsumlike}\PY{p}{(}\PY{n}{i}\PY{p}{)}\PY{p}{)}\PY{o}{\PYZhy{}}\PY{l+m+mf}{0.5}\PY{o}{*}\PY{p}{(}\PY{n}{f}\PY{p}{(}\PY{n}{x}\PY{p}{[}\PY{n}{lower\PYZus{}index}\PY{p}{]}\PY{p}{)}\PY{o}{+}\PY{n}{f}\PY{p}{(}\PY{n}{x}\PY{p}{[}\PY{n}{i}\PY{p}{]}\PY{p}{)}\PY{p}{)}\PY{p}{)}
             \PY{k}{return} \PY{n}{Ii}
         
         \PY{c+c1}{\PYZsh{}Its function to calculate cumulative sum till upper limit index i of input vector x}
         \PY{c+c1}{\PYZsh{}this is implemented with for loop}
         \PY{k}{def} \PY{n+nf}{cumsumlike}\PY{p}{(}\PY{n}{i}\PY{p}{)}\PY{p}{:}
             \PY{n}{temp}\PY{o}{=}\PY{l+m+mi}{0}
             \PY{k}{for} \PY{n}{k} \PY{o+ow}{in} \PY{n+nb}{range}\PY{p}{(}\PY{n}{i}\PY{p}{)}\PY{p}{:}
                 \PY{n}{temp}\PY{o}{+}\PY{o}{=}\PY{n}{f}\PY{p}{(}\PY{n}{x}\PY{p}{[}\PY{n}{k}\PY{p}{]}\PY{p}{)}
             \PY{k}{return} \PY{n}{temp}
         
         \PY{c+c1}{\PYZsh{}noting down time it takes to  run}
         \PY{n}{t1} \PY{o}{=} \PY{n}{t}\PY{o}{.}\PY{n}{time}\PY{p}{(}\PY{p}{)}
         \PY{k}{for} \PY{n}{k} \PY{o+ow}{in} \PY{n+nb}{range}\PY{p}{(}\PY{n+nb}{len}\PY{p}{(}\PY{n}{x}\PY{p}{)}\PY{p}{)}\PY{p}{:}
             \PY{n}{I}\PY{o}{.}\PY{n}{append}\PY{p}{(}\PY{n}{trapez}\PY{p}{(}\PY{l+m+mi}{0}\PY{p}{,}\PY{n}{k}\PY{p}{,}\PY{n}{h}\PY{p}{)}\PY{p}{)}          \PY{c+c1}{\PYZsh{}appending the values in vector}
         \PY{n}{t2} \PY{o}{=} \PY{n}{t}\PY{o}{.}\PY{n}{time}\PY{p}{(}\PY{p}{)}
         
         \PY{k}{print} \PY{p}{(}\PY{l+s+s2}{\PYZdq{}}\PY{l+s+s2}{Time took without vectorization : }\PY{l+s+si}{\PYZpc{}g}\PY{l+s+s2}{\PYZdq{}} \PY{o}{\PYZpc{}}\PY{p}{(}\PY{n}{t2}\PY{o}{\PYZhy{}}\PY{n}{t1}\PY{p}{)}\PY{p}{)}
         \PY{c+c1}{\PYZsh{}plotting Integral of x vs x}
         \PY{n}{fig4} \PY{o}{=} \PY{n}{figure}\PY{p}{(}\PY{p}{)}
         \PY{n}{plot}\PY{p}{(}\PY{n}{x}\PY{p}{,}\PY{n}{I}\PY{p}{,}\PY{l+s+s1}{\PYZsq{}}\PY{l+s+s1}{r.}\PY{l+s+s1}{\PYZsq{}}\PY{p}{)}
         \PY{n}{fig4}\PY{o}{.}\PY{n}{suptitle}\PY{p}{(}\PY{l+s+sa}{r}\PY{l+s+s2}{\PYZdq{}}\PY{l+s+s2}{Trapezoid rule : \PYZdl{}}\PY{l+s+s2}{\PYZbs{}}\PY{l+s+s2}{int\PYZus{}\PYZob{}0\PYZcb{}\PYZca{}\PYZob{}x\PYZcb{} dx/(1+t\PYZca{}\PYZob{}2\PYZcb{}) \PYZdl{}}\PY{l+s+s2}{\PYZdq{}}\PY{p}{,}\PY{n}{fontsize}\PY{o}{=}\PY{l+m+mi}{12}\PY{p}{)}
         \PY{n}{xlabel}\PY{p}{(}\PY{l+s+s2}{\PYZdq{}}\PY{l+s+s2}{x}\PY{l+s+s2}{\PYZdq{}}\PY{p}{)}
         \PY{n}{ylabel}\PY{p}{(}\PY{l+s+s2}{\PYZdq{}}\PY{l+s+s2}{\PYZdl{}}\PY{l+s+s2}{\PYZbs{}}\PY{l+s+s2}{int\PYZus{}\PYZob{}0\PYZcb{}\PYZca{}\PYZob{}x\PYZcb{} dx/(1+t\PYZca{}\PYZob{}2\PYZcb{}) \PYZdl{}}\PY{l+s+s2}{\PYZdq{}}\PY{p}{)}
         \PY{n}{fig4}\PY{o}{.}\PY{n}{savefig}\PY{p}{(}\PY{l+s+s1}{\PYZsq{}}\PY{l+s+s1}{4.jpg}\PY{l+s+s1}{\PYZsq{}}\PY{p}{)}
         \PY{n}{show}\PY{p}{(}\PY{p}{)}
         
             
\end{Verbatim}


    \begin{Verbatim}[commandchars=\\\{\},fontsize=\small]
Time took without vectorization : 0.005651

    \end{Verbatim}

	\begin{figure}[H]
    \centering
    \caption{Figure 4: Plot of $\int_{0}^{x} dx/(1+t^{2}) $ using trapezoidal rule with loops}
    \adjustimage{max size={0.7\linewidth}{0.7\paperheight}}{output_1_1.png}
	\end{figure}
	
    
    \subsubsection{Trapezoidal rule using Vectorized Method}

    \begin{Verbatim}[commandchars=\\\{\},fontsize=\small]
{\color{incolor}In [{\color{incolor}39}]:} \PY{c+c1}{\PYZsh{}Using Vectorized code and noting the time it takes to run}
         \PY{n}{t3} \PY{o}{=} \PY{n}{t}\PY{o}{.}\PY{n}{time}\PY{p}{(}\PY{p}{)}
         \PY{n}{I\PYZus{}vect} \PY{o}{=} \PY{n}{h}\PY{o}{*}\PY{p}{(}\PY{n}{cumsum}\PY{p}{(}\PY{n}{f}\PY{p}{(}\PY{n}{x}\PY{p}{)}\PY{p}{)}\PY{o}{\PYZhy{}}\PY{l+m+mf}{0.5}\PY{o}{*}\PY{p}{(}\PY{n}{f}\PY{p}{(}\PY{n}{x}\PY{p}{[}\PY{l+m+mi}{0}\PY{p}{]}\PY{p}{)}\PY{o}{+}\PY{n}{f}\PY{p}{(}\PY{n}{x}\PY{p}{)}\PY{p}{)}\PY{p}{)}          \PY{c+c1}{\PYZsh{}vectorized code}
         \PY{n}{t4} \PY{o}{=} \PY{n}{t}\PY{o}{.}\PY{n}{time}\PY{p}{(}\PY{p}{)}
         
         \PY{k}{print} \PY{p}{(}\PY{l+s+s2}{\PYZdq{}}\PY{l+s+s2}{Time took with vectorization : }\PY{l+s+si}{\PYZpc{}g}\PY{l+s+s2}{\PYZdq{}} \PY{o}{\PYZpc{}}\PY{p}{(}\PY{n}{t4}\PY{o}{\PYZhy{}}\PY{n}{t3}\PY{p}{)}\PY{p}{)}
         \PY{k}{print} \PY{p}{(}\PY{l+s+s2}{\PYZdq{}}\PY{l+s+s2}{Speed up factor while vectorizing code : }\PY{l+s+si}{\PYZpc{}g}\PY{l+s+s2}{\PYZdq{}} \PY{o}{\PYZpc{}} \PY{p}{(}\PY{p}{(}\PY{n}{t2}\PY{o}{\PYZhy{}}\PY{n}{t1}\PY{p}{)}\PY{o}{/}\PY{p}{(}\PY{n}{t4}\PY{o}{\PYZhy{}}\PY{n}{t3}\PY{p}{)}\PY{p}{)}\PY{p}{)}
         
         \PY{c+c1}{\PYZsh{}plotting integral  vs x using vectorized technique}
         \PY{n}{fig5} \PY{o}{=} \PY{n}{figure}\PY{p}{(}\PY{p}{)}
         \PY{n}{plot}\PY{p}{(}\PY{n}{x}\PY{p}{,}\PY{n}{I\PYZus{}vect}\PY{p}{)}
         \PY{n}{fig5}\PY{o}{.}\PY{n}{suptitle}\PY{p}{(}\PY{l+s+sa}{r}\PY{l+s+s2}{\PYZdq{}}\PY{l+s+s2}{Vectorized method : \PYZdl{}}\PY{l+s+s2}{\PYZbs{}}\PY{l+s+s2}{int\PYZus{}\PYZob{}0\PYZcb{}\PYZca{}\PYZob{}x\PYZcb{} dx/(1+t\PYZca{}\PYZob{}2\PYZcb{}) \PYZdl{}}\PY{l+s+s2}{\PYZdq{}}\PY{p}{,}\PY{n}{fontsize}\PY{o}{=}\PY{l+m+mi}{12}\PY{p}{)}
         \PY{n}{xlabel}\PY{p}{(}\PY{l+s+s2}{\PYZdq{}}\PY{l+s+s2}{x}\PY{l+s+s2}{\PYZdq{}}\PY{p}{)}
         \PY{n}{ylabel}\PY{p}{(}\PY{l+s+s2}{\PYZdq{}}\PY{l+s+s2}{\PYZdl{}}\PY{l+s+s2}{\PYZbs{}}\PY{l+s+s2}{int\PYZus{}\PYZob{}0\PYZcb{}\PYZca{}\PYZob{}x\PYZcb{} dx/(1+t\PYZca{}\PYZob{}2\PYZcb{}) \PYZdl{}}\PY{l+s+s2}{\PYZdq{}}\PY{p}{)}
         \PY{n}{fig5}\PY{o}{.}\PY{n}{savefig}\PY{p}{(}\PY{l+s+s1}{\PYZsq{}}\PY{l+s+s1}{5.jpg}\PY{l+s+s1}{\PYZsq{}}\PY{p}{)}
         \PY{n}{show}\PY{p}{(}\PY{p}{)}    
\end{Verbatim}

    \begin{Verbatim}[commandchars=\\\{\},fontsize=\small]
Time took with vectorization : 0.000273943
Speed up factor while vectorizing code : 20.6284

    \end{Verbatim}

	\begin{figure}[H]
    \centering
    \caption{Figure 5: Plot of $\int_{0}^{x} dx/(1+t^{2}) $ using trapezoidal rule after Vectorizing the code}
    \adjustimage{max size={0.7\linewidth}{0.7\paperheight}}{output_2_1.png}
	\end{figure}
	
    
    \subsubsection{Calculating Estimate error and Exact error associated with Trapezoidal rule by halving stepsize}

\begin{Verbatim}[commandchars=\\\{\},fontsize=\small]
{\color{incolor}In [{\color{incolor}60}]:} \PY{c+c1}{\PYZsh{}Estimating error by halving stepsize when greater the certain tolerance}
         \PY{c+c1}{\PYZsh{}initialising h vector}
         
         \PY{n}{h} \PY{o}{=} \PY{p}{[}\PY{p}{]}
         \PY{n}{tol} \PY{o}{=} \PY{l+m+mi}{10}\PY{o}{*}\PY{o}{*}\PY{o}{\PYZhy{}}\PY{l+m+mi}{8}    \PY{c+c1}{\PYZsh{}tolerance of 10\PYZca{}(\PYZhy{}8)}
         \PY{n}{est\PYZus{}err} \PY{o}{=} \PY{p}{[}\PY{p}{]}    \PY{c+c1}{\PYZsh{}estimated error initialisation}
         \PY{n}{act\PYZus{}err} \PY{o}{=} \PY{p}{[}\PY{p}{]}    \PY{c+c1}{\PYZsh{}actual\PYZus{}error initialisation}
         \PY{n}{i}\PY{o}{=}\PY{l+m+mi}{0}
         \PY{n}{h}\PY{o}{.}\PY{n}{append}\PY{p}{(}\PY{l+m+mf}{0.5}\PY{p}{)}
         
         
         \PY{c+c1}{\PYZsh{}while loop runs until est\PYZus{}err is less than tolerance}
         
         \PY{k}{while}\PY{p}{(}\PY{n+nb+bp}{True}\PY{p}{)}\PY{p}{:}
             \PY{c+c1}{\PYZsh{}temperory estimated\PYZus{}Error array,used to find max error among common points}
             \PY{n}{est\PYZus{}err\PYZus{}temp} \PY{o}{=} \PY{p}{[}\PY{p}{]}  
             \PY{n}{h}\PY{o}{.}\PY{n}{append}\PY{p}{(}\PY{n}{h}\PY{p}{[}\PY{n}{i}\PY{p}{]}\PY{o}{/}\PY{l+m+mf}{2.0}\PY{p}{)}             \PY{c+c1}{\PYZsh{} halving h by 2}
             \PY{n}{x}\PY{o}{=}\PY{n}{arange}\PY{p}{(}\PY{l+m+mi}{0}\PY{p}{,}\PY{l+m+mi}{5}\PY{p}{,}\PY{n}{h}\PY{p}{[}\PY{n}{i}\PY{p}{]}\PY{p}{)}           \PY{c+c1}{\PYZsh{} creating input with current stepsize}
             \PY{n}{x\PYZus{}next} \PY{o}{=} \PY{n}{arange}\PY{p}{(}\PY{l+m+mi}{0}\PY{p}{,}\PY{l+m+mi}{5}\PY{p}{,}\PY{n}{h}\PY{p}{[}\PY{n}{i}\PY{o}{+}\PY{l+m+mi}{1}\PY{p}{]}\PY{p}{)}  \PY{c+c1}{\PYZsh{}input with half of current stepsize}
             
             \PY{c+c1}{\PYZsh{}calculating Integrals with current h and h/2}
             \PY{n}{I\PYZus{}curr} \PY{o}{=} \PY{n}{h}\PY{p}{[}\PY{n}{i}\PY{p}{]}\PY{o}{*}\PY{p}{(}\PY{n}{cumsum}\PY{p}{(}\PY{n}{f}\PY{p}{(}\PY{n}{x}\PY{p}{)}\PY{p}{)}\PY{o}{\PYZhy{}}\PY{l+m+mf}{0.5}\PY{o}{*}\PY{p}{(}\PY{n}{f}\PY{p}{(}\PY{n}{x}\PY{p}{[}\PY{l+m+mi}{0}\PY{p}{]}\PY{p}{)}\PY{o}{+}\PY{n}{f}\PY{p}{(}\PY{n}{x}\PY{p}{)}\PY{p}{)}\PY{p}{)}
             \PY{n}{I\PYZus{}next} \PY{o}{=} \PY{n}{h}\PY{p}{[}\PY{n}{i}\PY{o}{+}\PY{l+m+mi}{1}\PY{p}{]}\PY{o}{*}\PY{p}{(}\PY{n}{cumsum}\PY{p}{(}\PY{n}{f}\PY{p}{(}\PY{n}{x\PYZus{}next}\PY{p}{)}\PY{p}{)}\PY{o}{\PYZhy{}}\PY{l+m+mf}{0.5}\PY{o}{*}\PY{p}{(}\PY{n}{f}\PY{p}{(}\PY{n}{x\PYZus{}next}\PY{p}{[}\PY{l+m+mi}{0}\PY{p}{]}\PY{p}{)}\PY{o}{+}\PY{n}{f}\PY{p}{(}\PY{n}{x\PYZus{}next}\PY{p}{)}\PY{p}{)}\PY{p}{)}
             
             \PY{c+c1}{\PYZsh{}finding common elements }
             \PY{n}{x\PYZus{}com} \PY{o}{=} \PY{n}{np}\PY{o}{.}\PY{n}{intersect1d}\PY{p}{(}\PY{n}{x}\PY{p}{,}\PY{n}{x\PYZus{}next}\PY{p}{)}
             
             \PY{c+c1}{\PYZsh{}finding error between Integrals at common elements}
             \PY{k}{for} \PY{n}{k} \PY{o+ow}{in} \PY{n+nb}{range}\PY{p}{(}\PY{n+nb}{len}\PY{p}{(}\PY{n}{x\PYZus{}com}\PY{p}{)}\PY{p}{)}\PY{p}{:}
                 \PY{n}{est\PYZus{}err\PYZus{}temp}\PY{o}{.}\PY{n}{append}\PY{p}{(}\PY{n}{I\PYZus{}next}\PY{p}{[}\PY{l+m+mi}{2}\PY{o}{*}\PY{n}{k}\PY{p}{]}\PY{o}{\PYZhy{}}\PY{n}{I\PYZus{}curr}\PY{p}{[}\PY{n}{k}\PY{p}{]}\PY{p}{)}
             
             \PY{c+c1}{\PYZsh{}finding index of max error among common elements}
             \PY{n}{arg\PYZus{}max\PYZus{}err} \PY{o}{=} \PY{n}{argmax}\PY{p}{(}\PY{n}{absolute}\PY{p}{(}\PY{n}{est\PYZus{}err\PYZus{}temp}\PY{p}{)}\PY{p}{)}
             
             \PY{c+c1}{\PYZsh{}finding actual error and estimated error}
             \PY{n}{act\PYZus{}err}\PY{o}{.}\PY{n}{append}\PY{p}{(}\PY{n}{arctan}\PY{p}{(}\PY{n}{x\PYZus{}com}\PY{p}{[}\PY{n}{arg\PYZus{}max\PYZus{}err}\PY{p}{]}\PY{p}{)}\PY{o}{\PYZhy{}}\PY{n}{I\PYZus{}curr}\PY{p}{[}\PY{n}{arg\PYZus{}max\PYZus{}err}\PY{p}{]}\PY{p}{)}
             \PY{n}{est\PYZus{}err}\PY{o}{.}\PY{n}{append}\PY{p}{(}\PY{n}{est\PYZus{}err\PYZus{}temp}\PY{p}{[}\PY{n}{arg\PYZus{}max\PYZus{}err}\PY{p}{]}\PY{p}{)}      
             
             \PY{c+c1}{\PYZsh{}incrementing i when est\PYZus{}error is greater than tolerance}
             \PY{k}{if}\PY{p}{(}\PY{n}{est\PYZus{}err}\PY{p}{[}\PY{n}{i}\PY{p}{]}\PY{o}{\PYZgt{}}\PY{n}{tol}\PY{p}{)}\PY{p}{:}
                 \PY{n}{i}\PY{o}{+}\PY{o}{=}\PY{l+m+mi}{1}
             \PY{k}{else}\PY{p}{:}
                 \PY{k}{break}\PY{p}{;}
         
         \PY{c+c1}{\PYZsh{}Tabulating h values vs est\PYZus{}error vs act\PYZus{}errors}
         \PY{n}{table} \PY{o}{=} \PY{n+nb}{zip}\PY{p}{(}\PY{n}{h}\PY{p}{,}\PY{n}{est\PYZus{}err}\PY{p}{,}\PY{n}{act\PYZus{}err}\PY{p}{)}
         \PY{n}{headers} \PY{o}{=} \PY{p}{[}\PY{l+s+s2}{\PYZdq{}}\PY{l+s+s2}{Stepsize h}\PY{l+s+s2}{\PYZdq{}}\PY{p}{,}\PY{l+s+s2}{\PYZdq{}}\PY{l+s+s2}{Estimated Error}\PY{l+s+s2}{\PYZdq{}}\PY{p}{,}\PY{l+s+s2}{\PYZdq{}}\PY{l+s+s2}{Actual Error}\PY{l+s+s2}{\PYZdq{}}\PY{p}{]}
         \PY{c+c1}{\PYZsh{}tabulating arctan values vs quad function values}
         \PY{k}{print} \PY{n}{tabulate}\PY{p}{(}\PY{n}{table}\PY{p}{,}\PY{n}{tablefmt}\PY{o}{=}\PY{l+s+s2}{\PYZdq{}}\PY{l+s+s2}{fancy\PYZus{}grid}\PY{l+s+s2}{\PYZdq{}}\PY{p}{,}\PY{n}{headers}\PY{o}{=}\PY{n}{headers}\PY{p}{)} 
             
         \PY{c+c1}{\PYZsh{}printing the best value of h,it is last but index since its do\PYZhy{}while loop}
         \PY{k}{print}\PY{l+s+s2}{\PYZdq{}}\PY{l+s+s2}{Best value of h is : }\PY{l+s+si}{\PYZpc{}g}\PY{l+s+s2}{\PYZdq{}} \PY{o}{\PYZpc{}}\PY{p}{(}\PY{n}{h}\PY{p}{[}\PY{n+nb}{len}\PY{p}{(}\PY{n}{h}\PY{p}{)}\PY{o}{\PYZhy{}}\PY{l+m+mi}{2}\PY{p}{]}\PY{p}{)}
         
         \PY{n}{fig6} \PY{o}{=} \PY{n}{figure}\PY{p}{(}\PY{p}{)}
         \PY{n}{loglog}\PY{p}{(}\PY{n}{h}\PY{p}{[}\PY{p}{:}\PY{o}{\PYZhy{}}\PY{l+m+mi}{1}\PY{p}{]}\PY{p}{,}\PY{n}{est\PYZus{}err}\PY{p}{,}\PY{l+s+s1}{\PYZsq{}}\PY{l+s+s1}{g+}\PY{l+s+s1}{\PYZsq{}}\PY{p}{)}
         \PY{n}{loglog}\PY{p}{(}\PY{n}{h}\PY{p}{[}\PY{p}{:}\PY{o}{\PYZhy{}}\PY{l+m+mi}{1}\PY{p}{]}\PY{p}{,}\PY{n}{act\PYZus{}err}\PY{p}{,}\PY{l+s+s1}{\PYZsq{}}\PY{l+s+s1}{ro}\PY{l+s+s1}{\PYZsq{}}\PY{p}{)}
         \PY{n}{legend}\PY{p}{(}\PY{p}{(}\PY{l+s+s2}{\PYZdq{}}\PY{l+s+s2}{Estimated error}\PY{l+s+s2}{\PYZdq{}}\PY{p}{,}\PY{l+s+s2}{\PYZdq{}}\PY{l+s+s2}{Exact error}\PY{l+s+s2}{\PYZdq{}}\PY{p}{)}\PY{p}{)}
         \PY{n}{fig6}\PY{o}{.}\PY{n}{suptitle}\PY{p}{(}\PY{l+s+sa}{r}\PY{l+s+s2}{\PYZdq{}}\PY{l+s+s2}{Estimated Error vs Actual error for \PYZdl{}}\PY{l+s+s2}{\PYZbs{}}\PY{l+s+s2}{int\PYZus{}\PYZob{}0\PYZcb{}\PYZca{}\PYZob{}x\PYZcb{} dx/(1+t\PYZca{}\PYZob{}2\PYZcb{}) \PYZdl{}}\PY{l+s+s2}{\PYZdq{}}\PY{p}{,} \PY{n}{fontsize}\PY{o}{=}\PY{l+m+mi}{12}\PY{p}{)}
         \PY{n}{xlabel}\PY{p}{(}\PY{l+s+s2}{\PYZdq{}}\PY{l+s+s2}{h}\PY{l+s+s2}{\PYZdq{}}\PY{p}{)}
         \PY{n}{ylabel}\PY{p}{(}\PY{l+s+s2}{\PYZdq{}}\PY{l+s+s2}{Error}\PY{l+s+s2}{\PYZdq{}}\PY{p}{)}
         \PY{n}{fig6}\PY{o}{.}\PY{n}{savefig}\PY{p}{(}\PY{l+s+s1}{\PYZsq{}}\PY{l+s+s1}{6.jpg}\PY{l+s+s1}{\PYZsq{}}\PY{p}{)}
         \PY{n}{show}\PY{p}{(}\PY{p}{)}
\end{Verbatim}

\begin{table}[H]
\centering
\begin{tabular}{|c|c|c|}
\hline
h & estimated error & exact error \\
\hline
0.5 & 0.0102941 & 0.0136476 \\
0.25 & 0.00251891 & 0.00335349 \\
0.125 & 0.000632009 & 0.000842472 \\
0.0625 & 0.000158555 & 0.00021139 \\
0.03125 & 3.96273e-05 & 5.28354e-05 \\
0.015625 & 9.91109e-06 & 1.32147e-05 \\
0.0078125 & 2.47773e-06 & 3.30364e-06 \\
0.00390625 & 6.1943e-07 & 8.25906e-07 \\
0.00195312 & 1.54857e-07 & 2.06476e-07 \\
0.000976562 & 3.87144e-08 & 5.16191e-08 \\
0.000488281 & 9.67859e-09 & 1.29048e-08 \\
\hline
\end{tabular}
\caption{Best value of h is : 0.000488281}
\end{table}

     
	\begin{figure}[H]
    \centering
    \caption{Figure 6: Comparison between Estimated Error and Exact Error in loglog plot}
    \adjustimage{max size={0.7\linewidth}{0.7\paperheight}}{output_3_1.png}
	\end{figure}
	

    % Add a bibliography block to the postdoc
    
    
    \section{Results and Discussion}
    
	\begin{itemize}
	\item So in this assignment we utilised the scientific python to do numerical computations like MATLAB and we saw that plotting and handling arrays and vectors were very simple. \\
    \item It also showcased that Vectorizing code improves the speed of the code by a great factor than using traditional for loops.And we also used scipy quad function to integrate a function and compared the error associated with it from actual value. 
    \item And later we used Numerical methods to calculate integral of f(x)  by trapezoidal rule.This method is very useful since it can be used incase of Non-integral functions too!.
    \item And We also learnt how to find the Estimated Error from error between common points with different stepsizes which is very useful  incase of non-integrable functions as exact values are not known.
    \item And we compared the Estimated Error and Exact Error and got to know that the estimated error (i.e., the maximum error on common points) is not equal to the actual error. This is an
important thing to realise - numerical methods are always a little uncertain.
Only in special cases do we know the exact error. With a function whose
integral is not available, we have to depend on the estimated error to know
when to stop.
	\end{itemize}


    
    \end{document}
