
% Default to the notebook output style

    


% Inherit from the specified cell style.




    
\documentclass[10pt,a4paper]{article}
	
    \usepackage[T1]{fontenc}
    % Nicer default font (+ math font) than Computer Modern for most use cases
    \usepackage{mathpazo}

  \usepackage{float}
	\restylefloat{table}
		
    % Basic figure setup, for now with no caption control since it's done
    % automatically by Pandoc (which extracts ![](path) syntax from Markdown).
    \usepackage{graphicx}
    % We will generate all images so they have a width \maxwidth. This means
    % that they will get their normal width if they fit onto the page, but
    % are scaled down if they would overflow the margins.
    \makeatletter
    \def\maxwidth{\ifdim\Gin@nat@width>\linewidth\linewidth
    \else\Gin@nat@width\fi}
    \makeatother
    \let\Oldincludegraphics\includegraphics
    % Set max figure width to be 80% of text width, for now hardcoded.
    \renewcommand{\includegraphics}[1]{\Oldincludegraphics[width=.8\maxwidth]{#1}}
    % Ensure that by default, figures have no caption (until we provide a
    % proper Figure object with a Caption API and a way to capture that
    % in the conversion process - todo).
    \usepackage{caption}
    \DeclareCaptionLabelFormat{nolabel}{}
    \captionsetup{labelformat=nolabel}

    \usepackage{adjustbox} % Used to constrain images to a maximum size 
    \usepackage{xcolor} % Allow colors to be defined
    \usepackage{enumerate} % Needed for markdown enumerations to work
    \usepackage{geometry} % Used to adjust the document margins
    \usepackage{amsmath} % Equations
    \usepackage{amssymb} % Equations
    \usepackage{textcomp} % defines textquotesingle
    % Hack from http://tex.stackexchange.com/a/47451/13684:
    \AtBeginDocument{%
        \def\PYZsq{\textquotesingle}% Upright quotes in Pygmentized code
    }
    



\usepackage{color}
\usepackage{verbatim}
\definecolor{verbgray}{gray}{0.9}

\let\oldv\verbatim
\let\oldendv\endverbatim

\def\verbatim{\par\setbox0\vbox\bgroup\oldv}
\def\endverbatim{\oldendv\egroup\fboxsep0pt \noindent\colorbox[gray]{0.8}{\usebox0}\par}

\usepackage{color}
\usepackage{xcolor}
\usepackage{listings}

\usepackage{caption}
\DeclareCaptionFont{white}{\color{white}}
\DeclareCaptionFormat{listing}{\colorbox{gray}{\parbox{\textwidth}{#1#2#3}}}
\captionsetup[lstlisting]{format=listing,labelfont=white,textfont=white}

% This concludes the preamble

        \usepackage{fancyvrb}
\DefineVerbatimEnvironment{Verbatim}{Verbatim}{%
  gobble=2,
  numbers=left,
  numbersep=2mm,
  frame=lines,
  framerule=0.2mm,
}

    \usepackage{upquote} % Upright quotes for verbatim code
    \usepackage{eurosym} % defines \euro
    \usepackage[mathletters]{ucs} % Extended unicode (utf-8) support
    \usepackage[utf8x]{inputenc} % Allow utf-8 characters in the tex document
    \usepackage{fancyvrb} % verbatim replacement that allows latex
    \usepackage{grffile} % extends the file name processing of package graphics 
                         % to support a larger range 
    % The hyperref package gives us a pdf with properly built
    % internal navigation ('pdf bookmarks' for the table of contents,
    % internal cross-reference links, web links for URLs, etc.)
    \usepackage{hyperref}
    \usepackage{longtable} % longtable support required by pandoc >1.10
    \usepackage{booktabs}  % table support for pandoc > 1.12.2
    \usepackage[inline]{enumitem} % IRkernel/repr support (it uses the enumerate* environment)
    \usepackage[normalem]{ulem} % ulem is needed to support strikethroughs (\sout)
                                % normalem makes italics be italics, not underlines
    

    
    
    % Colors for the hyperref package
    \definecolor{urlcolor}{rgb}{0,.145,.698}
    \definecolor{linkcolor}{rgb}{.71,0.21,0.01}
    \definecolor{citecolor}{rgb}{.12,.54,.11}

    % ANSI colors
    \definecolor{ansi-black}{HTML}{3E424D}
    \definecolor{ansi-black-intense}{HTML}{282C36}
    \definecolor{ansi-red}{HTML}{E75C58}
    \definecolor{ansi-red-intense}{HTML}{B22B31}
    \definecolor{ansi-green}{HTML}{00A250}
    \definecolor{ansi-green-intense}{HTML}{007427}
    \definecolor{ansi-yellow}{HTML}{DDB62B}
    \definecolor{ansi-yellow-intense}{HTML}{B27D12}
    \definecolor{ansi-blue}{HTML}{208FFB}
    \definecolor{ansi-blue-intense}{HTML}{0065CA}
    \definecolor{ansi-magenta}{HTML}{D160C4}
    \definecolor{ansi-magenta-intense}{HTML}{A03196}
    \definecolor{ansi-cyan}{HTML}{60C6C8}
    \definecolor{ansi-cyan-intense}{HTML}{258F8F}
    \definecolor{ansi-white}{HTML}{C5C1B4}
    \definecolor{ansi-white-intense}{HTML}{A1A6B2}

    % commands and environments needed by pandoc snippets
    % extracted from the output of `pandoc -s`
    \providecommand{\tightlist}{%
      \setlength{\itemsep}{0pt}\setlength{\parskip}{0pt}}
    \DefineVerbatimEnvironment{Highlighting}{Verbatim}{commandchars=\\\{\}}
    % Add ',fontsize=\small' for more characters per line
    \newenvironment{Shaded}{}{}
    \newcommand{\KeywordTok}[1]{\textcolor[rgb]{0.00,0.44,0.13}{\textbf{{#1}}}}
    \newcommand{\DataTypeTok}[1]{\textcolor[rgb]{0.56,0.13,0.00}{{#1}}}
    \newcommand{\DecValTok}[1]{\textcolor[rgb]{0.25,0.63,0.44}{{#1}}}
    \newcommand{\BaseNTok}[1]{\textcolor[rgb]{0.25,0.63,0.44}{{#1}}}
    \newcommand{\FloatTok}[1]{\textcolor[rgb]{0.25,0.63,0.44}{{#1}}}
    \newcommand{\CharTok}[1]{\textcolor[rgb]{0.25,0.44,0.63}{{#1}}}
    \newcommand{\StringTok}[1]{\textcolor[rgb]{0.25,0.44,0.63}{{#1}}}
    \newcommand{\CommentTok}[1]{\textcolor[rgb]{0.38,0.63,0.69}{\textit{{#1}}}}
    \newcommand{\OtherTok}[1]{\textcolor[rgb]{0.00,0.44,0.13}{{#1}}}
    \newcommand{\AlertTok}[1]{\textcolor[rgb]{1.00,0.00,0.00}{\textbf{{#1}}}}
    \newcommand{\FunctionTok}[1]{\textcolor[rgb]{0.02,0.16,0.49}{{#1}}}
    \newcommand{\RegionMarkerTok}[1]{{#1}}
    \newcommand{\ErrorTok}[1]{\textcolor[rgb]{1.00,0.00,0.00}{\textbf{{#1}}}}
    \newcommand{\NormalTok}[1]{{#1}}
    
    % Additional commands for more recent versions of Pandoc
    \newcommand{\ConstantTok}[1]{\textcolor[rgb]{0.53,0.00,0.00}{{#1}}}
    \newcommand{\SpecialCharTok}[1]{\textcolor[rgb]{0.25,0.44,0.63}{{#1}}}
    \newcommand{\VerbatimStringTok}[1]{\textcolor[rgb]{0.25,0.44,0.63}{{#1}}}
    \newcommand{\SpecialStringTok}[1]{\textcolor[rgb]{0.73,0.40,0.53}{{#1}}}
    \newcommand{\ImportTok}[1]{{#1}}
    \newcommand{\DocumentationTok}[1]{\textcolor[rgb]{0.73,0.13,0.13}{\textit{{#1}}}}
    \newcommand{\AnnotationTok}[1]{\textcolor[rgb]{0.38,0.63,0.69}{\textbf{\textit{{#1}}}}}
    \newcommand{\CommentVarTok}[1]{\textcolor[rgb]{0.38,0.63,0.69}{\textbf{\textit{{#1}}}}}
    \newcommand{\VariableTok}[1]{\textcolor[rgb]{0.10,0.09,0.49}{{#1}}}
    \newcommand{\ControlFlowTok}[1]{\textcolor[rgb]{0.00,0.44,0.13}{\textbf{{#1}}}}
    \newcommand{\OperatorTok}[1]{\textcolor[rgb]{0.40,0.40,0.40}{{#1}}}
    \newcommand{\BuiltInTok}[1]{{#1}}
    \newcommand{\ExtensionTok}[1]{{#1}}
    \newcommand{\PreprocessorTok}[1]{\textcolor[rgb]{0.74,0.48,0.00}{{#1}}}
    \newcommand{\AttributeTok}[1]{\textcolor[rgb]{0.49,0.56,0.16}{{#1}}}
    \newcommand{\InformationTok}[1]{\textcolor[rgb]{0.38,0.63,0.69}{\textbf{\textit{{#1}}}}}
    \newcommand{\WarningTok}[1]{\textcolor[rgb]{0.38,0.63,0.69}{\textbf{\textit{{#1}}}}}
    
    
    % Define a nice break command that doesn't care if a line doesn't already
    % exist.
    \def\br{\hspace*{\fill} \\* }
    % Math Jax compatability definitions
    \def\gt{>}
    \def\lt{<}
    
     % Document parameters
    \title{Simulating Tubelight model in Python \\ Assignment 6}
    \author{Rohithram R, EE16B031 \\ B.Tech Electrical Engineering, IIT Madras}
    \date{\today \\ First created on March 10,2018}	
    
    

    % Pygments definitions
    
\makeatletter
\def\PY@reset{\let\PY@it=\relax \let\PY@bf=\relax%
    \let\PY@ul=\relax \let\PY@tc=\relax%
    \let\PY@bc=\relax \let\PY@ff=\relax}
\def\PY@tok#1{\csname PY@tok@#1\endcsname}
\def\PY@toks#1+{\ifx\relax#1\empty\else%
    \PY@tok{#1}\expandafter\PY@toks\fi}
\def\PY@do#1{\PY@bc{\PY@tc{\PY@ul{%
    \PY@it{\PY@bf{\PY@ff{#1}}}}}}}
\def\PY#1#2{\PY@reset\PY@toks#1+\relax+\PY@do{#2}}

\expandafter\def\csname PY@tok@w\endcsname{\def\PY@tc##1{\textcolor[rgb]{0.73,0.73,0.73}{##1}}}
\expandafter\def\csname PY@tok@c\endcsname{\let\PY@it=\textit\def\PY@tc##1{\textcolor[rgb]{0.25,0.50,0.50}{##1}}}
\expandafter\def\csname PY@tok@cp\endcsname{\def\PY@tc##1{\textcolor[rgb]{0.74,0.48,0.00}{##1}}}
\expandafter\def\csname PY@tok@k\endcsname{\let\PY@bf=\textbf\def\PY@tc##1{\textcolor[rgb]{0.00,0.50,0.00}{##1}}}
\expandafter\def\csname PY@tok@kp\endcsname{\def\PY@tc##1{\textcolor[rgb]{0.00,0.50,0.00}{##1}}}
\expandafter\def\csname PY@tok@kt\endcsname{\def\PY@tc##1{\textcolor[rgb]{0.69,0.00,0.25}{##1}}}
\expandafter\def\csname PY@tok@o\endcsname{\def\PY@tc##1{\textcolor[rgb]{0.40,0.40,0.40}{##1}}}
\expandafter\def\csname PY@tok@ow\endcsname{\let\PY@bf=\textbf\def\PY@tc##1{\textcolor[rgb]{0.67,0.13,1.00}{##1}}}
\expandafter\def\csname PY@tok@nb\endcsname{\def\PY@tc##1{\textcolor[rgb]{0.00,0.50,0.00}{##1}}}
\expandafter\def\csname PY@tok@nf\endcsname{\def\PY@tc##1{\textcolor[rgb]{0.00,0.00,1.00}{##1}}}
\expandafter\def\csname PY@tok@nc\endcsname{\let\PY@bf=\textbf\def\PY@tc##1{\textcolor[rgb]{0.00,0.00,1.00}{##1}}}
\expandafter\def\csname PY@tok@nn\endcsname{\let\PY@bf=\textbf\def\PY@tc##1{\textcolor[rgb]{0.00,0.00,1.00}{##1}}}
\expandafter\def\csname PY@tok@ne\endcsname{\let\PY@bf=\textbf\def\PY@tc##1{\textcolor[rgb]{0.82,0.25,0.23}{##1}}}
\expandafter\def\csname PY@tok@nv\endcsname{\def\PY@tc##1{\textcolor[rgb]{0.10,0.09,0.49}{##1}}}
\expandafter\def\csname PY@tok@no\endcsname{\def\PY@tc##1{\textcolor[rgb]{0.53,0.00,0.00}{##1}}}
\expandafter\def\csname PY@tok@nl\endcsname{\def\PY@tc##1{\textcolor[rgb]{0.63,0.63,0.00}{##1}}}
\expandafter\def\csname PY@tok@ni\endcsname{\let\PY@bf=\textbf\def\PY@tc##1{\textcolor[rgb]{0.60,0.60,0.60}{##1}}}
\expandafter\def\csname PY@tok@na\endcsname{\def\PY@tc##1{\textcolor[rgb]{0.49,0.56,0.16}{##1}}}
\expandafter\def\csname PY@tok@nt\endcsname{\let\PY@bf=\textbf\def\PY@tc##1{\textcolor[rgb]{0.00,0.50,0.00}{##1}}}
\expandafter\def\csname PY@tok@nd\endcsname{\def\PY@tc##1{\textcolor[rgb]{0.67,0.13,1.00}{##1}}}
\expandafter\def\csname PY@tok@s\endcsname{\def\PY@tc##1{\textcolor[rgb]{0.73,0.13,0.13}{##1}}}
\expandafter\def\csname PY@tok@sd\endcsname{\let\PY@it=\textit\def\PY@tc##1{\textcolor[rgb]{0.73,0.13,0.13}{##1}}}
\expandafter\def\csname PY@tok@si\endcsname{\let\PY@bf=\textbf\def\PY@tc##1{\textcolor[rgb]{0.73,0.40,0.53}{##1}}}
\expandafter\def\csname PY@tok@se\endcsname{\let\PY@bf=\textbf\def\PY@tc##1{\textcolor[rgb]{0.73,0.40,0.13}{##1}}}
\expandafter\def\csname PY@tok@sr\endcsname{\def\PY@tc##1{\textcolor[rgb]{0.73,0.40,0.53}{##1}}}
\expandafter\def\csname PY@tok@ss\endcsname{\def\PY@tc##1{\textcolor[rgb]{0.10,0.09,0.49}{##1}}}
\expandafter\def\csname PY@tok@sx\endcsname{\def\PY@tc##1{\textcolor[rgb]{0.00,0.50,0.00}{##1}}}
\expandafter\def\csname PY@tok@m\endcsname{\def\PY@tc##1{\textcolor[rgb]{0.40,0.40,0.40}{##1}}}
\expandafter\def\csname PY@tok@gh\endcsname{\let\PY@bf=\textbf\def\PY@tc##1{\textcolor[rgb]{0.00,0.00,0.50}{##1}}}
\expandafter\def\csname PY@tok@gu\endcsname{\let\PY@bf=\textbf\def\PY@tc##1{\textcolor[rgb]{0.50,0.00,0.50}{##1}}}
\expandafter\def\csname PY@tok@gd\endcsname{\def\PY@tc##1{\textcolor[rgb]{0.63,0.00,0.00}{##1}}}
\expandafter\def\csname PY@tok@gi\endcsname{\def\PY@tc##1{\textcolor[rgb]{0.00,0.63,0.00}{##1}}}
\expandafter\def\csname PY@tok@gr\endcsname{\def\PY@tc##1{\textcolor[rgb]{1.00,0.00,0.00}{##1}}}
\expandafter\def\csname PY@tok@ge\endcsname{\let\PY@it=\textit}
\expandafter\def\csname PY@tok@gs\endcsname{\let\PY@bf=\textbf}
\expandafter\def\csname PY@tok@gp\endcsname{\let\PY@bf=\textbf\def\PY@tc##1{\textcolor[rgb]{0.00,0.00,0.50}{##1}}}
\expandafter\def\csname PY@tok@go\endcsname{\def\PY@tc##1{\textcolor[rgb]{0.53,0.53,0.53}{##1}}}
\expandafter\def\csname PY@tok@gt\endcsname{\def\PY@tc##1{\textcolor[rgb]{0.00,0.27,0.87}{##1}}}
\expandafter\def\csname PY@tok@err\endcsname{\def\PY@bc##1{\setlength{\fboxsep}{0pt}\fcolorbox[rgb]{1.00,0.00,0.00}{1,1,1}{\strut ##1}}}
\expandafter\def\csname PY@tok@kc\endcsname{\let\PY@bf=\textbf\def\PY@tc##1{\textcolor[rgb]{0.00,0.50,0.00}{##1}}}
\expandafter\def\csname PY@tok@kd\endcsname{\let\PY@bf=\textbf\def\PY@tc##1{\textcolor[rgb]{0.00,0.50,0.00}{##1}}}
\expandafter\def\csname PY@tok@kn\endcsname{\let\PY@bf=\textbf\def\PY@tc##1{\textcolor[rgb]{0.00,0.50,0.00}{##1}}}
\expandafter\def\csname PY@tok@kr\endcsname{\let\PY@bf=\textbf\def\PY@tc##1{\textcolor[rgb]{0.00,0.50,0.00}{##1}}}
\expandafter\def\csname PY@tok@bp\endcsname{\def\PY@tc##1{\textcolor[rgb]{0.00,0.50,0.00}{##1}}}
\expandafter\def\csname PY@tok@fm\endcsname{\def\PY@tc##1{\textcolor[rgb]{0.00,0.00,1.00}{##1}}}
\expandafter\def\csname PY@tok@vc\endcsname{\def\PY@tc##1{\textcolor[rgb]{0.10,0.09,0.49}{##1}}}
\expandafter\def\csname PY@tok@vg\endcsname{\def\PY@tc##1{\textcolor[rgb]{0.10,0.09,0.49}{##1}}}
\expandafter\def\csname PY@tok@vi\endcsname{\def\PY@tc##1{\textcolor[rgb]{0.10,0.09,0.49}{##1}}}
\expandafter\def\csname PY@tok@vm\endcsname{\def\PY@tc##1{\textcolor[rgb]{0.10,0.09,0.49}{##1}}}
\expandafter\def\csname PY@tok@sa\endcsname{\def\PY@tc##1{\textcolor[rgb]{0.73,0.13,0.13}{##1}}}
\expandafter\def\csname PY@tok@sb\endcsname{\def\PY@tc##1{\textcolor[rgb]{0.73,0.13,0.13}{##1}}}
\expandafter\def\csname PY@tok@sc\endcsname{\def\PY@tc##1{\textcolor[rgb]{0.73,0.13,0.13}{##1}}}
\expandafter\def\csname PY@tok@dl\endcsname{\def\PY@tc##1{\textcolor[rgb]{0.73,0.13,0.13}{##1}}}
\expandafter\def\csname PY@tok@s2\endcsname{\def\PY@tc##1{\textcolor[rgb]{0.73,0.13,0.13}{##1}}}
\expandafter\def\csname PY@tok@sh\endcsname{\def\PY@tc##1{\textcolor[rgb]{0.73,0.13,0.13}{##1}}}
\expandafter\def\csname PY@tok@s1\endcsname{\def\PY@tc##1{\textcolor[rgb]{0.73,0.13,0.13}{##1}}}
\expandafter\def\csname PY@tok@mb\endcsname{\def\PY@tc##1{\textcolor[rgb]{0.40,0.40,0.40}{##1}}}
\expandafter\def\csname PY@tok@mf\endcsname{\def\PY@tc##1{\textcolor[rgb]{0.40,0.40,0.40}{##1}}}
\expandafter\def\csname PY@tok@mh\endcsname{\def\PY@tc##1{\textcolor[rgb]{0.40,0.40,0.40}{##1}}}
\expandafter\def\csname PY@tok@mi\endcsname{\def\PY@tc##1{\textcolor[rgb]{0.40,0.40,0.40}{##1}}}
\expandafter\def\csname PY@tok@il\endcsname{\def\PY@tc##1{\textcolor[rgb]{0.40,0.40,0.40}{##1}}}
\expandafter\def\csname PY@tok@mo\endcsname{\def\PY@tc##1{\textcolor[rgb]{0.40,0.40,0.40}{##1}}}
\expandafter\def\csname PY@tok@ch\endcsname{\let\PY@it=\textit\def\PY@tc##1{\textcolor[rgb]{0.25,0.50,0.50}{##1}}}
\expandafter\def\csname PY@tok@cm\endcsname{\let\PY@it=\textit\def\PY@tc##1{\textcolor[rgb]{0.25,0.50,0.50}{##1}}}
\expandafter\def\csname PY@tok@cpf\endcsname{\let\PY@it=\textit\def\PY@tc##1{\textcolor[rgb]{0.25,0.50,0.50}{##1}}}
\expandafter\def\csname PY@tok@c1\endcsname{\let\PY@it=\textit\def\PY@tc##1{\textcolor[rgb]{0.25,0.50,0.50}{##1}}}
\expandafter\def\csname PY@tok@cs\endcsname{\let\PY@it=\textit\def\PY@tc##1{\textcolor[rgb]{0.25,0.50,0.50}{##1}}}

\def\PYZbs{\char`\\}
\def\PYZus{\char`\_}
\def\PYZob{\char`\{}
\def\PYZcb{\char`\}}
\def\PYZca{\char`\^}
\def\PYZam{\char`\&}
\def\PYZlt{\char`\<}
\def\PYZgt{\char`\>}
\def\PYZsh{\char`\#}
\def\PYZpc{\char`\%}
\def\PYZdl{\char`\$}
\def\PYZhy{\char`\-}
\def\PYZsq{\char`\'}
\def\PYZdq{\char`\"}
\def\PYZti{\char`\~}
% for compatibility with earlier versions
\def\PYZat{@}
\def\PYZlb{[}
\def\PYZrb{]}
\makeatother


    % Exact colors from NB
    \definecolor{incolor}{rgb}{0.0, 0.0, 0.5}
    \definecolor{outcolor}{rgb}{0.545, 0.0, 0.0}



    
    % Prevent overflowing lines due to hard-to-break entities
    \sloppy 
    % Setup hyperref package
    \hypersetup{
      breaklinks=true,  % so long urls are correctly broken across lines
      colorlinks=true,
      urlcolor=urlcolor,
      linkcolor=linkcolor,
      citecolor=citecolor,
      }
    % Slightly bigger margins than the latex defaults
    
    \geometry{verbose,tmargin=1in,bmargin=1in,lmargin=1in,rmargin=1in}
    
    

    \begin{document}
    
    
    \maketitle
    
    

    
    \begin{abstract}
    \end{abstract}

 This report will discuss about simulation of tubelight in python using
a basic 1-Dimensional model of tubelight.It analyses the electron
density , Emission Light intensity and the electron phase space as
function of length of the tubelight and discusses the effect of changing
the gas inside the tubelight which is done by changing the threshold
velocity and probability that ioniztion occur and observe their plots.

    \section{Introduction}\label{introduction}



\begin{itemize}
\tightlist
\item
  We use a 1-Dimensional model of the tubelight for simulating in
  python.
\item
  We create a simulation universe. The tube is divided into n sections.
\item
  In each time instant, M electrons are injected. We run the simulation
  for nk turns. The electrons are unable to excite the atoms till they
  have a velocity of u0.
\item
  Beyond this velocity, there is a probability p in each turn that a
  collision will occur and an atom excited.
\item
  Note that the electron's velocity reduces to zero if it collides.
\item
  Assumptions made while creating this simulation universe are as
  follows:
\item
  Uniform electric field is present, that accelerates electrons.
\item
  Electrons are emitted by the cathode with zero energy, and accelerate
  in this field with a = 1
\item
  In our model, we will assume that the relaxation is immediate. The
  electron loses all its energy and the process starts again.
\item
  Electrons reaching the anode are absorbed and lost. Each ``time
  step'', an average of N electrons are introduced at the cathode.
\item
  The actual number of electrons is determined by finding the integer
  part of a random number that is ``normally distributed'' with standard
  deviation of 2 and mean 5.
\item
  Our Aim is to plot the light intensity as a function of position after
  the process has reached steady state.
\item
  And we try to understand the ``dark spaces'' and we will try to find
  them in our simulation.
\end{itemize}

    \section{Python Code :}\label{python-code}

\subsection{Import the Libraries}\label{import-the-libraries}

    \begin{Verbatim}[commandchars=\\\{\}]
{\color{incolor}In [{\color{incolor}1}]:} \PY{c+c1}{\PYZsh{} load libraries and set plot parameters}
        \PY{k+kn}{from} \PY{n+nn}{pylab} \PY{k}{import} \PY{o}{*}
        \PY{o}{\PYZpc{}}\PY{k}{matplotlib} inline
        \PY{k+kn}{import} \PY{n+nn}{mpl\PYZus{}toolkits}\PY{n+nn}{.}\PY{n+nn}{mplot3d}\PY{n+nn}{.}\PY{n+nn}{axes3d} \PY{k}{as} \PY{n+nn}{p3}
        \PY{k+kn}{import} \PY{n+nn}{sys}
        \PY{k+kn}{from}  \PY{n+nn}{tabulate} \PY{k}{import} \PY{n}{tabulate}
        
        
        \PY{k+kn}{from} \PY{n+nn}{IPython}\PY{n+nn}{.}\PY{n+nn}{display} \PY{k}{import} \PY{n}{set\PYZus{}matplotlib\PYZus{}formats}
        \PY{n}{set\PYZus{}matplotlib\PYZus{}formats}\PY{p}{(}\PY{l+s+s1}{\PYZsq{}}\PY{l+s+s1}{pdf}\PY{l+s+s1}{\PYZsq{}}\PY{p}{,} \PY{l+s+s1}{\PYZsq{}}\PY{l+s+s1}{png}\PY{l+s+s1}{\PYZsq{}}\PY{p}{)}
        \PY{n}{plt}\PY{o}{.}\PY{n}{rcParams}\PY{p}{[}\PY{l+s+s1}{\PYZsq{}}\PY{l+s+s1}{savefig.dpi}\PY{l+s+s1}{\PYZsq{}}\PY{p}{]} \PY{o}{=} \PY{l+m+mi}{75}
        
        \PY{n}{plt}\PY{o}{.}\PY{n}{rcParams}\PY{p}{[}\PY{l+s+s1}{\PYZsq{}}\PY{l+s+s1}{figure.autolayout}\PY{l+s+s1}{\PYZsq{}}\PY{p}{]} \PY{o}{=} \PY{k+kc}{False}
        \PY{n}{plt}\PY{o}{.}\PY{n}{rcParams}\PY{p}{[}\PY{l+s+s1}{\PYZsq{}}\PY{l+s+s1}{figure.figsize}\PY{l+s+s1}{\PYZsq{}}\PY{p}{]} \PY{o}{=} \PY{l+m+mi}{12}\PY{p}{,} \PY{l+m+mi}{9}
        \PY{n}{plt}\PY{o}{.}\PY{n}{rcParams}\PY{p}{[}\PY{l+s+s1}{\PYZsq{}}\PY{l+s+s1}{axes.labelsize}\PY{l+s+s1}{\PYZsq{}}\PY{p}{]} \PY{o}{=} \PY{l+m+mi}{18}
        \PY{n}{plt}\PY{o}{.}\PY{n}{rcParams}\PY{p}{[}\PY{l+s+s1}{\PYZsq{}}\PY{l+s+s1}{axes.titlesize}\PY{l+s+s1}{\PYZsq{}}\PY{p}{]} \PY{o}{=} \PY{l+m+mi}{20}
        \PY{n}{plt}\PY{o}{.}\PY{n}{rcParams}\PY{p}{[}\PY{l+s+s1}{\PYZsq{}}\PY{l+s+s1}{font.size}\PY{l+s+s1}{\PYZsq{}}\PY{p}{]} \PY{o}{=} \PY{l+m+mi}{16}
        \PY{n}{plt}\PY{o}{.}\PY{n}{rcParams}\PY{p}{[}\PY{l+s+s1}{\PYZsq{}}\PY{l+s+s1}{lines.linewidth}\PY{l+s+s1}{\PYZsq{}}\PY{p}{]} \PY{o}{=} \PY{l+m+mf}{2.0}
        \PY{n}{plt}\PY{o}{.}\PY{n}{rcParams}\PY{p}{[}\PY{l+s+s1}{\PYZsq{}}\PY{l+s+s1}{lines.markersize}\PY{l+s+s1}{\PYZsq{}}\PY{p}{]} \PY{o}{=} \PY{l+m+mi}{6}
        \PY{n}{plt}\PY{o}{.}\PY{n}{rcParams}\PY{p}{[}\PY{l+s+s1}{\PYZsq{}}\PY{l+s+s1}{legend.fontsize}\PY{l+s+s1}{\PYZsq{}}\PY{p}{]} \PY{o}{=} \PY{l+m+mi}{14}
        \PY{n}{plt}\PY{o}{.}\PY{n}{rcParams}\PY{p}{[}\PY{l+s+s1}{\PYZsq{}}\PY{l+s+s1}{legend.numpoints}\PY{l+s+s1}{\PYZsq{}}\PY{p}{]} \PY{o}{=} \PY{l+m+mi}{2}
        \PY{n}{plt}\PY{o}{.}\PY{n}{rcParams}\PY{p}{[}\PY{l+s+s1}{\PYZsq{}}\PY{l+s+s1}{legend.loc}\PY{l+s+s1}{\PYZsq{}}\PY{p}{]} \PY{o}{=} \PY{l+s+s1}{\PYZsq{}}\PY{l+s+s1}{best}\PY{l+s+s1}{\PYZsq{}}
        \PY{n}{plt}\PY{o}{.}\PY{n}{rcParams}\PY{p}{[}\PY{l+s+s1}{\PYZsq{}}\PY{l+s+s1}{legend.fancybox}\PY{l+s+s1}{\PYZsq{}}\PY{p}{]} \PY{o}{=} \PY{k+kc}{True}
        \PY{n}{plt}\PY{o}{.}\PY{n}{rcParams}\PY{p}{[}\PY{l+s+s1}{\PYZsq{}}\PY{l+s+s1}{legend.shadow}\PY{l+s+s1}{\PYZsq{}}\PY{p}{]} \PY{o}{=} \PY{k+kc}{True}
        \PY{n}{plt}\PY{o}{.}\PY{n}{rcParams}\PY{p}{[}\PY{l+s+s1}{\PYZsq{}}\PY{l+s+s1}{text.usetex}\PY{l+s+s1}{\PYZsq{}}\PY{p}{]} \PY{o}{=} \PY{k+kc}{True}
        \PY{n}{plt}\PY{o}{.}\PY{n}{rcParams}\PY{p}{[}\PY{l+s+s1}{\PYZsq{}}\PY{l+s+s1}{font.family}\PY{l+s+s1}{\PYZsq{}}\PY{p}{]} \PY{o}{=} \PY{l+s+s2}{\PYZdq{}}\PY{l+s+s2}{serif}\PY{l+s+s2}{\PYZdq{}}
        \PY{n}{plt}\PY{o}{.}\PY{n}{rcParams}\PY{p}{[}\PY{l+s+s1}{\PYZsq{}}\PY{l+s+s1}{font.serif}\PY{l+s+s1}{\PYZsq{}}\PY{p}{]} \PY{o}{=} \PY{l+s+s2}{\PYZdq{}}\PY{l+s+s2}{cm}\PY{l+s+s2}{\PYZdq{}}
        \PY{n}{plt}\PY{o}{.}\PY{n}{rcParams}\PY{p}{[}\PY{l+s+s1}{\PYZsq{}}\PY{l+s+s1}{text.latex.preamble}\PY{l+s+s1}{\PYZsq{}}\PY{p}{]} \PY{o}{=} \PY{l+s+sa}{r}\PY{l+s+s2}{\PYZdq{}}\PY{l+s+s2}{\PYZbs{}}\PY{l+s+s2}{usepackage}\PY{l+s+si}{\PYZob{}subdepth\PYZcb{}}\PY{l+s+s2}{, }\PY{l+s+s2}{\PYZbs{}}\PY{l+s+s2}{usepackage}\PY{l+s+si}{\PYZob{}type1cm\PYZcb{}}\PY{l+s+s2}{\PYZdq{}}
\end{Verbatim}


    \subsection{Part B:}\label{part-b}

\begin{itemize}
\tightlist
\item
  To get user update of the parameters of the tubelight model using
  sys.argv.
\item
  The program takes 6 arguments.
\item
  The arguments for this simulation are as follows:

  \begin{itemize}
  \tightlist
  \item
    The arguments must be given in this order when the program is run in
    command line
  \item
    n \(\to\) spatial grid size (No of sections in which length of
    tubelight is divided)
  \item
    M \(\to\) number of electrons injected per turn.
  \item
    Msig \(\to\) Standard deviation of injected electrons
  \item
    nk \(\to\) number of turns to simulate (time for which the
    simulation is running)
  \item
    u0 \(\to\) threshold velocity.(Velocity after which ionization can
    happen)
  \item
    p \(\to\) probability that ionization will occur
  \end{itemize}
\end{itemize}

    \begin{Verbatim}[commandchars=\\\{\}]
{\color{incolor}In [{\color{incolor}2}]:} \PY{c+c1}{\PYZsh{} To get the arguments using sys.argv from command line}
        \PY{k}{if}\PY{p}{(}\PY{n+nb}{len}\PY{p}{(}\PY{n}{sys}\PY{o}{.}\PY{n}{argv}\PY{p}{)}\PY{o}{==}\PY{l+m+mi}{7}\PY{p}{)}\PY{p}{:}
            \PY{n}{n}\PY{p}{,}\PY{n}{M}\PY{p}{,}\PY{n}{Msig}\PY{p}{,}\PY{n}{nk}\PY{p}{,}\PY{n}{u0}\PY{p}{,}\PY{n}{p} \PY{o}{=} \PY{n}{sys}\PY{o}{.}\PY{n}{argv}\PY{p}{[}\PY{l+m+mi}{1}\PY{p}{:}\PY{p}{]}
        \PY{k}{else}\PY{p}{:}
            \PY{n}{n}\PY{o}{=}\PY{l+m+mi}{100}                   \PY{c+c1}{\PYZsh{} spatial grid size.}
            \PY{n}{M}\PY{o}{=}\PY{l+m+mi}{5}                     \PY{c+c1}{\PYZsh{} number of electrons injected per turn.}
            \PY{n}{Msig} \PY{o}{=} \PY{l+m+mi}{2}                \PY{c+c1}{\PYZsh{}Standard deviation of injected electrons}
            \PY{n}{nk}\PY{o}{=}\PY{l+m+mi}{500}                  \PY{c+c1}{\PYZsh{} number of turns to simulate.}
            \PY{n}{u0}\PY{o}{=}\PY{l+m+mi}{5}                    \PY{c+c1}{\PYZsh{} threshold velocity.}
            \PY{n}{p}\PY{o}{=}\PY{l+m+mf}{0.25}                  \PY{c+c1}{\PYZsh{} probability that ionization will occur}
        
        \PY{n}{n}\PY{o}{=}\PY{n+nb}{int}\PY{p}{(}\PY{n}{n}\PY{p}{)}                   
        \PY{n}{M}\PY{o}{=}\PY{n+nb}{int}\PY{p}{(}\PY{n}{M}\PY{p}{)}                   
        \PY{n}{Msig} \PY{o}{=}\PY{n+nb}{int}\PY{p}{(}\PY{n}{Msig}\PY{p}{)}               
        \PY{n}{nk}\PY{o}{=}\PY{n+nb}{int}\PY{p}{(}\PY{n}{nk}\PY{p}{)}           
        \PY{n}{u0}\PY{o}{=}\PY{n+nb}{int}\PY{p}{(}\PY{n}{u0}\PY{p}{)}               
        \PY{n}{p}\PY{o}{=}\PY{n+nb}{float}\PY{p}{(}\PY{n}{p}\PY{p}{)}     
\end{Verbatim}


    \subsection{Part C:}\label{part-c}

\begin{itemize}
\tightlist
\item
  Create vectors to hold the electron information. The dimension should
  be nM (Because in the worst case per turn each section of tubelight
  could have M electrons,and there totally n sections ,so totally nM).

  \begin{itemize}
  \tightlist
  \item
    So the following arrays are declared with nM size to avoid Out of
    memory error.
  \item
    Electron position xx
  \item
    Electron velocity u
  \item
    Displacement in current turn dx
  \item
    To Create them initially with zeros in them.
  \end{itemize}
\item
  Now ,we want to accumulate all the position,velocity and Intensity of
  existing electrons for all turns (nk)

  \begin{itemize}
  \tightlist
  \item
    Intensity of emitted light, I
  \item
    Electron position, X
  \item
    Electron velocity, V
  \item
    When collision takes place,We record that as emitted light in I,but
    we do not know the length of these arrays.Since collision happens
    randomly. So we create them as lists and extend them as required.
  \end{itemize}
\end{itemize}

    \begin{Verbatim}[commandchars=\\\{\}]
{\color{incolor}In [{\color{incolor}3}]:} \PY{c+c1}{\PYZsh{}Initialsing all these arrays with zeros with size nM}
        \PY{n}{N} \PY{o}{=} \PY{n}{n}\PY{o}{*}\PY{n}{M}
        \PY{n}{xx} \PY{o}{=} \PY{n}{np}\PY{o}{.}\PY{n}{zeros}\PY{p}{(}\PY{n}{N}\PY{p}{)}
        \PY{n}{u} \PY{o}{=} \PY{n}{np}\PY{o}{.}\PY{n}{zeros}\PY{p}{(}\PY{n}{N}\PY{p}{)}
        \PY{n}{dx} \PY{o}{=} \PY{n}{np}\PY{o}{.}\PY{n}{zeros}\PY{p}{(}\PY{n}{N}\PY{p}{)}
        
        \PY{c+c1}{\PYZsh{} Declare intensity and electron position and velocity }
        \PY{n}{I} \PY{o}{=} \PY{p}{[}\PY{p}{]}
        \PY{n}{X} \PY{o}{=} \PY{p}{[}\PY{p}{]}
        \PY{n}{V} \PY{o}{=} \PY{p}{[}\PY{p}{]}
\end{Verbatim}


    \subsection{Part D:}\label{part-d}

\begin{itemize}
\tightlist
\item
  To run the simulation for nk turns using forloop and corresponding
  functions given below to update the positions and velocities and
  Intensity of electrons after each turn.
\item
  Equations used for updating the position and velocities of electrons
  are given below:
\item
  Assumptions made are:

  \begin{itemize}
  \tightlist
  \item
    a = 1 and \(\Delta t\) = 1
  \end{itemize}
\end{itemize}

\begin{equation}
  dx_i = u_i\Delta t + \frac{1}{2}a(\Delta)^{2} = u_i + 0.5
  \end{equation}

\begin{equation}
 x_i \leftarrow x_i + dx_i
  \end{equation}

\begin{equation}
 u_i \leftarrow u_i + 1
  \end{equation}

\begin{equation}
 x_i \leftarrow x_i - dx\rho
  \end{equation}

\begin{itemize}
\tightlist
\item
  Here r is a random number between 0 and 1.
\end{itemize}

    \begin{Verbatim}[commandchars=\\\{\}]
{\color{incolor}In [{\color{incolor}4}]:} \PY{c+c1}{\PYZsh{} Function to find the electrons inside the tubelight}
        \PY{k}{def} \PY{n+nf}{findexistElectrons}\PY{p}{(}\PY{n}{xx}\PY{p}{)}\PY{p}{:}
            \PY{c+c1}{\PYZsh{}ii is a vector containing the indices of vector xx }
            \PY{c+c1}{\PYZsh{}that have positive entries.}
            \PY{n}{ii} \PY{o}{=} \PY{n}{where}\PY{p}{(}\PY{n}{xx}\PY{o}{\PYZgt{}}\PY{l+m+mi}{0}\PY{p}{)}
            \PY{k}{return} \PY{n}{ii}\PY{p}{[}\PY{l+m+mi}{0}\PY{p}{]}    
\end{Verbatim}


    \begin{Verbatim}[commandchars=\\\{\}]
{\color{incolor}In [{\color{incolor}5}]:} \PY{c+c1}{\PYZsh{} function to upate the Velocity,Displacement,position of electrons to zero}
        \PY{c+c1}{\PYZsh{} of electrons which hit anode i.e its position \PYZgt{} n(Outside tubelight)}
        \PY{k}{def} \PY{n+nf}{updatePosVel}\PY{p}{(}\PY{n}{xx}\PY{p}{,}\PY{n}{u}\PY{p}{,}\PY{n}{dx}\PY{p}{)}\PY{p}{:}
            \PY{n}{indexes} \PY{o}{=} \PY{n}{where}\PY{p}{(}\PY{n}{xx}\PY{o}{\PYZgt{}}\PY{n}{n}\PY{p}{)}
            \PY{n}{xx}\PY{p}{[}\PY{n}{indexes}\PY{p}{]} \PY{o}{=} \PY{l+m+mi}{0}
            \PY{n}{u}\PY{p}{[}\PY{n}{indexes}\PY{p}{]} \PY{o}{=} \PY{l+m+mi}{0}
            \PY{n}{dx}\PY{p}{[}\PY{n}{indexes}\PY{p}{]} \PY{o}{=} \PY{l+m+mi}{0}
            \PY{k}{return} \PY{n}{xx}\PY{p}{,}\PY{n}{u}\PY{p}{,}\PY{n}{dx}
\end{Verbatim}


    \begin{Verbatim}[commandchars=\\\{\}]
{\color{incolor}In [{\color{incolor}6}]:} \PY{c+c1}{\PYZsh{} function to find the energetic electrons inside tubelight}
        \PY{c+c1}{\PYZsh{} by checking its velocity is more than threshold and returns the indices}
        \PY{c+c1}{\PYZsh{} of those electrons}
        \PY{k}{def} \PY{n+nf}{findEnergeticElectrons}\PY{p}{(}\PY{n}{u}\PY{p}{)}\PY{p}{:}
            \PY{n}{kk} \PY{o}{=} \PY{n}{where}\PY{p}{(}\PY{n}{u} \PY{o}{\PYZgt{}}\PY{o}{=} \PY{n}{u0}\PY{p}{)}\PY{p}{[}\PY{l+m+mi}{0}\PY{p}{]}
            \PY{n}{ll} \PY{o}{=} \PY{n}{where}\PY{p}{(}\PY{n}{rand}\PY{p}{(}\PY{n+nb}{len}\PY{p}{(}\PY{n}{kk}\PY{p}{)}\PY{p}{)}\PY{o}{\PYZlt{}}\PY{o}{=}\PY{n}{p}\PY{p}{)}
            \PY{n}{kl} \PY{o}{=} \PY{n}{kk}\PY{p}{[}\PY{n}{ll}\PY{p}{]}
            \PY{k}{return} \PY{n}{kl}
\end{Verbatim}


    \begin{Verbatim}[commandchars=\\\{\}]
{\color{incolor}In [{\color{incolor}7}]:} \PY{c+c1}{\PYZsh{} function to inject electrons by finding out indexes of electrons }
        \PY{c+c1}{\PYZsh{} whose position is less than 0 .}
        \PY{k}{def} \PY{n+nf}{toInjectElectron}\PY{p}{(}\PY{n}{xx}\PY{p}{,}\PY{n}{m}\PY{p}{)}\PY{p}{:}
            \PY{n}{inj} \PY{o}{=} \PY{n}{where}\PY{p}{(}\PY{n}{xx} \PY{o}{\PYZlt{}}\PY{o}{=} \PY{l+m+mi}{0}\PY{p}{)}
            \PY{k}{return} \PY{n}{inj}
\end{Verbatim}


    \begin{Verbatim}[commandchars=\\\{\}]
{\color{incolor}In [{\color{incolor}8}]:} \PY{c+c1}{\PYZsh{} For loop to run the simulation nk times}
        \PY{k}{for} \PY{n}{j} \PY{o+ow}{in} \PY{n+nb}{range}\PY{p}{(}\PY{l+m+mi}{1}\PY{p}{,}\PY{n}{nk}\PY{p}{)}\PY{p}{:}
        
            \PY{n}{ii} \PY{o}{=} \PY{n}{findexistElectrons}\PY{p}{(}\PY{n}{xx}\PY{p}{)}      \PY{c+c1}{\PYZsh{}to find electrons inside tubelight}
            \PY{n}{X}\PY{o}{.}\PY{n}{extend}\PY{p}{(}\PY{n}{xx}\PY{p}{[}\PY{n}{ii}\PY{p}{]}\PY{o}{.}\PY{n}{tolist}\PY{p}{(}\PY{p}{)}\PY{p}{)}        \PY{c+c1}{\PYZsh{}Storing active electrons in X each turn}
            \PY{n}{V}\PY{o}{.}\PY{n}{extend}\PY{p}{(}\PY{n}{u}\PY{p}{[}\PY{n}{ii}\PY{p}{]}\PY{o}{.}\PY{n}{tolist}\PY{p}{(}\PY{p}{)}\PY{p}{)}         \PY{c+c1}{\PYZsh{}Storing velocities of these electrons}
        
            \PY{n}{dx}\PY{p}{[}\PY{n}{ii}\PY{p}{]} \PY{o}{=} \PY{n}{u}\PY{p}{[}\PY{n}{ii}\PY{p}{]} \PY{o}{+} \PY{l+m+mf}{0.5}             \PY{c+c1}{\PYZsh{}displacement of electron from kinematics}
            \PY{n}{xx}\PY{p}{[}\PY{n}{ii}\PY{p}{]} \PY{o}{=} \PY{n}{xx}\PY{p}{[}\PY{n}{ii}\PY{p}{]} \PY{o}{+} \PY{n}{dx}\PY{p}{[}\PY{n}{ii}\PY{p}{]}         \PY{c+c1}{\PYZsh{}updating position by adding the displacement}
            \PY{n}{u}\PY{p}{[}\PY{n}{ii}\PY{p}{]}  \PY{o}{=} \PY{n}{u}\PY{p}{[}\PY{n}{ii}\PY{p}{]} \PY{o}{+} \PY{l+m+mi}{1}               \PY{c+c1}{\PYZsh{}updating the velocity at new position}
            \PY{c+c1}{\PYZsh{}update position,velocity of electrons which hit anode}
            \PY{n}{xx}\PY{p}{,}\PY{n}{u}\PY{p}{,}\PY{n}{dx} \PY{o}{=} \PY{n}{updatePosVel}\PY{p}{(}\PY{n}{xx}\PY{p}{,}\PY{n}{u}\PY{p}{,}\PY{n}{dx}\PY{p}{)}  
            \PY{n}{kl} \PY{o}{=} \PY{n}{findEnergeticElectrons}\PY{p}{(}\PY{n}{u}\PY{p}{)}   \PY{c+c1}{\PYZsh{}indexes of energetic electrons}
            \PY{n}{u}\PY{p}{[}\PY{n}{kl}\PY{p}{]} \PY{o}{=} \PY{l+m+mi}{0}                        \PY{c+c1}{\PYZsh{}set velocity of energetic electrons to zero}
            \PY{n}{xx}\PY{p}{[}\PY{n}{kl}\PY{p}{]} \PY{o}{=} \PY{n}{xx}\PY{p}{[}\PY{n}{kl}\PY{p}{]}\PY{o}{\PYZhy{}}\PY{n}{dx}\PY{p}{[}\PY{n}{kl}\PY{p}{]}\PY{o}{*}\PY{n}{random}\PY{p}{(}\PY{p}{)}  \PY{c+c1}{\PYZsh{}updating position after collision}
        
            \PY{n}{I}\PY{o}{.}\PY{n}{extend}\PY{p}{(}\PY{n}{xx}\PY{p}{[}\PY{n}{kl}\PY{p}{]}\PY{o}{.}\PY{n}{tolist}\PY{p}{(}\PY{p}{)}\PY{p}{)}        \PY{c+c1}{\PYZsh{}Storing ionized electrons positions}
            \PY{n}{m}\PY{o}{=}\PY{n+nb}{int}\PY{p}{(}\PY{n}{randn}\PY{p}{(}\PY{p}{)}\PY{o}{*}\PY{n}{Msig}\PY{o}{+}\PY{n}{M}\PY{p}{)}            \PY{c+c1}{\PYZsh{}Actual no of injected electrons}
            \PY{n}{inj} \PY{o}{=} \PY{n}{toInjectElectron}\PY{p}{(}\PY{n}{xx}\PY{p}{,}\PY{n}{m}\PY{p}{)}     \PY{c+c1}{\PYZsh{}indexes where to inject electrons}
            \PY{n}{xx}\PY{p}{[}\PY{n}{inj}\PY{p}{]} \PY{o}{=} \PY{l+m+mi}{1}                      \PY{c+c1}{\PYZsh{}Set their position with 1}
\end{Verbatim}


    \subsection{Part E:}\label{part-e}

\begin{itemize}
\tightlist
\item
  To plot three graphs:
\item
  Histogram of Electron density
\item
  Histogram of Emission Intensity
\item
  Electron phase space plot
\item
  And to analyse the graphs obtained
\end{itemize}

    \begin{Verbatim}[commandchars=\\\{\}]
{\color{incolor}In [{\color{incolor}9}]:} \PY{n}{fig1} \PY{o}{=} \PY{n}{figure}\PY{p}{(}\PY{p}{)}
        \PY{n}{ax1} \PY{o}{=} \PY{n}{fig1}\PY{o}{.}\PY{n}{add\PYZus{}subplot}\PY{p}{(}\PY{l+m+mi}{111}\PY{p}{)}
        \PY{n}{ax1}\PY{o}{.}\PY{n}{hist}\PY{p}{(}\PY{n}{X}\PY{p}{,}\PY{n}{n}\PY{p}{,} \PY{n}{alpha}\PY{o}{=}\PY{l+m+mf}{0.9}\PY{p}{,} \PY{n}{histtype}\PY{o}{=}\PY{l+s+s1}{\PYZsq{}}\PY{l+s+s1}{bar}\PY{l+s+s1}{\PYZsq{}}\PY{p}{,} \PY{n}{ec}\PY{o}{=}\PY{l+s+s1}{\PYZsq{}}\PY{l+s+s1}{black}\PY{l+s+s1}{\PYZsq{}}\PY{p}{)}
        \PY{n}{ax1}\PY{o}{.}\PY{n}{legend}\PY{p}{(}\PY{p}{)}
        \PY{n}{xticks}\PY{p}{(}\PY{n+nb}{range}\PY{p}{(}\PY{l+m+mi}{0}\PY{p}{,}\PY{n}{n}\PY{o}{+}\PY{l+m+mi}{1}\PY{p}{,}\PY{l+m+mi}{10}\PY{p}{)}\PY{p}{)}
        \PY{n}{title}\PY{p}{(}\PY{l+s+sa}{r}\PY{l+s+s2}{\PYZdq{}}\PY{l+s+s2}{Figure 1: Histogram of electron density}\PY{l+s+s2}{\PYZdq{}}\PY{p}{)}
        \PY{n}{xlabel}\PY{p}{(}\PY{l+s+s2}{\PYZdq{}}\PY{l+s+s2}{\PYZdl{}n\PYZdl{}}\PY{l+s+s2}{\PYZdq{}}\PY{p}{)}
        \PY{n}{ylabel}\PY{p}{(}\PY{l+s+s2}{\PYZdq{}}\PY{l+s+s2}{No of Electrons}\PY{l+s+s2}{\PYZdq{}}\PY{p}{)}
        \PY{n}{grid}\PY{p}{(}\PY{p}{)}
        \PY{n}{savefig}\PY{p}{(}\PY{l+s+s2}{\PYZdq{}}\PY{l+s+s2}{Figure1.jpg}\PY{l+s+s2}{\PYZdq{}}\PY{p}{)}
\end{Verbatim}


    \begin{center}
    \adjustimage{max size={0.9\linewidth}{0.9\paperheight}}{output_15_0.pdf}
    \end{center}
    { \hspace*{\fill} \\}
    
    \subsubsection{Results and Discussion:}\label{results-and-discussion}

\begin{itemize}
\tightlist
\item
  From the plot,we observe that it electron density decreases with
  length of the tubelight
\end{itemize}

    \begin{Verbatim}[commandchars=\\\{\}]
{\color{incolor}In [{\color{incolor}10}]:} \PY{n}{fig2} \PY{o}{=} \PY{n}{figure}\PY{p}{(}\PY{p}{)}
         \PY{n}{ax2} \PY{o}{=} \PY{n}{fig2}\PY{o}{.}\PY{n}{add\PYZus{}subplot}\PY{p}{(}\PY{l+m+mi}{111}\PY{p}{)}
         \PY{n}{Idata} \PY{o}{=} \PY{n}{ax2}\PY{o}{.}\PY{n}{hist}\PY{p}{(}\PY{n}{I}\PY{p}{,}\PY{n}{n}\PY{p}{,}\PY{n}{alpha}\PY{o}{=}\PY{l+m+mf}{0.9}\PY{p}{,} \PY{n}{histtype}\PY{o}{=}\PY{l+s+s1}{\PYZsq{}}\PY{l+s+s1}{bar}\PY{l+s+s1}{\PYZsq{}}\PY{p}{,} \PY{n}{ec}\PY{o}{=}\PY{l+s+s1}{\PYZsq{}}\PY{l+s+s1}{black}\PY{l+s+s1}{\PYZsq{}}\PY{p}{)}
         \PY{n}{ax2}\PY{o}{.}\PY{n}{legend}\PY{p}{(}\PY{p}{)}
         \PY{n}{xticks}\PY{p}{(}\PY{n+nb}{range}\PY{p}{(}\PY{l+m+mi}{0}\PY{p}{,}\PY{n}{n}\PY{o}{+}\PY{l+m+mi}{1}\PY{p}{,}\PY{l+m+mi}{10}\PY{p}{)}\PY{p}{)}
         \PY{n}{title}\PY{p}{(}\PY{l+s+sa}{r}\PY{l+s+s2}{\PYZdq{}}\PY{l+s+s2}{Figure 2: Emission Intensity along the length of Tubelight}\PY{l+s+s2}{\PYZdq{}}\PY{p}{)}
         \PY{n}{xlabel}\PY{p}{(}\PY{l+s+s2}{\PYZdq{}}\PY{l+s+s2}{\PYZdl{}n\PYZdl{}}\PY{l+s+s2}{\PYZdq{}}\PY{p}{)}
         \PY{n}{ylabel}\PY{p}{(}\PY{l+s+s2}{\PYZdq{}}\PY{l+s+s2}{I}\PY{l+s+s2}{\PYZdq{}}\PY{p}{)}
         \PY{n}{grid}\PY{p}{(}\PY{p}{)}
         \PY{n}{savefig}\PY{p}{(}\PY{l+s+s2}{\PYZdq{}}\PY{l+s+s2}{Figure2.jpg}\PY{l+s+s2}{\PYZdq{}}\PY{p}{)}
\end{Verbatim}


    \begin{center}
    \adjustimage{max size={0.9\linewidth}{0.9\paperheight}}{output_17_0.pdf}
    \end{center}
    { \hspace*{\fill} \\}
    
    \subsubsection{Results and Discussion:}\label{results-and-discussion}

\begin{itemize}
\tightlist
\item
  From the plot of Emission intensity we observe that the region upto 10
  is where electrons are building up their energy. Beyond that is a
  region where the emission decays, representing the fewer energetic
  electrons that reached there before colliding. At the next peak.But
  this is a diffuse peak since the zero energy location of different
  electrons is different.
\end{itemize}

    \begin{Verbatim}[commandchars=\\\{\}]
{\color{incolor}In [{\color{incolor}11}]:} \PY{n}{fig3} \PY{o}{=} \PY{n}{figure}\PY{p}{(}\PY{p}{)}
         \PY{n}{ax3} \PY{o}{=} \PY{n}{fig3}\PY{o}{.}\PY{n}{add\PYZus{}subplot}\PY{p}{(}\PY{l+m+mi}{111}\PY{p}{)}
         \PY{n}{ax3}\PY{o}{.}\PY{n}{plot}\PY{p}{(}\PY{n}{X}\PY{p}{,}\PY{n}{V}\PY{p}{,}\PY{l+s+s1}{\PYZsq{}}\PY{l+s+s1}{go}\PY{l+s+s1}{\PYZsq{}}\PY{p}{)}
         \PY{n}{ax3}\PY{o}{.}\PY{n}{legend}\PY{p}{(}\PY{p}{)}
         \PY{n}{title}\PY{p}{(}\PY{l+s+sa}{r}\PY{l+s+s2}{\PYZdq{}}\PY{l+s+s2}{Figure 3: Electron Phase space plot}\PY{l+s+s2}{\PYZdq{}}\PY{p}{)}
         \PY{n}{xlabel}\PY{p}{(}\PY{l+s+s2}{\PYZdq{}}\PY{l+s+s2}{\PYZdl{}x\PYZdl{}}\PY{l+s+s2}{\PYZdq{}}\PY{p}{)}
         \PY{n}{xticks}\PY{p}{(}\PY{n+nb}{range}\PY{p}{(}\PY{l+m+mi}{0}\PY{p}{,}\PY{n}{n}\PY{o}{+}\PY{l+m+mi}{1}\PY{p}{,}\PY{l+m+mi}{10}\PY{p}{)}\PY{p}{)}
         \PY{n}{ylabel}\PY{p}{(}\PY{l+s+s2}{\PYZdq{}}\PY{l+s+s2}{\PYZdl{}v\PYZdl{}}\PY{l+s+s2}{\PYZdq{}}\PY{p}{)}
         \PY{n}{grid}\PY{p}{(}\PY{p}{)}
         \PY{n}{savefig}\PY{p}{(}\PY{l+s+s2}{\PYZdq{}}\PY{l+s+s2}{Figure3.jpg}\PY{l+s+s2}{\PYZdq{}}\PY{p}{)}
\end{Verbatim}


    \begin{center}
    \adjustimage{max size={0.9\linewidth}{0.9\paperheight}}{output_19_0.png}
    \end{center}
    { \hspace*{\fill} \\}
    
    \subsubsection{Results and Discussion:}\label{results-and-discussion}

\begin{itemize}
\tightlist
\item
  From the plot of electron phase space we observe that it follows a
  velocity varies parabolically with increase in length of the tubelight
  i.e position of electrons.
\end{itemize}

    \subsection{Part F:}\label{part-f}

\begin{itemize}
\tightlist
\item
  To print the table of Intensity as a function of position of electrons
  in the tubelight
\item
  The intensity data is obtained from histogram function which returns a
  tuple consisting of Intensity array,bins.
\item
  The second output of tuple from hist function actually gives the
  dividing positions between bins, and so has a dimension one greater
  than the population array.
\item
  Convert to mid point values by:
\end{itemize}

\begin{equation}
xpos=0.5(bins[0:-1]+bins[1:])
   \end{equation}

\begin{itemize}
\tightlist
\item
  This averages the vector containing left positions of all the bins and
  the vector containing the right positions of all the bins.
\end{itemize}

    \begin{Verbatim}[commandchars=\\\{\}]
{\color{incolor}In [{\color{incolor}12}]:} \PY{n}{bins} \PY{o}{=} \PY{n}{Idata}\PY{p}{[}\PY{l+m+mi}{1}\PY{p}{]}
         \PY{n}{xpos} \PY{o}{=}\PY{l+m+mf}{0.5}\PY{o}{*}\PY{p}{(}\PY{n}{bins}\PY{p}{[}\PY{l+m+mi}{0}\PY{p}{:}\PY{o}{\PYZhy{}}\PY{l+m+mi}{1}\PY{p}{]}\PY{o}{+}\PY{n}{bins}\PY{p}{[}\PY{l+m+mi}{1}\PY{p}{:}\PY{p}{]}\PY{p}{)}
         
         \PY{n}{table} \PY{o}{=} \PY{n+nb}{zip}\PY{p}{(}\PY{n}{xpos}\PY{p}{,}\PY{n}{Idata}\PY{p}{[}\PY{l+m+mi}{0}\PY{p}{]}\PY{p}{)}
         
         \PY{n}{headers} \PY{o}{=} \PY{p}{[}\PY{l+s+s2}{\PYZdq{}}\PY{l+s+s2}{Position}\PY{l+s+s2}{\PYZdq{}}\PY{p}{,}\PY{l+s+s2}{\PYZdq{}}\PY{l+s+s2}{Count}\PY{l+s+s2}{\PYZdq{}}\PY{p}{]}
         \PY{c+c1}{\PYZsh{}tabulating Intensity of electrons Vs position inside the tubelight}
         \PY{n+nb}{print}\PY{p}{(}\PY{l+s+s2}{\PYZdq{}}\PY{l+s+s2}{Intensity data:}\PY{l+s+s2}{\PYZdq{}}\PY{p}{)}
         \PY{n+nb}{print}\PY{p}{(}\PY{n}{tabulate}\PY{p}{(}\PY{n}{table}\PY{p}{,}\PY{n}{tablefmt}\PY{o}{=}\PY{l+s+s2}{\PYZdq{}}\PY{l+s+s2}{fancy\PYZus{}grid}\PY{l+s+s2}{\PYZdq{}}\PY{p}{,}\PY{n}{headers}\PY{o}{=}\PY{n}{headers}\PY{p}{)}\PY{p}{)}
\end{Verbatim}

\begin{table}[H]
\textbf{Tabulating values of Intensity Vs Position}\par\medskip
\begin{tabular}{|c|c|}
\hline
Position & Count \\ 
	\hline
9.45695 & 524 \\
10.3659 & 467 \\
11.2749 & 404 \\
12.1838 & 539 \\
13.0928 & 321 \\
14.0017 & 333 \\
14.9107 & 221 \\
15.8196 & 189 \\
16.7286 & 317 \\
17.6376 & 444 \\
18.5465 & 261 \\
19.4555 & 248 \\
20.3644 & 280 \\
21.2734 & 244 \\
22.1823 & 234 \\
23.0913 & 418 \\
24.0003 & 366 \\
24.9092 & 268 \\
25.8182 & 271 \\
26.7271 & 261 \\
27.6361 & 245 \\
28.545 & 221 \\
29.454 & 284 \\
30.3629 & 292 \\
31.2719 & 280 \\
32.1809 & 260 \\
33.0898 & 291 \\
33.9988 & 272 \\
34.9077 & 250 \\
35.8167 & 228 \\
36.7256 & 230 \\
37.6346 & 239 \\
38.5436 & 296 \\
39.4525 & 294 \\
40.3615 & 285 \\
41.2704 & 263 \\
42.1794 & 256 \\
43.0883 & 228 \\
43.9973 & 207 \\
44.9062 & 243 \\
45.8152 & 260 \\
46.7242 & 259 \\
47.6331 & 299 \\
48.5421 & 283 \\
49.451 & 262 \\
50.36 & 245 \\
51.2689 & 285 \\
52.1779 & 250 \\
53.0869 & 260 \\
53.9958 & 259 \\
54.9048 & 224 \\
55.8137 & 254 \\
56.7227 & 252 \\
57.6316 & 236 \\
58.5406 & 230 \\
59.4495 & 275 \\
60.3585 & 249 \\
61.2675 & 247 \\
62.1764 & 252 \\
63.0854 & 204 \\
63.9943 & 209 \\
64.9033 & 269 \\
65.8122 & 262 \\
66.7212 & 220 \\
67.6302 & 217 \\
68.5391 & 244 \\
69.4481 & 271 \\
70.357 & 252 \\
71.266 & 240 \\
72.1749 & 237 \\
73.0839 & 263 \\
73.9929 & 230 \\
74.9018 & 242 \\
75.8108 & 236 \\
76.7197 & 227 \\
77.6287 & 225 \\
78.5376 & 260 \\
79.4466 & 245 \\
80.3555 & 246 \\
81.2645 & 273 \\
82.1735 & 230 \\
83.0824 & 237 \\
83.9914 & 236 \\
84.9003 & 233 \\
85.8093 & 248 \\
86.7182 & 226 \\
87.6272 & 225 \\
88.5362 & 222 \\
89.4451 & 242 \\
90.3541 & 233 \\
91.263 & 227 \\
92.172 & 248 \\
93.0809 & 224 \\
93.9899 & 238 \\
94.8988 & 188 \\
95.8078 & 153 \\
96.7168 & 121 \\
97.6257 & 93 \\
98.5347 & 48 \\
99.4436 & 17 \\
	\hline
\end{tabular}
\end{table}

    \subsection{Results and Discussion:}\label{results-and-discussion}

\begin{itemize}
\tightlist
\item
  As we increase u0 and p that is threshold velocity and probability of
  ionization peak intensity will increase and the position at which the
  peak occurs also increased and velocity at that point also increases
  according to plot and equations we obtained.
\end{itemize}


    % Add a bibliography block to the postdoc
    
    
    
    \end{document}
